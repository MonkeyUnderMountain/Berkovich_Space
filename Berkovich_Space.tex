\documentclass[sectionlevel=book]{noteformyself}


\input{Accessories/notation.tex}                 % 加载符号定义文件
\addbibresource{Accessories/ref.bib}             % 添加参考文献文件
\newcommand{\Yang}[1]{\textcolor{red}{Yang: #1}}


\title{Berkovich Space}
\author{Tianle Yang}
\date{\today}
\authorpage{\href{https://www.tianleyang.com}{https://www.tianleyang.com}}

% available on the mode "\sectionlevel=chapter" or "\sectionlevel=book".
\setCJKfamilyfont{lxgwwenkai}{LXGW WenKai} % 定义霞鹜文楷,若未安装,请去掉相关代码编译或使用其他字体
\coversentence{\CJKfamily{lxgwwenkai}“”}
\coverimage{} % 封面图片
\covertitlefont{Allura} % 封面标题字体, 若未安装,请去掉相关代码编译或使用其他字体
% \coverlinecolor{red} % 封面线条颜色
% \covertextcolor{green} % 封面文字颜色

% available on the mode "\sectionlevel=book".
\authoremail{\href{mailto:loveandjustice@88.com}{loveandjustice@88.com}}
\texsource{\href{https://github.com/MonkeyUnderMountain/Berkovich_Space.git}{github.com/MonkeyUnderMountain/Berkovich\_Space}}
\version{0.1.0}


\begin{document}

    \maketitle

    \tableofcontents % 生成目录

    The main references are \cite{Ber90,BGR84}.

    \chapter{Commutative banach algebras}
       \section{Semi-normed Rings and Modules}


\subsection{Semi-normed algebraic structures}

    \begin{definition}\label{def:semi-normed_abelian_group}
        Let \(M\) be an abelian group.
        A \emph{semi-norm} on \(M\) is a function \(\|\cdot\|: M \to \bbR_+\) such that
        \begin{itemize}
            \item \(\|0\| = 0\);
            \item \(\forall x,y \in M, \|x + y\| \leq \|x\| + \|y\|\).
        \end{itemize}
        If we further have \(\|x\| = 0 \iff x = 0\), then we say \(\|\cdot\|\) is a \emph{norm}.
        A \emph{semi-normed abelian group} (resp. \emph{normed abelian group}) is an abelian group equipped with a semi-norm (resp. norm).
    \end{definition}

    \begin{definition}\label{def:bounded_semi-norm}
        Let \(\|\cdot\|_1\) and \(\|\cdot\|_2\) be two semi-norms on an abelian group \(M\).
        We say \(\|\cdot\|_1\) is \emph{bounded} by \(\|\cdot\|_2\) if there exists a constant \(C > 0\) such that \(\forall x \in M, \|x\|_1 \leq C\|x\|_2\).
    \end{definition}

    \begin{remark}\label{rmk:semi-norms_bounded_by_each_other_induces_the_same_topologies}
        If two semi-norms (resp. norms) on an abelian group \(M\) are bounded by each other, then they induce the same topology on \(M\).
    \end{remark}

    \begin{definition}\label{def:residue_semi-norm}
        Let \(M\) be a semi-normed abelian group and \(N \subseteq M\) be a subgroup.
        The \emph{residue semi-norm} on the quotient group \(M/N\) is defined as
        \[
            \|x + N\|_{M/N} = \inf_{y \in N} \|x + y\|_M.
        \]
    \end{definition}

    \begin{remark}\label{rmk:residue_semi-norm}
        The residue semi-norm is a norm if and only if \(N\) is closed in \(M\).
    \end{remark}

    \begin{definition}\label{def:bounded_and_admissible_homomorphism}
        Let \(M\) and \(N\) be two semi-normed abelian groups.
        A group homomorphism \(f: M \to N\) is called \emph{bounded} if there exists a constant \(C > 0\) such that \(\forall x \in M, \|f(x)\|_N \leq C\|x\|_M\).
        
        A bounded homomorphism \(f: M \to N\) is called \emph{admissible} if the induced isomorphism \(M/\ker f \to \Image f\) is an isometry, i.e., \(\forall x \in M, \|f(x)\|_N = \inf_{y \in \ker f} \|x + y\|_M\).
    \end{definition}

    \begin{definition}\label{def:semi-normed_rings}
        Let \(R\) be a ring (commutative with unity).
        A \emph{semi-norm} on \(R\) is a semi-norm \(\|\cdot\|\) on the underlying abelian group of \(R\) such that \(\forall x,y \in R, \|xy\| \leq \|x\|\|y\|\) and \(\|1\| = 1\).
        A \emph{semi-normed ring} is a ring equipped with a semi-norm.
    \end{definition}

    \begin{definition}\label{def:multiplicative_and_power_multiplicative_semi-norm}
        A semi-norm \(\|\cdot\|\) on a ring \(R\) is called \emph{multiplicative} if \(\forall x,y \in R, \|xy\| = \|x\|\|y\|\).
        It is called \emph{power-multiplicative} if \(\forall x \in R, \|x^n\| = \|x\|^n\) for all integers \(n \geq 1\).
        % A power-multiplicative semi-norm is also called \emph{uniform}.
    \end{definition}

    \begin{definition}\label{def:semi-normed_modules}
        Let \((R, \|\cdot\|_R)\) be a normed ring.
        A \emph{semi-normed \(R\)-module} is a pair \((M, \|\cdot\|_M)\) where \(M\) is an \(R\)-module and \(\|\cdot\|_M\) is a semi-norm on the underlying abelian group of \(M\) such that there exists \(C > 0\) with \(\forall a \in R, x \in M, \|ax\|_M \leq C \|a\|_R \|x\|_M\).
    \end{definition}

    One can talk about boundedness, admissibility and residue semi-norms in the contexts of semi-normed rings and semi-normed modules similar to those in semi-normed abelian groups.


    % \Yang{To be continued...}

\subsection{banach rings}

    \begin{definition}\label{def:complete_semi-normed_abelian_group}
        A (semi-)norm on an abelian group \(M\) induces a (pseudo-)metric \(d(x,y) = \|x - y\|\) on \(M\).
        A (semi-)normed abelian group \(M\) is called \emph{complete} if it is complete as a (pseudo-)metric space.
    \end{definition}

    \begin{definition}\label{def:banach_ring}
        A \emph{banach ring} is a complete normed ring.
    \end{definition}

    \begin{definition}\label{def:completion_of_normed_algebraic_structures}
        Let \((A, \|\cdot\|_A)\) be a (semi-)normed algebraic structure, e.g., a (semi-)normed abelian group, a (semi-)normed ring, or a (semi-)normed module.
        The \emph{completion} of \(A\), denoted by \(\widehat{A}\), is the completion of \(A\) as a (pseudo-)metric space.
        Since \(A\) is dense in its completion, the algebraic operations and (semi-)norms on \(A\) can be uniquely extended to the completion.
        % \Yang{To be continued.}
    \end{definition}
    
    Let \(R\) be a normed ring and \(M,N\) be semi-normed \(R\)-modules.
    There is a natural semi-norm on the tensor product \(M \otimes_R N\) defined as
    \[
        \|z\|_{M \otimes_R N} = \inf \left\{ \sum_{i} \|x_i\|_M \|y_i\|_N : z = \sum_i x_i \otimes y_i, x_i \in M, y_i \in N \right\}.
    \]

    \begin{definition}\label{def:complete_tensor_product}
        Let \(R\) be a complete normed ring and \(M,N\) complete semi-normed \(R\)-modules.
        The \emph{complete tensor product} \(M \widehat{\otimes}_R N\) is defined as the completion of the semi-normed \(R\)-module \(M \otimes_R N\).
    \end{definition}

    \begin{definition}\label{def:spectral_radius_on_banach_rings}
        Let \(R\) be a banach ring.
        For each \(f \in R\), the \emph{spectral radius} of \(f\) is defined as
        \[
            \rho(f) = \lim_{n \to \infty} \|f^n\|^{1/n}.
        \]
    \end{definition}

    \Yang{Since , \(\rho(f)\) exists.}

    \begin{definition}\label{def:uniform_banach_ring}
        A banach ring \(R\) is called \emph{uniform} if its norm is power-multiplicative.
    \end{definition}

    \begin{proposition}\label{prop:spectral_radius_defines_a_power-multiplicative_semi-norm}
        Let \((R,\|\cdot\|)\) be a banach ring.
        The spectral radius \(\rho(\cdot)\) defines a power-multiplicative semi-norm on \(R\) that is bounded by \(\|\cdot\|\).
    \end{proposition}
    \begin{proof}
        \Yang{To be continued.}
    \end{proof}

    \begin{definition}\label{def:quasi_nilpotent_element}
        Let \(R\) be a banach ring.
        An element \(f \in R\) is called \emph{quasi-nilpotent} if \(\rho(f) = 0\).
        All quasi-nilpotent elements of \(R\) form an ideal, denoted by \(\Qnil(R)\).
    \end{definition}

    \begin{definition}\label{def:uniformization_of_banach_rings}
        Let \(R\) be a banach ring.
        The \emph{uniformization} of \(R\), denoted by \(R \to R^u\), is the banach ring with the universal property among all bounded homomorphisms from \(R\) to uniform banach rings.
        \Yang{To be continued.}
    \end{definition}

    \begin{proposition}\label{prop:the_uniformization_of_banach_rings_given_by_spectral_radius}
        Let \(R\) be a banach ring.
        The completion of \(R/\Qnil(R)\) with respect to the spectral radius \(\rho(\cdot)\) is the uniformization of \(R\).
    \end{proposition}
    \begin{proof}
        \Yang{To be continued.}
    \end{proof}

    % \Yang{To be continued...}


\subsection{Complete field}

    \begin{definition}\label{def:valuation_field}
        A multiplicative norm on a field is also called an \emph{absolute value}.
        A \emph{valuation field} is a field equipped with an absolute value.
    \end{definition}

    \begin{remark}\label{rmk:additive_and_multiplicative_valuation_on_a_field}
        Let \(\kk\) be a field.
        Recall that a \emph{valuation} on \(\kk\) is a function \(v: \kk^\times \to \bbR\) such that
        \begin{itemize}
            \item \(\forall x,y \in \kk^\times, v(xy) = v(x) + v(y)\);
            \item \(\forall x,y \in \kk^\times, v(x + y) \geq \min\{v(x), v(y)\}\).
        \end{itemize}
        We can extend \(v\) to the whole field \(\kk\) by defining \(v(0) = +\infty\).
        Fix a real number \(\varepsilon \in (0,1)\).
        Then \(v\) induces an absolute value \(|\cdot|_v: \kk \to \bbR_+\) defined by \(|x|_v = \varepsilon^{v(x)}\) for each \(x \in \kk\).

        In some literature, the valuation \(v\) is called an \emph{additive valuation} and the induced absolute value \(|\cdot|_v\) is called a \emph{multiplicative valuation}.
        In this note, the term \emph{valuation} always refers to the additive valuation.
    \end{remark}

    \begin{definition}\label{def:non-archimedean_valuation}
        A valuation field \((\kk, |\cdot|)\) is called \emph{non-Archimedean} if \(\forall x,y \in \kk, |x + y| \leq \max\{|x|, |y|\}\), i.e., the norm satisfies the ultrametric inequality.
        Otherwise, it is called \emph{Archimedean}.
    \end{definition}
    
    \begin{definition}\label{def:complete_field}
        A \emph{complete field} is a valuation field which is complete as a metric space.
    \end{definition}

    \begin{lemma}\label{lem:ring_of_integers_of_non-archimedean_field}
        Let \(\kk\) be a non-Archimedean complete field.
        Then the set \(\kk^\circ = \{x \in \kk : |x| \leq 1\}\) is a subring of \(\kk\), which is a local ring.
        Moreover, the set \(\kk^{\circ\circ} = \{x \in \kk : |x| < 1\}\) is the maximal ideal of \(\kk^\circ\).
    \end{lemma}

    \begin{definition}\label{def:ring_of_integers_of_non-archimedean_field}
        Let \(\kk\) be a non-Archimedean complete field.
        The subring \(\kk^\circ\) is called the \emph{ring of integers} of \(\kk\).
        The set \(\kk^{\circ\circ} = \{x \in \kk : |x| < 1\}\) is the maximal ideal of \(\kk^\circ\).
        The residue field \(\calk_\kk = \kk^\circ / \kk^{\circ\circ}\) is called the \emph{residue field} of \(\kk\).
        \Yang{To be revised.}
    \end{definition}

    % Notation test \(\calk_\kk\) or \(\widetilde{\kk}\) or \(\rkk_\kk\) for the residue field of \(\kk\).
    % \(\calk_{\bbQ_p}\)


\subsection{Examples}

    \begin{example}\label{eg:trivial_normed_rings}
        Let \(R\) be arbitrary ring.
        The \emph{trivial norm} on \(R\) is defined as \(\|x\| = 0\) if \(x = 0\) and \(\|x\| = 1\) if \(x \neq 0\).
        The ring \(R\) equipped with the trivial norm is a normed ring.
    \end{example}

    \begin{example}\label{eg:C_and_R_as_complete_fields}
        The fields \(\bbC\) and \(\bbR\) equipped with the usual absolute value are complete fields.
    \end{example}

    \begin{example}\label{eg:p-adic_fields_as_complete_fields}
        The field \(\bbQ_p\) of \(p\)-adic numbers equipped with the \(p\)-adic norm is a complete non-Archimedean field.
    \end{example}

    \begin{example}\label{eg:ring_of_absolutely_convergent_power_series_as_banach_rings}
        Let \(R\) be a banach ring and \(r > 0\) be a real number.
        We define the ring of absolutely convergent power series over \(\kk\) with radius \(r\) as
        \[ R\left<T/r\right> \coloneqq \left\{\sum_{n=0}^{\infty} a_n T^n \in R[[T]] : \sum_{n=0}^{\infty} \|a_n\| r^n < \infty \right\}. \]
        Equipped with the norm \(\|\sum_{n=0}^{\infty} a_n T^n\| = \sum_{n=0}^{\infty} \|a_n\| r^n\), the ring \(R\left<T/r\right>\) is a banach ring.

        When \(R = \kk\) is a 
        \Yang{To be checked.}
    \end{example}

    \begin{example}\label{eg:Tate_algebras_with_Gauss_norm_as_banach_algebras}
        Let \(\kk\) be a non-Archimedean complete field.
        We define
        \[ \kk\{T_1/r_1, \ldots, T_n/r_n\} \coloneqq \left\{\sum_{I \in \bbN^n} a_I T^I \in \kk[[T_1, \ldots, T_n]] : \lim_{|I| \to \infty} |a_I| r^I = 0 \right\}, \]
        where \(r = (r_1, \ldots, r_n)\) is an n-tuple of positive real numbers, \(T^I = T_1^{i_1} \cdots T_n^{i_n}\) for \(I = (i_1, \ldots, i_n)\), and \(|I| = i_1 + \cdots + i_n\).
        Equipped with the norm \(\|\sum_{I \in \bbN^n} a_I T^I\| = \sup_{I \in \bbN^n} |a_I| r^I\), the affinoid \(\kk\)-algebra \(\kk\{T_1/r_1, \ldots, T_n/r_n\}\) is a banach \(\kk\)-algebra.
        This is called the \emph{Tate algebra} over \(\kk\) with polyradius \(r\) equipped with the \emph{Gauss norm}.
        We will denote \(\kk\{\underline{T/r}\} = \kk\{T_1/r_1, \ldots, T_n/r_n\}\) for simplicity.
    \end{example}
    \Yang{To be continued...}
       \section{Affinoid algebras}

\subsection{The first properties}

    \begin{definition}\label{def:affinoid_algebras}
        Let \(\kk\) be a non-archimedean field. 
        A banach \(\kk\)-algebra \(A\) is called a \emph{ affinoid \(\kk\)-algebra} if there exists an admissible surjective homomorphism
        \[
            \varphi: \kk\{ r_1^{-1} T_1, \ldots, r_n^{-1} T_n \} \twoheadrightarrow A
        \]
        for some \(n \in \bbN\) and \(r_1, \ldots, r_n \in \bbR_{>0}\).

        If one can choose \(r_1 = \cdots = r_n = 1\), then we say that \(A\) is a \emph{strict affinoid \(\kk\)-algebra}.
    \end{definition}

    \begin{definition}\label{def:restricted_Laurent_series}
        Let \(\kk\) be a non-archimedean field.
        We define the \emph{ring of restricted Laurent series} over \(\kk\) as 
        \[ \KK_r = \bfL_{\kk,r} = \left\{\sum_{n \in \bbZ} a_n T^n : a_n \in \kk, \lim_{|n| \to \infty} |a_n| r^n = 0 \right\} \]
        equipped with the norm
        \[ \|f\| = \sup_{n \in \bbZ} |a_n| r^n. \]
    \end{definition}

    \Yang{Is \(\KK_r\) always a field?}
    \Yang{Do we have \(\bfL_{\kk,r} = \Frac (\kk\{T/r\})\)?}

    \begin{proposition}\label{prop:restricted_Laurent_series_is_a_field_when_r_is_not_root_of_absolute_value}
        Let \(\kk\) be a non-archimedean field.
        If \(r \notin \sqrt{|\kk^\times|}\), then \(\KK_r\) is a complete non-archimedean field with non-trivial absolute value extending that of \(\kk\).
    \end{proposition}

    \begin{proposition}\label{prop:affinoid_algebra_is_noetherian_and_ideal_is_clsoed}
        Let \(A\) be an affinoid \(\kk\)-algebra.
        Then \(A\) is noetherian, and every ideal of \(A\) is closed.
    \end{proposition}

    \begin{proposition}\label{prop:the_norm_on_affinoid_is_bounded_by_spectral_radius}
        Let \(A\) be an affinoid \(\kk\)-algebra.
        Then there exists a constant \(C > 0\) and \(N > 0\) such that for all \(f \in A\) and \(n \geq N\), we have
        \[
            \|f^n\| \leq C \rho(f)^n.   
        \]
    \end{proposition}

    \begin{proposition}\label{prop:affinoid_algebra_is_strict_when_radius_in_the_radical_of_valuation_set}
        Let \(A\) be an affinoid \(\kk\)-algebra.
        If and only if \(\rho(f) \in \sqrt{|\kk|}\) for all \(f \in A\), then \(A\) is strict.
        \Yang{To be complete.}
    \end{proposition}


\subsection{Finite banach module}

    There are three different categories of finite modules over an affinoid algebra \(A\):
    \begin{itemize}
        \item The category \(\Banmod_A\) of finite banach \(A\)-modules with \(A\)-linear maps as morphisms.
        \item The category \(\Banmod_A^b\) of finite banach \(A\)-modules with bounded \(A\)-linear maps as morphisms.
        \item The category \(\module_A\) of finite \(A\)-modules with all \(A\)-linear maps as morphisms.
    \end{itemize}

    \begin{theorem}\label{thm:equivalent_of_categroy_of_finite_banach_module_and_algebraic_module}
        Let \(A\) be an affinoid \(\kk\)-algebra.
        Then the category of finite banach \(A\)-modules with bounded \(A\)-linear maps as morphisms is equivalent to the category of finite \(A\)-modules with \(A\)-linear maps as morphisms.
        \Yang{To be revised.}
    \end{theorem}

    For simplicity, we will just write \(\mod_A\) to denote the category of finite banach \(A\)-modules with bounded \(A\)-linear maps as morphisms.

    \chapter{Affinoid spaces}
       \section{Spectrum}

\subsection{Definition}

    \begin{definition}\label{def:spectrum_of_Banach_rings}
        Let \(R\) be a Banach ring.
        The \emph{spectrum} \(\scrM(R)\) of \(R\) is defined as the set of all multiplicative semi-norms on \(R\) that are bounded with respect to the given norm on \(R\).
        For every point \(x \in \scrM(R)\), we denote the corresponding multiplicative semi-norm by \(|\cdot|_x\).
        We equip \(\scrM(R)\) with the weakest topology such that for each \(f \in R\), the evaluation map \(\scrM(R) \to \bbR_{\geq 0}\), defined by \(x \mapsto |f|_x\eqqcolon f(x)\), is continuous.
    \end{definition}
    
    \begin{definition}\label{def:pullback_of_ring_homomorphism_of_banach_rings_on_spectrum}
        Let \(\varphi: R \to S\) be a bounded ring homomorphism of Banach rings.
        The \emph{pullback} map \(\scrM(\varphi): \scrM(S) \to \scrM(R)\) is defined by \(\scrM(\varphi)(x) = x \circ \varphi: f \mapsto |\varphi(f)|_x\) for each \(x \in \scrM(S)\).
        % \Yang{To be revised.}
    \end{definition}

    % For \(x \in \scrM(R)\), \Yang{the kernel of the multiplicative semi-norm \(|\cdot|_x\) is a closed prime ideal of \(R\)}, denoted by \(\wp_x\).
    % The semi-norm \(|\cdot|_x\) induces a multiplicative norm on the residue field \(\rkk(x) = \Frac(R/\wp_x)\), denoted by \(|\cdot|_{x}\) as well. 

    \begin{definition}\label{def:character_of_banach_rings}
        Let \(R\) be a Banach ring.
        A \emph{character} of \(R\) is a bounded ring homomorphism \(\chi: R \to K\), where \(K\) is a completed field.
        Two characters \(\chi_1: R \to K_1\) and \(\chi_2: R \to K_2\) are said to be \emph{equivalent} if there exists a commutative diagram of bounded ring homomorphisms
        \[
            \begin{tikzcd}
                & R \arrow[dl, "\chi_1"'] \arrow[dr, "\chi_2"] \arrow[d] & \\
                K_1 & K \arrow[l,hook'] \arrow[r,hook] & K_2
            \end{tikzcd}
        \]
        for some completed field \(K\).
        % \Yang{This is wrong, need to revised.}
    \end{definition}

    \begin{proposition}\label{prop:spectrum_of_banach_rings_and_equivalence_class_of_characters}
        Let \(R\) be a Banach ring.
        The spectrum \(\scrM(R)\) is in bijection with the equivalence classes of characters of \(R\).
    \end{proposition}
    \begin{proof}
        \Yang{To be completed}
    \end{proof}

    \begin{proposition}\label{prop:map_from_M_R_to_Spec_R}
        Let \(R\) be a Banach ring.
        For each \(x \in \scrM(R)\), let \(\wp_x\) be the kernel of the multiplicative semi-norm \(|\cdot|_x\).
        Then \(\wp_x\) is a closed prime ideal of \(R\), and \(x \mapsto \wp_x\) defines a continuous map from \(\scrM(R)\) to \(\Spec(R)\) equipped with the Zariski topology.
    \end{proposition}
    \begin{proof}
        \Yang{To be completed}
    \end{proof}

    \begin{definition}
        Let \(R\) be a Banach ring.
        For each \(x \in \scrM(R)\), the \emph{completed residue field} at the point \(x\) is defined as the completion of the residue field \(\rkk(x) = \Frac(R/\wp_x)\) with respect to the multiplicative norm induced by the semi-norm \(|\cdot|_x\), denoted by \(\scrH(x)\).
    \end{definition}

    \begin{definition}\label{def:Gelfand_transform_of_banach_rings}
        Let \(R\) be a Banach ring.
        The \emph{Gel'fand transform} of \(R\) is the bounded ring homomorphism
        \[
            \Gamma: R \to \prod_{x \in \scrM(R)} \scrH(x), \quad f \mapsto (f(x))_{x \in \scrM(R)},
        \]
        where the norm on the product \(\prod_{x \in \scrM(R)} \scrH(x)\) is given by the supremum norm.
    \end{definition}

    \begin{proposition}\label{prop:Gelfand_transform_and_the_uniformization_of_a_banach_ring}
        The Gel'fand transform \(\Gamma: R \to \prod_{x \in \scrM(R)} \scrH(x)\) of a Banach ring \(R\) factors through the uniformization \(R^u\) of \(R\), and the induced map \(R^u \to \prod_{x \in \scrM(R)} \scrH(x)\) is an isometric embedding.
        \Yang{To be checked.}
    \end{proposition}

    \begin{theorem}\label{thm:spectrum_of_Banach_rings_is_nonempty_compact_Hausdorff}
        Let \(R\) be a Banach ring.
        The spectrum \(\scrM(R)\) is a nonempty compact Hausdorff space.
    \end{theorem}
    \begin{proof}
        \Yang{To be continued.}
    \end{proof}

    \begin{lemma}\label{lem:spectrum_of_product_of_completed_fields}
        Let \(\{K_i\}_{i \in I}\) be a family of completed fields.
        Consider the Banach ring \(R = \prod_{i \in I} K_i\) equipped with the product norm.
        The spectrum \(\scrM(R)\) is homeomorphic to the Stone-\v{C}ech compactification of the discrete space \(I\).
    \end{lemma}

    \begin{remark}\label{rmk:some_fact_about_the_Stone_Cech_compactification}
        The Stone-\v{C}ech compactification of a discrete space is the largest compact Hausdorff space in which the original space can be densely embedded.
        \Yang{To be checked.}
    \end{remark}

    \begin{proposition}\label{prop:the_Galois_action_on_the_spectrum_of_banach_rings}
        Let \(K/k\) be a Galois extension of complete fields, and let \(R\) be a Banach \(k\)-algebra.
        The Galois group \(\Gal(K/k)\) acts on the spectrum \(\scrM(R \widehat{\otimes}_k K)\) via
        \[
            g \cdot x: f \mapsto |(1 \otimes g^{-1})(f)|_x
        \]
        for each \(g \in \Gal(K/k)\), \(x \in \scrM(R \widehat{\otimes}_k K)\) and \(f \in R \widehat{\otimes}_k K\).
        Moreover, the natural map \(\scrM(R \widehat{\otimes}_k K) \to \scrM(R)\) induces a homeomorphism
        \[
            \scrM(R \widehat{\otimes}_k K) / \Gal(K/k) \xrightarrow{\sim} \scrM(R).
        \]
        \Yang{To be checked.}
    \end{proposition}


\subsection{Examples}

    \begin{example}\label{eg:spectrum_of_valuation_field}
        Let \((\kk, |\cdot|)\) be a complete valuation field.
        The spectrum \(\scrM(\kk)\) consists of a single point corresponding to the given absolute value \(|\cdot|\) on \(\kk\).
        \Yang{To be checked.}
    \end{example}

    \begin{example}\label{eg:spectrum_of_Z_with_absolute_value_norm}
        Consider the Banach ring \((\bbZ, \|\cdot\|)\) with \(\|\cdot\| = |\cdot|_\infty\) is the usual absolute value norm on \(\bbZ\).
        Let \(|\cdot|_p\) denote the \(p\)-adic norm for each prime number \(p\), i.e., \(|n|_p = p^{-v_p(n)}\) for each \(n \in \bbZ\), where \(v_p(n)\) is the \(p\)-adic valuation of \(n\).
        The spectrum 
        \[ \scrM(\bbZ) = \{|\cdot|_{\infty}^\varepsilon \colon \varepsilon \in (0,1]\} \cup \{| \cdot |_p^\alpha \colon p \text{ is prime}, \alpha \in (0,\infty] \} \cup \{|\cdot|_0\}, \]
        where \(|a|_p^\infty := \lim_{\alpha \to \infty} |a|_p^\alpha\) for each \(a \in \bbZ\) and \(|\cdot|_0\) is the trivial norm on \(\bbZ\).
        \Yang{To be checked.}
    \end{example}

    % \begin{example}\label{eg:spectrum_of_Tate_algebras}
    \paragraph{Spectrum of Tate algebra in one variable} Let \(\kk\) be a complete non-archimedean field, and let \(A = \kk\{T/r\}\).
    We list some types of points in the spectrum \(\scrM(A)\).

    For each \(a \in \kk\) with \(|a| \leq r\), we have the \emph{type I} point \(x_a\) corresponding to the evaluation at \(a\), i.e., \(|f|_{x_a} := |f(a)|\) for each \(f \in A\).
    For each closed disk \(E = E(a,s) := \{b \in \kk : |b - a| \leq s\}\) with center \(a \in \kk\) and radius \(s \leq r\), we have the point \(x_{a,s}\) corresponding to the multiplicative semi-norm defined by
    \[ |f|_{x_E} := \sup_{b \in E(a,s)} |f(b)| \]
    for each \(f \in A\).
    If \(s \in |\kk^\times|\), then the point \(x_E\) is called a \emph{type II} point; otherwise, it is called a \emph{type III} point.
    
    Let \(\{E^{(s)}\}_s\) be a family of closed disks in \(\kk\) such that \(E^{(s)}\) is of radius \(s\), \(E^{(s_1)} \subsetneq E^{(s_2)}\) for any \(s_1 < s_2\) and \(\bigcap_s E^{(s)} = \emptyset\).
    Then we have the point \(x_{\{E^{(s)}\}}\) corresponding to the multiplicative semi-norm defined by
    \[ |f|_{x_{\{E^{(s)}\}}} := \inf_s |f|_{x_{E^{(s)}}} \]
    for each \(f \in A\).
    Such a point is called a \emph{type IV} point.

    \Yang{To be completed.}
    % \end{example}

    \begin{proposition}\label{prop:four_types_of_points_in_spectrum_of_Tate_algebra_in_one_variable}
        Let \(\kk\) be a complete non-archimedean field, and let \(r > 0\) be a positive real number.
        Consider the Tate algebra \(\kk\{r^{-1}T\}\) equipped with the Gauss norm.
        The points in the spectrum \(\scrM(\kk\{r^{-1}T\})\) can be classified into four types as described above.
        \Yang{To be checked}
    \end{proposition}
    \begin{proof}
        \Yang{To be completed.}
    \end{proof}

    \begin{proposition}\label{prop:the_complete_residue_field_of_all_four_types_points_in__spectrum_of_Tate_algebra_in_one_variable}
        Let \(\kk\) be a complete non-archimedean field, and let \(r > 0\) be a positive real number.
        Consider the Tate algebra \(\kk\{r^{-1}T\}\) equipped with the Gauss norm.
        The completed residue fields of the four types of points in the spectrum \(\scrM(\kk\{r^{-1}T\})\) are described as follows:
        \begin{itemize}
            \item For a type I point \(x_a\) with \(a \in \kk\) and \(|a| \leq r\), the completed residue field \(\scrH(x_a)\) is isomorphic to \(\kk\).
            \item For a type II point \(x_{a,s}\) with \(a \in \kk\) and \(s \in |\kk^\times|\), the completed residue field \(\scrH(x_{a,s})\) is isomorphic to the field of Laurent series over the residue field \(\calk_\kk\), i.e., \(\calk_\kk((t))\).
            \item For a type III point \(x_{a,s}\) with \(a \in \kk\) and \(s \notin |\kk^\times|\), the completed residue field \(\scrH(x_{a,s})\) is isomorphic to a transcendental extension of \(\calk_\kk\) of degree one.
            \item For a type IV point \(x_{\{E^{(s)}\}}\), the completed residue field \(\scrH(x_{\{E^{(s)}\}})\) is isomorphic to a transcendental extension of \(\calk_\kk\) of infinite degree.
        \end{itemize}
        \Yang{To be checked.}
    \end{proposition}

    \begin{example}\label{eg:completed_residue_field_in_spectrum_of_Tate_algebra_in_one_variable_over_Q_p}
        The completed residue field \(\scrH(x_a)\) for a type I point \(x_a\) with \(a \in \kk\) and \(|a| \leq r\) is isomorphic to \(\kk\).
        \Yang{To be complete.}
    \end{example}
       \section{Affinoid domains}


    \begin{definition}\label{def:affinoid_domains}
        Let \(A\) be a \(\kk\)-affinoid algebra, and let \(X = \scrM(A)\) be the associated affinoid space.
        A closed subset \(V \subseteq X\) is called an \emph{affinoid domain} if there exists a \(\kk\)-affinoid algebra \(A_V\) and a morphism of \(\kk\)-affinoid algebras \(\varphi: A \to A_V\) satisfying the following universal property:
        for every bounded homomorphism of \(\kk\)-affinoid algebras \(\psi: A \to B\) such that the induced map on spectra \(\scrM(\psi): \scrM(B) \to X\) has its image contained in \(V\), there exists a unique bounded homomorphism \(\theta: A_V \to B\) such that the following diagram commutes:
        \[
        \begin{tikzcd}
            & A_V  \arrow[dr, gray, "\theta"] & \\
            A \arrow[ur, "\varphi"] \arrow[rr, "\psi"'] & & B
        \end{tikzcd}
        \]
        In this case, we say that \(V\) is represented by the affinoid algebra \(A_V\).
    \end{definition}
    \begin{slogan}
        A closed subset \(V \subset X\) is an affinoid domain if the functor ``\(\Mor(-, V)\)'' is representable.
    \end{slogan}

    \Yang{Why we consider closed subset rather that open subset?}


    \begin{construction}\label{cons:weierstrass_domain}
        Let \(f=(f_1,\ldots,f_n),g = (g_1,\ldots,g_m)\) be two tuples of elements in \(A\).
        Set \(p=(p_1,\ldots,p_n)\) and \(q=(q_1,\ldots,q_m)\) be two tuples of positive real numbers.
        We define the following closed subsets of \(X\):
        \[ X\left(\underline{f/p},\underline{q/g}\right) := \left\{ x \in X \colon |f_i(x)| \le p_i, |g_j(x)| \ge q_j, 1 \le i \le n, 1 \le j \le m \right\}. \]
        Such a closed subset is called a \emph{Weierstrass domain} of \(X\).
        Moreover, we can define a \(\kk\)-affinoid algebra
        \[ A\left\langle \underline{f/p},\underline{q/g} \right\rangle := A\left\langle \frac{f_1}{p_1},\ldots,\frac{f_n}{p_n}, \frac{q_1}{g_1},\ldots,\frac{q_m}{g_m} \right\rangle, \]
        which is the quotient of the Tate algebra
        \[ A\left\langle T_1,\ldots,T_n,S_1,\ldots,S_m \right\rangle \]
        by the ideal generated by the elements \(p_i T_i - f_i\) for \(1 \le i \le n\) and \(g_j S_j - q_j\) for \(1 \le j \le m\).
        There is a natural bounded homomorphism \(\varphi: A \to A\langle \underline{f/p},\underline{q/g} \rangle\) induced by the inclusion.
        It can be shown that the closed subset \(X(\underline{f/p},\underline{q/g})\) is an affinoid domain represented by the affinoid algebra \(A\langle \underline{f/p},\underline{q/g} \rangle\).
        \Yang{To be checked}
    \end{construction}

    \begin{construction}\label{cons:rational_domain}
        Let \(f=(f_1,\ldots,f_n),g\) be elements in \(A\) such that the ideal generated by them is the whole algebra \(A\).
        Set \(p=(p_1,\ldots,p_n)\) be a tuple of positive real numbers.
        We define the following closed subset of \(X\):
        \[ X\left(\underline{f/p},g\right) := \left\{ x \in X \colon |f_i(x)| \le p_i |g(x)|, 1 \le i \le n \right\}. \]
        Such a closed subset is called a \emph{rational domain} of \(X\).
        Moreover, we can define a \(\kk\)-affinoid algebra
        \[ A\left\langle \underline{f/p},g^{-1} \right\rangle := A\left\langle \frac{f_1}{p_1 g},\ldots,\frac{f_n}{p_n g} \right\rangle, \]
        which is the quotient of the Tate algebra
        \[ A\left\langle T_1,\ldots,T_n \right\rangle \]
        by the ideal generated by the elements \(p_i g T_i - f_i\) for \(1 \le i \le n\).
        There is a natural bounded homomorphism \(\varphi: A \to A\langle \underline{f/p},g^{-1} \rangle\) induced by the inclusion.
        It can be shown that the closed subset \(X(\underline{f/p},g)\) is an affinoid domain represented by the affinoid algebra \(A\langle \underline{f/p},g^{-1} \rangle\).
        \Yang{To be checked}
    \end{construction}

    \begin{proposition}\label{prop:affinoid_domain_is_flat_over_base}
        Let \(A\) be a \(\kk\)-affinoid algebra, and let \(X = \scrM(A)\) be the associated affinoid space.
        Let \(V \subseteq X\) be an affinoid domain represented by the \(\kk\)-affinoid algebra \(A_V\).
        Then the natural bounded homomorphism \(\varphi: A \to A_V\) is flat.

        We have \(\scrM(A_V) \cong V\).
    \end{proposition}

    \chapter{Analytic spaces}
    % \appendix

    % \chapter{}

    \printbibliography[heading=bibintoc, title={References}] % 打印参考文献

\end{document}
