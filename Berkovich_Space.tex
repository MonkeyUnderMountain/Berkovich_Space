\documentclass[sectionlevel=book]{noteformyself}
\usepackage{draftwatermark}               % 水印宏包


\input{Accessories/notation.tex}                 % 加载符号定义文件
\addbibresource{Accessories/ref.bib}             % 添加参考文献文件
\newcommand{\Yang}[1]{\textcolor{red}{Yang: #1}}


\title{Berkovich Space}
\author{Tianle Yang}
\date{\today}
\authorpage{\href{https://www.tianleyang.com}{https://www.tianleyang.com}}

% available on the mode "\sectionlevel=chapter" or "\sectionlevel=book".
\setCJKfamilyfont{lxgwwenkai}{LXGW WenKai} % 定义霞鹜文楷,若未安装,请去掉相关代码编译或使用其他字体
\coversentence{\CJKfamily{lxgwwenkai}“”}
\coverimage{} % 封面图片
\covertitlefont{Allura} % 封面标题字体, 若未安装,请去掉相关代码编译或使用其他字体
% \coverlinecolor{red} % 封面线条颜色
% \covertextcolor{green} % 封面文字颜色

% available on the mode "\sectionlevel=book".
\authoremail{\href{mailto:loveandjustice@88.com}{loveandjustice@88.com}}
\texsource{\href{https://github.com/MonkeyUnderMountain/Berkovich_Space.git}{github.com/MonkeyUnderMountain/Berkovich\_Space}}
\version{0.1.0}


\begin{document}

   \maketitle

   \frontmatter

   \chapter*{Preface}
   \addcontentsline{toc}{chapter}{Preface}

   This document provides an introduction to Berkovich spaces, a fundamental concept in non-archimedean analytic geometry. 
   The theory of Berkovich spaces offers a powerful framework for studying analytic varieties over non-archimedean valued fields, providing a geometric approach that bridges algebraic and analytic methods.

   % The material is organized to build systematically from the foundational concepts of commutative Banach algebras through affinoid spaces to the general theory of analytic spaces. 
   % Each chapter includes detailed proofs and examples to illustrate the key concepts and techniques.
   The main references are \cite{Ber90,BGR84}.


   \tableofcontents % 生成目录

   \mainmatter

   \chapter{Non-archimedean analysis}
      \section{Valuation fields}


    \begin{definition}\label{def:valuation_field}
        Let \(\kk\) be a field.
        An \emph{absolute value} on \(\kk\) is a function \(\|\cdot|:\kk\to\bbR_{\ge0}\) satisfying the following properties for all \(x,y\in\kk\):
        \begin{enumerate}
            \item \(\|x\|=0\) if and only if \(x=0\);
            \item \(\|xy\|=\|x\|\cdot\|y\|\);
            \item \(\|x+y\|\leq\|x\|+\|y\|\).
        \end{enumerate}
        A field \(\kk\) equipped with an absolute value \(\|\cdot\|\) is called a \emph{valuation field}.
    \end{definition}

    \begin{remark}\label{rmk:additive_and_multiplicative_valuation_on_a_field}
        Let \(\kk\) be a field.
        Recall that a \emph{valuation} on \(\kk\) is a function \(v: \kk^\times \to \bbR\) such that
        \begin{itemize}
            \item \(\forall x,y \in \kk^\times, v(xy) = v(x) + v(y)\);
            \item \(\forall x,y \in \kk^\times, v(x + y) \geq \min\{v(x), v(y)\}\).
        \end{itemize}
        We can extend \(v\) to the whole field \(\kk\) by defining \(v(0) = +\infty\).
        Fix a real number \(\varepsilon \in (0,1)\).
        Then \(v\) induces an absolute value \(|\cdot|_v: \kk \to \bbR_+\) defined by \(|x|_v = \varepsilon^{v(x)}\) for each \(x \in \kk\).

        In some literature, the valuation \(v\) is called an \emph{additive valuation} and the induced absolute value \(|\cdot|_v\) is called a \emph{multiplicative valuation}.
        In this note, the term \emph{valuation} always refers to the additive valuation.
    \end{remark}

    \begin{definition}\label{def:complete_valuation_field}
        Let \((\kk,\|\cdot\|)\) be a valuation field.
        We say that \(\kk\) is \emph{complete} if the metric \(d(x,y) := \|x - y\|\) makes \(\kk\) a complete metric space.
    \end{definition}

    \begin{lemma}\label{lem:completion_of_valuation_field}
        Let \((\kk,\|\cdot\|)\) be a valuation field.
        Let \((\widehat{\kk},\|\cdot\|)\) be its completion as a metric space.
        Then the operations of addition and multiplication on \(\kk\) can be extended to \(\widehat{\kk}\) uniquely, making \((\widehat{\kk},\|\cdot\|)\) a complete valuation field containing \(\kk\) as a dense subfield.
    \end{lemma}

    % \begin{theorem}[Classification of absolute values on \(\bbQ\)]\label{thm:classification_absolute_values_Q}
    %     Let \(\|\cdot\|\) be an absolute value on \(\bbQ\).
    %     Then \(\|\cdot\|\) is either equivalent to the usual absolute value \(|\cdot|_\infty\) or to a \(p\)-adic absolute value \(|\cdot|_p\) for some prime \(p\).
    %     \Yang{The name is to be added.}
    % \end{theorem}

    \begin{definition}\label{def:spherically_complete}
        A valuation field \((\kk,\|\cdot\|)\) is called \emph{spherically complete} if every decreasing sequence of closed balls in \(\kk\) has a non-empty intersection.
    \end{definition}

      \section{Ultra-metric spaces}

    We will use \(B(x,r)\) (resp. \(E(x,r)\)) to denote the open ball (resp. closed ball) with center \(x\) and radius \(r\).
    % We will use \(E(x,r)\) to denote the closed ball with center \(x\) and radius \(r\).

    \begin{definition}\label{def:ultra-metric_space}
        A metric space \((X,d)\) is called an \emph{ultra-metric space} if its metric \(d\) satisfies the \emph{strong triangle inequality}:
        \[ d(x,z) \leq \max\{d(x,y), d(y,z)\},\quad\forall x,y,z\in X. \]
    \end{definition}

    If \((\kk,\|\cdot\|)\) is a non-archimedean field, then the metric \(d(x,y) := \|x-y\|\) on \(\kk\) makes \((\kk,d)\) an ultra-metric space.

    \begin{proposition}\label{prop:all_triangles_in_ultra-metric_space_are_isosceles}
        Let \((X,d)\) be an ultra-metric space.
        Then for any \(x,y,z \in X\), at least two of the three distances \(d(x,y), d(y,z), d(z,x)\) are equal.
        And the third distance is less than or equal to the common value of the other two.
    \end{proposition}
    \begin{proof}
        Suppose that \(d(x,y) \geq d(y,z)\).
        By the strong triangle inequality, we have
        \[ d(x,z) \leq \max\{d(x,y), d(y,z)\} = d(x,y). \]
        On the other hand, by the strong triangle inequality again, we have
        \[ d(x,y) \leq \max\{d(x,z), d(z,y)\} = \max\{d(x,z), d(y,z)\} \leq d(x,y). \]
        This shows that \(d(x,y) = \max\{d(x,z), d(y,z)\}\).
        Thus either \(d(x,z) = d(x,y) \geq d(y,z)\) or \(d(y,z) = d(x,y) \geq d(x,z)\).
        %\Yang{To be continued.}
    \end{proof}

    \begin{proposition}\label{prop:balls_in_ultra-metric_space_form_a_tree}
        Let \((X,d)\) be an ultra-metric space.
        Let \(D_i\) be (open or closed) ball in \(X\) for \(i=1,2\).
        If \(D_1 \cap D_2 \neq \emptyset\), then either \(D_1 \subseteq D_2\) or \(D_2 \subseteq D_1\).
    \end{proposition}
    \begin{proof}
        Suppose that \(D_i\) has center \(x_i\) and radius \(r_i\) for \(i=1,2\).
        Let \(y \in D_1 \cap D_2\).
        We have 
        \[ d(x_1,x_2) \leq \max\{d(x_1,y), d(y,x_2)\}. \]
        Without loss of generality, we may assume that \(d(x_1,x_2) \leq d(x_1,y)\).
        It follows that \(x_2 \in D_1\) since \(d(x_1,y) < r_1\) (or \(\leq r_1\)).

        If there exists \(z \in D_2 \setminus D_1\), we claim that \(D_1 \subseteq D_2\).
        We have \(d(x_1,z) > d(x_1,x_2)\).
        Then by \cref{prop:all_triangles_in_ultra-metric_space_are_isosceles},
        \[ r_1 \leq d(x_1,z) = d(x_2,z) \leq r_2. \]
        In particular, if \(D_2\) is an open ball, then we have strict inequality \(r_1 < r_2\).
        For any \(w \in D_1\), we have 
        \[ d(x_2,w) \leq \max\{d(x_2,x_1), d(x_1,w)\} \leq r_1 \leq r_2. \]
        Thus \(w \in D_2\) whatever \(D_2\) is open or closed, and it shows that \(D_1 \subseteq D_2\).
    \end{proof}

    \begin{proposition}\label{prop:all_balls_in_ultra-metric_space_are_clopen}
        Let \((X,d)\) be an ultra-metric space.
        Then both \(B(x,r)\) and \(E(x,r)\) are closed and open subsets of \(X\) for any \(x \in X\) and \(r > 0\).
    \end{proposition}
    \begin{proof}
        We show that the sphere \(S(x,r) := \{y \in X \mid d(x,y) = r\}\) is open in \(X\).
        Note that if \(y \in S(x,r)\), then for any \(r' < r\), we have \(B(y,r') \cap E(x,r) \neq \emptyset\) and \(x \in E(x,r) \setminus B(y,r')\).
        Thus by \cref{prop:balls_in_ultra-metric_space_form_a_tree}, we have \(B(y,r') \subseteq E(x,r)\).
        If \(B(y,r') \cap B(x,r) \neq \emptyset\), then by \cref{prop:balls_in_ultra-metric_space_form_a_tree} again, we have \(B(y,r') \subseteq B(x,r)\).
        However, \(y \in B(y,r') \setminus B(x,r)\), a contradiction.
        Thus \(B(y,r') \subseteq E(x,r) \setminus B(x,r) = S(x,r)\).
        It yields that \(S(x,r) = \bigcup_{y \in S(x,r)} B(y,r/2)\) is open in \(X\).

        Since \(E(x,r) = B(x,r) \cup S(x,r)\) and \(B(x,r) = E(x,r) \setminus S(x,r)\), both \(B(x,r)\) and \(E(x,r)\) are open and closed in \(X\).
    \end{proof}

    \begin{corollary}\label{prop:ultra-metric_space_is_totally_disconnected}
        Let \((X,d)\) be an ultra-metric space.
        Then \(X\) is totally disconnected, i.e., the only connected subsets of \(X\) are the set with at most one point.
    \end{corollary}
    \begin{proof}
        Suppose that \(S \subset X\) has at least two distinct points \(x,y \in S\).
        Let \(r := d(x,y) > 0\).
        Consider the open ball \(B(x,r/2)\).
        By \cref{prop:all_balls_in_ultra-metric_space_are_clopen}, \(B(x,r/2)\) is both open and closed in \(X\).
        Thus \(B(x,r/2) \cap S\) is both open and closed in \(S\), however, it is non-empty and not equal to \(S\) since it contains \(x\) but not \(y\).
        This shows that \(S\) is disconnected.
    \end{proof}

      \section{Residue fields and reductions}

\subsection{Recover non-archimedean complete fields algebraically}
    
    In this subsection, let \(\kk\) be a non-archimedean field.
    Set \(I_{r,<} := \{x \in \kk\colon \|x\| < r\}\) and \(I_{r,\leq} := \{x \in \kk\colon \|x\| \leq r\}\) for each \(r \in (0,1)\).

    \begin{proposition}\label{prop:ideal_of_integers_ring_of_NA_field}
        The sets \(I_{r,<}\) and \(I_{r,\leq}\) are ideals of the ring of integers \(\kk^\circ\).
        Conversely, any ideal of \(\kk^\circ\) is of the form \(I_{r,<}\) or \(I_{r,\leq}\) for some \(r \in (0,1)\).
        \Yang{To be checked.}
    \end{proposition}
    \begin{proof}
        \Yang{To be checked.}
    \end{proof}

    \begin{proposition}\label{prop:recover_complete_non-archimedean_fields_from_projective_limits}
        We have 
        \[ \widehat{\kk}^\circ \cong \varprojlim_{r \in (0,1)} \kk^\circ / I_r. \]
        \Yang{To be checked.}
    \end{proposition}

    \begin{proposition}\label{prop:locally_compact_NA_field_iff_it_is_pro-finite}
        Let \(\kk\) be a non-archimedean field.
        Then \(\kk\) is totally bounded iff \(\kk^\circ / I_r\) is finite for each \(r \in (0,1)\).
        Moreover, if \(\kk\) is complete, then it is locally compact iff \(\kk^\circ/I_r\) is finite for each \(r \in (0,1)\).
        \Yang{To be checked.}
    \end{proposition}
    \begin{slogan}
        ``Locally compact \(\iff\) pro-finite.''
    \end{slogan}
    \begin{proof}
        
    \end{proof}

    \begin{proposition}\label{prop:integers_ring_of_NA_field_is_noetherian_iff_discrete_valuation}
        The ring \(\kk^\circ\) is noetherian iff \(\kk\) is a discrete valuation field.
        \Yang{To be revised.}
    \end{proposition}

    \begin{proposition}\label{prop:locally_compact_NA_fields_iff_finite_residue_field_and_discrete_valuation}
        Let \(\kk\) be a complete non-archimedean field. 
        Then \(\kk\) is locally compact iff \(\kk\) is a discrete valuation field and its residue field \(\calk_\kk\) is finite.
        \Yang{To be checked.}
    \end{proposition}
    \begin{proof}
        \Yang{To be added.}
    \end{proof}


\subsection{Hensel's Lemma}

    \begin{theorem}[Hensel's lemma]\label{prop:Hensel_lemma}
        Let \(\kk\) be a complete non-archimedean field and \(F(T) \in \kk^\circ[T]\) a monic polynomial.
        Suppose that the reduction \(f(T) \in \calk_\kk[T]\) of \(F(T)\) factors as
        \[ f(T) = g(T) h(T), \]
        where \(g(T), h(T) \in \calk_\kk[T]\) are monic polynomials that are coprime in \(\calk_\kk[T]\).
        Then there exist monic polynomials \(G(T), H(T) \in \kk^\circ[T]\) such that
        \[ F(T) = G(T) H(T), \]
        and the reductions of \(G(T), H(T)\) in \(\calk_\kk[T]\) are \(g(T), h(T)\) respectively.
        \Yang{To be checked.}
    \end{theorem}
    \begin{proof}
        \Yang{To be added.}
    \end{proof}

    \begin{corollary}\label{prop:Hensel_lemma_to_solve_polynomial}
        Let \(\kk\) be a complete non-archimedean field and \(F(T) \in \kk^\circ[T]\) a monic polynomial.
        Suppose that the reduction \(f(T) \in \calk_\kk[T]\) of \(F(T)\) has a simple root \(\alpha \in \calk_\kk\).
        Then there exists a root \(a \in \kk^\circ\) of \(F(T)\) whose reduction is \(\alpha\).
        \Yang{To be revised.}
    \end{corollary}
    \begin{proof}
        \Yang{To be added.}
    \end{proof}


\subsection{Newton polygons}

    \Yang{To be filled.}
      \section{Finite field extensions}

\subsection{Finite-dimensional vector space}

    \begin{definition}\label{def:norm_on_vector_space_over_valuation_field}
        Let \(\kk\) be a valuation field and \(V\) a vector space over \(\kk\).
        A \emph{norm} on \(V\) is a function \(\|\cdot\|: V \to \bbR_{\geq 0}\) satisfying the following properties for all \(x, y \in V\) and \(a \in \kk\):
        \begin{enumerate}
            \item \(\|x\| = 0\) if and only if \(x = 0\);
            \item \(\|a x\| = |a| \cdot \|x\|\);
            \item \(\|x + y\| \leq \|x\| + \|y\|\).
        \end{enumerate}
    \end{definition}

    \begin{example}\label{def:maximal_norm_on_finite-dimensional_vector_space}
        Let \(\kk\) be a valuation field and \(V\) a finite-dimensional vector space over \(\kk\) with basis \(\{e_1, e_2, \ldots, e_n\}\).
        For any \(x = a_1 e_1 + a_2 e_2 + \cdots + a_n e_n \in V\), define
        \[
            \|x\|_{\max} := \max_{1 \leq i \leq n} |a_i|.
        \]
        Then \(\|\cdot\|_{\max}\) is a norm on \(V\), called the \emph{maximal norm} with respect to the basis \(\{e_1, e_2, \ldots, e_n\}\).
    \end{example}

    \begin{example}\label{def:1-norm_no_finite-dimensional_vector_space}
        Setting as in \cref{def:maximal_norm_on_finite-dimensional_vector_space}, for any \(x = a_1 e_1 + a_2 e_2 + \cdots + a_n e_n \in V\), define
        \[
            \|x\|_1 := |a_1| + |a_2| + \cdots + |a_n|.
        \]
        Then \(\|\cdot\|_1\) is also a norm on \(V\).
    \end{example}

    \begin{definition}\label{def:equivalent_norm_on_vector_space}
        Let \(\kk\) be a valuation field and \(V\) a vector space over \(\kk\).
        Two norms \(\|\cdot\|_1\) and \(\|\cdot\|_2\) on \(V\) are said to be \emph{equivalent} if there exist positive constants \(C_1, C_2 > 0\) such that for all \(x \in V\),
        \[
            C_1 \|x\|_1 \leq \|x\|_2 \leq C_2 \|x\|_1.
        \]
    \end{definition}

    \begin{lemma}\label{prop:norms_are_equivalent_iff_induces_the_same_topology}
        Let \(\kk\) be a valuation field and \(V\) a vector space over \(\kk\).
        Two norms \(\|\cdot\|_1\) and \(\|\cdot\|_2\) on \(V\) are equivalent if and only if they induce the same topology on \(V\).
    \end{lemma}
    \begin{proof}
        The sufficiency is clear.
        Now suppose that \(\|\cdot\|_1\) and \(\|\cdot\|_2\) induce the same topology on \(V\).
        Hence the unit open ball with respect to \(\|\cdot\|_1\) contains a unit open ball with respect to \(\|\cdot\|_2\).
        That is, 
        \[ \{x \in V : \|x\|_1 < 1\} \supseteq \{x \in V : \|x\|_2 < C\}. \]
        Then for every \(x \in V\) with \(\|x\|_1 = 1\), we have \(\|x\|_2 \geq C = C \|x\|_1\).
        By scaling, we get that for every \(x \in V\),
        \[ \|x\|_2 \geq C \|x\|_1. \]
        Similar for the other direction, we conclude that \(\|\cdot\|_1\) and \(\|\cdot\|_2\) are equivalent.
    \end{proof}

    \begin{proposition}\label{prop:finite_dimensional_vector_space_over_complete_fields_is_complete}
        Let \(V\) be a normed finite-dimensional vector space over a complete valuation field \(\kk\).
        Then \(V\) is complete.
    \end{proposition}
    \begin{proof}
        \Yang{To be added.}
    \end{proof}

    \begin{theorem}\label{prop:norm_on_finite_dimensional_vector_space_are_equivalent}
        Let \(V\) be a finite-dimensional vector space over a complete field \(\kk\).
        Then all norms on \(V\) are equivalent.
    \end{theorem}
    \begin{proof}
        Fix a basis \(\{e_1, e_2, \ldots, e_n\}\) of \(V\) and let \(\|\cdot\|_{\max}\) be the maximal norm with respect to this basis as in \cref{def:maximal_norm_on_finite-dimensional_vector_space}.
        Let \(\|\cdot\|\) be any norm on \(V\).
        It suffices to show that \(\|\cdot\|\) and \(\|\cdot\|_{\max}\) are equivalent.
        First we have 
        \[ \|y\| \leq \sum_{i=1}^n |a_i| \|e_i\| \leq \left(\sum_{i=1}^n \|e_i\|\right) \|y\|_{\max} \]
        for any \(y = a_1 e_1 + a_2 e_2 + \cdots + a_n e_n \in V\).
        It remains to show that there exists a constant \(C > 0\) such that for any \(y \in V\),
        \[ \|y\|_{\max} \leq C \|y\|. \]
        \Yang{To be added.}
    \end{proof}

    \begin{remark}\label{rmk:finite-dimensional_vector_space_over_non_complete_fields}
        If the base field \(\kk\) is not complete, then \cref{prop:norm_on_finite_dimensional_vector_space_are_equivalent} may fail.
        For example, let \(\kk = \bbQ\) with the usual absolute value, and let \(V = \bbQ[\alpha]\) with \(\alpha^2-\alpha-1=0\).
        There are two embeddings of \(V\) into \(\bbR\):
        \[ \iota_1: a + b\alpha \mapsto a + b\frac{1+\sqrt{5}}{2}, \quad \iota_2: a + b\alpha \mapsto a + b\frac{1-\sqrt{5}}{2}. \]
        Define two norms on \(V\) by
        \[
            \|x\|_1 := |\iota_1(x)|, \quad \|x\|_2 := |\iota_2(x)|,
        \]
        where \(|\cdot|\) is the usual absolute value on \(\bbR\).
        Then \(\|\cdot\|_1\) and \(\|\cdot\|_2\) are not equivalent since \(\iota_2(\alpha^n) \to 0\) as \(n \to \infty\) while \(\iota_1(\alpha^n) \to \infty\).
    \end{remark}

    The following lemma is a classical result in functional analysis, which will be used in the next subsection.

    \begin{lemma}\label{prop:operator_norm_on_M_n_k_with_k_complete}
        Let \(\kk\) be a complete field and \(V\) a normed finite-dimensional vector space over \(\kk\).
        Then 
        \[ \|\cdot\| : \End_{\kk}(V) \to \bbR_{\geq 0}, \quad T \mapsto \sup_{x \in V \setminus \{0\}} \frac{\|T(x)\|}{\|x\|} \]
        defines a norm on the \(\kk\)-vector space \(\End_{\kk}(V)\) satisfying
        \[ \|AB\| \leq \|A\| \cdot \|B\|, \quad \forall A, B \in \End_{\kk}(V). \]
    \end{lemma}
    \begin{proof}
        First we show the existence of the supremum, i.e., there exists \(C > 0\) such that for all \(x \in V \setminus \{0\}\), \(\|T(x)\| \leq C \|x\|\).
        Fix a basis \(\{e_1, e_2, \ldots, e_n\}\) of \(V\) and let \(\|\cdot\|_{\max}\) be the maximal norm with respect to this basis.
        Since all norms on \(V\) are bounded by each other by \cref{prop:norm_on_finite_dimensional_vector_space_are_equivalent}, we only need to show that there exists \(C > 0\) such that for all \(x \in V \setminus \{0\}\), \(\|T(x)\|_{1} \leq C \|x\|_{\max}\).
        Write \(T(e_i) = \sum_{j=1}^n a_{ij} e_j\) for \(1 \leq i \leq n\).
        For any \(x = \sum_{i=1}^n x_i e_i \in V\), we have
        \[ \|T(x)\|_1 = \left\|\sum_{j=1}^n \left(\sum_{i=1}^n a_{ij} x_i\right) e_j \right\|_1 = \sum_{j=1}^n \left|\sum_{i=1}^n a_{ij} x_i\right| \leq \left(\sum_{1 \leq i, j \leq n} |a_{ij}|\right) \|x\|_{\max}. \]
        Thus the supremum is finite.

        The linearity and positive-definiteness of \(\|\cdot\|\) are clear.
        It remains to show the triangle inequality and sub-multiplicativity.
        For any \(A, B \in \End_{\kk}(V)\), we have
        \[ \frac{\|(A + B)(x)\|}{\|x\|} = \frac{\|A(x)\|}{\|x\|} + \frac{\|B(x)\|}{\|x\|} \leq \|A\| + \|B\|. \]
        Taking supremum over all \(x \in V \setminus \{0\}\) gives \(\|A + B\| \leq \|A\| + \|B\|\).
        We have 
        \[ \|AB(x)\| \leq \|A\| \cdot \|B(x)\| \leq \|A\| \cdot \|B\| \cdot \|x\| \]
        and hence \(\|AB(x)\|/\|x\| \leq \|A\| \cdot \|B\|\).
        Taking supremum we get \(\|AB\| \leq \|A\| \cdot \|B\|\). 
    \end{proof}

\subsection{Finite field extensions}

    \begin{lemma}\label{prop:existence_of_absolute_value_on_finite_extension_of_complete_fields}
        Let \(\kk\) be a complete field and \(\bfl\) a finite extension of \(\kk\).
        Then there exists an absolute value on \(\bfl\) extending the absolute value on \(\kk\).
    \end{lemma}
    \begin{proof}
        Fix a norm \(\|\cdot\|_V\) on the \(\kk\)-vector space \(V = \bfl\).
        The norm \(\|\cdot\|_V\) induces an operator norm \(\|\cdot\|_{\op}\) on the \(\kk\)-vector space \(\End_{\kk}(V)\) as in \cref{prop:operator_norm_on_M_n_k_with_k_complete}.
        For any \(a \in \bfl\), let \(\mu_a \in \End_{\kk}(V)\) be the \(\kk\)-linear map defined by multiplication by \(a\).
        Note that \(a \mapsto \mu_a\) gives a embedding of \(\kk\)-algebras and if \(a \in \kk\), \(\|\mu_a\|_{\op} = \|a\|_{\kk}\).
        Thus the restriction of \(\|\cdot\|_{\op}\) to \(\bfl\) gives an norm on \(\bfl\) extending that on \(\kk\).
        The normed ring \((\bfl, \|\cdot\|_{\op})\) is a Banach ring since it is a finite-dimensional vector space over the complete field \(\kk\).
        By \cref{thm:norm_spectrum_of_Banach_rings_is_nonempty}, there exists a multiplicative seminorm \(\|\cdot\|_{\bfl}\) on \(\bfl\) bounded by \(\|\cdot\|_{\op}\).
        In particular, \(\|\cdot\|_{\bfl}\) is bounded by \(\|\cdot\|_{\kk}\) on \(\kk\).
        On a field, if one norm is bounded by another norm, then they must be equal (consider the inverse elements).
        Thus \(\|\cdot\|_{\bfl}\) extends the absolute value on \(\kk\).
    \end{proof}

    \begin{theorem}\label{prop:absolute_value_on_finite_extension_of_complete_fields}
        Let \(\kk\) be a complete field and \(\bfl\) a finite extension of \(\kk\).
        Then the absolute value on \(\bfl\) which extends the absolute value on \(\kk\) is uniquely determined by the absolute value on \(\kk\).
        Furthermore, we have 
        \[ \|\cdot\|_{\bfl} = \|N_{\bfl/\kk}(\cdot)\|_{\kk}^{1/n}, \]
        where \(n = [\bfl : \kk]\) and \(N_{\bfl/\kk}\) is the norm map from \(\bfl\) to \(\kk\).
        % \Yang{To be checked.}
    \end{theorem}
    \begin{proof}
        Let \(\|\cdot\|_{\bfl}\) be arbitrary absolute value on \(\bfl\) extending that on \(\kk\).
        We will show that \(\|\cdot\|_{\bfl}\) must be equal to \(\|N_{\bfl/\kk}(\cdot)\|_{\kk}^{1/n}\).
        For any \(a \in \bfl\), set \(b = a^n/N_{\bfl/\kk}(a) \in \bfl\).
        Then \(N_{\bfl/\kk}(b) = 1\) and 
        \[ \|b\|_{\bfl} = \frac{\|a\|_{\bfl}^n}{\|N_{\bfl/\kk}(a)\|_{\kk}}. \]
        Thus it suffices to show that \(\|b\|_{\bfl} = 1\) whenever \(N_{\bfl/\kk}(b) = 1\).

        Note that the norm map \(N_{\bfl/\kk}: \bfl \to \kk\) is the determinant of the \(\kk\)-linear map \(\mu_b \in \End_{\kk}(V)\) defined by multiplication by \(b\).
        Hence it is continuous on \(\bfl\) (since it is a polynomial in the entries of the matrix representation).
        If \(\|b\|_{\bfl} < 1\), then \(\|b^m\|_{\bfl} \to 0\) as \(m \to \infty\).
        Thus \(N_{\bfl/\kk}(b^m) = \det(\mu_{b^m}) \to 0\) as \(m \to \infty\), contradicting the fact that \(N_{\bfl/\kk}(b^m) = 1\) for all \(m\).
        Similarly, if \(\|b\|_{\bfl} > 1\), then just consider \(b^{-1}\).

        % Fix a norm \(\|\cdot\|_V\) on the \(\kk\)-vector space \(V = \bfl\).
        % The norm \(\|\cdot\|_V\) induces an operator norm \(\|\cdot\|_{\op}\) on the \(\kk\)-vector space \(\End_{\kk}(V)\) as in \cref{prop:operator_norm_on_M_n_k_with_k_complete}.
        % For any \(a \in \bfl\), let \(\mu_a \in \End_{\kk}(V)\) be the \(\kk\)-linear map defined by multiplication by \(a\).
        % Note that \(a \mapsto \mu_a\) is a \(\kk\)-algebra homomorphism and if \(a \in \kk\), \(\|\mu_a\|_{\op} = \|a\|_{\kk}\).

        % \begin{step}\label{step_in_thm_absolute_value_on_finite_extension_of_complete_fields:the_field_norm_is_the_spectral_radius}
        %     Show that 
        %     \[ \|N_{\bfl/\kk}(a)\|_{\kk} = \lim_{m \to \infty} \sqrt[m]{\|\mu_{a^{mn}}\|_{\op}} \eqqcolon \rho(a^n). \]
        % \end{step}

        % The existence of the limit \(\rho(a) = \lim_{m \to \infty} \sqrt[m]{\|\mu_{a^m}\|_{\op}}\) follows from the sub-multiplicativity of the operator norm.
        % And note that \(\rho(a^n) = \rho(a)^n\) for all \(a \in \bfl\) and \(n \in \bbN\).
        % We can assume that \(a \neq 0\).
        % Let \(b = a^n/N_{\bfl/\kk}(a) \in \bfl\).
        % Then \(N_{\bfl/\kk}(b) = 1\) and 
        % \[ \rho(b) = \lim_{m \to \infty} \sqrt[m]{\|\mu_{b^m}\|_{\op}} = \lim_{m \to \infty} \sqrt[m]{\frac{\|\mu_{a^{mn}}\|_{\op}}{\|N_{\bfl/\kk}(a)^m\|_{\kk}}} = \frac{\rho(a^n)}{\|N_{\bfl/\kk}(a)\|_{\kk}} \]
        % since \(N_{\bfl/\kk}\) is multiplicative and \(N_{\bfl/\kk}(a) \in \kk\).
        % Thus it suffices to show that \(\rho(b) = 1\). 

        % Suppose that \(\rho(b) < 1\).
        % Then there exists \(\delta \in (0,1)\) and \(M > 0\) such that for all \(m \geq M\), \(\|\mu_{b^m}\|_{\op} < \delta^m\).
        % In particular, we have \(\mu_{b^m} \to 0\) as \(m \to \infty\) in the normed vector space \(\End_{\kk}(V)\).
        % However, note that \(\det\) is continuous on \(\End_{\kk}(V)\) (since it is a polynomial in the entries of the matrix representation).
        % Thus we have \(N_{\bfl/\kk}(b^m) = \det(\mu_{b^m}) \to 0\) as \(m \to \infty\), contradicting the fact that \(N_{\bfl/\kk}(b^m) = 1\) for all \(m\).
        
        % if \(\rho(b) > 1\), 
        % \Yang{We can not get \(\rho(b^{-1}) < 1\) directly.}

        % \begin{step}\label{step_in_thm_absolute_value_on_finite_extension_of_complete_fields:triangle_inequality}
        %     Show that 
        %     \[ \rho(a + b) \leq \rho(a) + \rho(b) \]
        %     and hence \(\|\cdot\|_{\bfl} = \|N_{\bfl/\kk}(\cdot)\|_{\kk}^{1/n}\) defines an absolute value on \(\bfl\) extending that on \(\kk\).
        % \end{step}


        % \begin{step}\label{step_in_thm_absolute_value_on_finite_extension_of_complete_fields:uniqueness_of_absolute_value}
        %     Show the uniqueness of the absolute value on \(\bfl\) extending that on \(\kk\).
        % \end{step}




        % \Yang{To be added.}
    \end{proof}

    % \begin{remark}\label{rmk:compatiblity_of_extension_and_completion}
    %     \Yang{I want to discuss some compatiblity of extension and completion.}
    % \end{remark}

    \begin{proposition}\label{prop:completion_of_algebraically_closed_valuation_fields_is_algebraically_closed}
        Let \(\kk\) be an algebraically closed non-archimedean field.
        Then its completion \(\widehat{\kk}\) is also algebraically closed.
        % \Yang{To be checked.}
    \end{proposition}
    \begin{proof}
        Let \(f \in \widehat{\kk}[X]\) be a non-constant polynomial.
        We will show that \(f\) has a root in \(\widehat{\kk}\).
        Take a sequence of polynomials \(\{f_n\}_{n \in \bbN}\) in \(\kk[X]\) converging to \(f\) coefficient-wisely and of the same degree \(d\).
        Since \(\kk\) is algebraically closed, each \(f_n\) splits completely in \(\kk\) and hence in \(\widehat{\kk}\).
        Write \(f_n(X) = \prod_{i=1}^d (X - \alpha_{n,i})\) with \(\alpha_{n,i} \in \widehat{\kk}\).

        Let \(\bfl\) be a finite extension of \(\widehat{\kk}\) such that \(f\) has a root \(\alpha\) in \(\bfl\).
        For every \(\varepsilon > 0\), if there are infinitely many \(n\) such that \(\alpha_{n,i} \notin B(\alpha, \varepsilon)\) for all \(1 \leq i \leq d\), 
        then we have \(|f_n(\alpha)| \geq \varepsilon^d\) for infinitely many \(n\), contradicting the fact that \(f_n(\alpha) \to f(\alpha) = 0\).
        Thus for every \(\varepsilon > 0\), there exists \(N > 0\) such that for all \(n \geq N\), there exists \(1 \leq i \leq d\) with \(\alpha_{n,i} \in B(\alpha, \varepsilon)\).
        That is, we can find a sequence \(\alpha_{n,i_n} \in \kk\) converging to \(\alpha\).
        Since \(\widehat{\kk}\) is complete, we have \(\alpha \in \widehat{\kk}\).
        % \Yang{To be added.}
    \end{proof}

      \section{Analytic functions}

\subsection{Failure of continuous and differentiable functions}

    \begin{definition}\label{def:differentiable_functions}
        Let \((\kk,\|\cdot\|)\) be a non-archimedean field and \(U \subset \kk\) be an open subset.
        A function \(f:U \to \kk\) is said to be \emph{differentiable} at a point \(a \in U\) if the limit
        \[
            f'(a) := \lim_{x \to a} \frac{f(x)-f(a)}{x-a}
        \]
        exists in \(\kk\).
        If \(f\) is differentiable at every point in \(U\), we say that \(f\) is differentiable on \(U\).
        \Yang{to be revised.}
    \end{definition}

    \begin{proposition}\label{prop:failure_of_mean_valuable_theorem}
        Let \((\kk,\|\cdot\|)\) be a non-archimedean field.
        Then there exists a continuous function \(f:\kk \to \kk\) such that for any \(x,y \in \kk\) with \(x \neq y\), we have
        \[
            \frac{f(x)-f(y)}{x-y} = 0.
        \]
    \end{proposition}

\subsection{Power series}

    \begin{proposition}\label{prop:convergent_series_over_non_archimedean_fields}
        Let \((\kk,\|\cdot\|)\) be a complete non-archimedean field and \(\sum_{n=0}^{+\infty} a_n\) be a series in \(\kk\).
        Then the series \(\sum_{n=0}^{+\infty} a_n\) converges if and only if \(\lim_{n \to +\infty} a_n = 0\).
        \Yang{To be checked.}
    \end{proposition}

    \begin{definition}\label{def:Tate_algebra}
        Let \((\kk,\|\cdot\|)\) be a complete non-archimedean field.
    \end{definition}

\subsection{Analytic functions}

    As in the case of real analysis, we can define analytic functions over non-archimedean fields using power series.
      \section[Example: p-adic fields]{Example: \(p\)-adic fields}

\subsection[p-adic fields]{\(p\)-adic fields}

    \begin{construction}\label{constr:p-adic_absolute_value_on_number_field}
        Let \(K\) be a number field and \(\frakp\) be a prime ideal of the ring of integers \(\calO_K\) of \(K\).
        Considering the localization \((\calO_K)_\frakp\) of \(\calO_K\) at \(\frakp\), which is a discrete valuation ring, denote by \(v_\frakp: K^\times \to \bbZ\) the corresponding discrete valuation.
        The \emph{\(p\)-adic absolute value} on \(K\) associated to \(\frakp\) is defined as
        \[ |x|_\frakp := N(\frakp)^{-v_\frakp(x)},\quad\forall x \in K, \]
        where \(N(\frakp) := \#(\calO_K / \frakp)\) is the norm of \(\frakp\).

        The completion of \(K\) with respect to the \(p\)-adic absolute value \(|\cdot|_\frakp\) is denoted by \(K_\frakp\), called the \emph{\(\frakp\)-adic field}.
    \end{construction}

    One can just focus on the case \(K = \bbQ\) and \(\frakp = (p)\) for a prime number \(p\).

    \begin{example}\label{eg:p-adic_field}
        Let \(p\) be a prime number. 
        For every \(r \in \bbQ\), we can write \(r\) as \(r = p^n \frac{a}{b}\), where \(n \in \bbZ\) and \(a,b \in \bbZ\) are integers not divisible by \(p\).
        The \emph{\(p\)-adic absolute value} on \(\bbQ\) is defined as
        \[ |r|_p := p^{-n}. \]
      
        The \(p\)-adic field \(\bbQ_p\) can be described concretely as follows:
        \[ \bbQ_p = \left\{ \sum_{i = n}^{+\infty} a_i p^i \middle| n \in \bbZ, a_i \in \{0, 1, \ldots, p-1\} \right\}. \]
        For \(x = \sum_{i = n}^{+\infty} a_i p^i \in \bbQ_p\) with \(a_n \neq 0\), its \(p\)-adic absolute value is given by \(|x|_p = p^{-n}\).
        The operations of addition and multiplication on \(\bbQ_p\) are defined similarly as those on decimal expansions.
    \end{example}

    \begin{proposition}\label{prop:group_structure_of_Q_p_times}
        The multiplicative group \(\bbQ_p^\times\) of the \(p\)-adic field \(\bbQ_p\) admits the following decomposition:
        \[ \bbQ_p^\times \cong p^\bbZ \times \bbZ_p^\times, \]
        where \(p^\bbZ := \{p^n \mid n \in \bbZ\}\) and \(\bbZ_p^\times := \{x \in \bbQ_p \mid |x|_p = 1\}\) is the group of units of the ring of \(p\)-adic integers \(\bbZ_p\).
        \Yang{To be checked.}
    \end{proposition}

    \Yang{What is the relation between the finite extension of \(\bbQ_p\) and \(K_\frakp\)?}

\subsection{Completion}

    \begin{proposition}\label{prop:algebraically_closure_of_Q_p_is_not_complete}
        The algebraic closure \(\overline{\bbQ}_p\) of \(\bbQ_p\) is not complete with respect to the extension of the \(p\)-adic absolute value \(|\cdot|_p\).
    \end{proposition}
    
    \begin{construction}\label{constr:p-adic_complex_number}
        Let \(p\) be a prime number.
        The field \(\bbC_p\) of \emph{\(p\)-adic complex numbers} is defined as the completion of the algebraic closure of \(\bbQ_p\) with respect to the unique extension of the \(p\)-adic absolute value \(|\cdot|_p\) on \(\bbQ_p\).
        The field \(\bbC_p\) is algebraically closed and complete with respect to \(|\cdot|_p\).
        \Yang{To be completed.}
    \end{construction}

    \begin{proposition}\label{prop:C_p_is_not_spherically_complete}
        The field \(\bbC_p\) of \(p\)-adic complex numbers is not spherically complete.
    \end{proposition}

    \begin{construction}\label{constr:spherically_complete_p-adic_fields}
        Let \(p\) be a prime number.
        \Yang{We construct the \emph{spherically complete \(p\)-adic field} \(\Omega_p\).}
        \Yang{To be completed.}
    \end{construction}

\subsection{Elementary functions}

    Exponential, logarithmic, and the interpolation functions.

   \chapter{Affinoid algebras}
      % \section{Valuation fields}


    \begin{definition}\label{def:valuation_field}
        A \emph{valuation field} is a field \(\kk\) equipped with an absolute value \(\|\cdot\|: \kk \to \bbR_{\ge 0}\) that satisfies the following properties for all \(x,y \in \kk\):
        \begin{enumerate}
            \item Non-negativity: \(\|x\| = 0\) if and only if \(x = 0\).
            \item Multiplicativity: \(\|xy\| = \|x\| \|y\|\).
            \item Triangle inequality: \(\|x + y\| \le \|x\| + \|y\|\).
        \end{enumerate}
    \end{definition}

    \begin{theorem}[Classification of absolute values on \(\bbQ\)]\label{thm:classification_absolute_values_Q}
        Let \(\|\cdot\|\) be an absolute value on \(\bbQ\).
        Then \(\|\cdot\|\) is either equivalent to the usual absolute value \(|\cdot|_\infty\) or to a \(p\)-adic absolute value \(|\cdot|_p\) for some prime \(p\).
        \Yang{The name is to be added.}
    \end{theorem}
      \section{Semi-normed Rings and Modules}


\subsection{Semi-normed algebraic structures}

    \begin{definition}\label{def:semi-normed_abelian_group_rings_modules_and_algebras}
        Let \(G\) be an abelian group.
        A \emph{semi-norm} on \(G\) is a function \(\|\cdot\|: G \to \bbR_{\geq 0}\) such that
        \begin{itemize}
            \item \(\|0\| = 0\);
            \item \(\forall x,y \in G, \|x + y\| \leq \|x\| + \|y\|\).
        \end{itemize}
        Suppose that \(R\) is a ring (commutative with unity) and \(\|\cdot\|\) is a semi-norm on the underlying abelian group of \(R\).
        We further require that
        \begin{itemize}
            \item \(\|1\| = 1\);
            \item \(\forall x,y \in R, \|xy\| \leq \|x\|\|y\|\).
        \end{itemize}
        Suppose that \((M,\|\cdot\|_M)\) is an \(R\)-module and \(\|\cdot\|_M\) is a semi-norm on the underlying abelian group of \(M\).
        We further require that 
        \begin{itemize}
            \item \(\forall a \in R, x \in M, \|ax\|_M \leq \|a\| \|x\|_M\).
        \end{itemize}
        Suppose that \((A, \|\cdot\|_A)\) is an \(R\)-algebra and \(\|\cdot\|_A\) is a semi-norm on the underlying \(R\)-module of \(A\).
        We further require that this semi-norm is a semi-norm on the underlying ring of \(A\).
    \end{definition}

    \begin{definition}\label{def:norm_on_algebraic_structure}
        Let \(A\) be an abelian group (or ring, \(R\)-module, \(R\)-algebra) equipped with a semi-norm \(\|\cdot\|\).
        If \(\forall x \in A, \|x\| = 0 \iff x = 0\), then we say \(\|\cdot\|\) is a \emph{norm}.
    \end{definition}

    \Yang{Note that this definition of semi-normed module is a little different of \cite{Ber90}}

    \begin{definition}\label{def:bounded_semi-norm}
        Let \(\|\cdot\|_1\) and \(\|\cdot\|_2\) be two semi-norms on an abelian group (or ring, \(R\)-module, \(R\)-algebra) \(A\).
        We say \(\|\cdot\|_1\) is \emph{bounded} by \(\|\cdot\|_2\) if there exists a constant \(C > 0\) such that \(\forall x \in A, \|x\|_1 \leq C\|x\|_2\).
        If \(\|\cdot\|_1\) and \(\|\cdot\|_2\) are bounded by each other, we say they are \emph{equivalent}.
    \end{definition}

    \begin{remark}\label{rmk:semi-norms_bounded_by_each_other_induces_the_same_topologies}
        Equivalent semi-norms induce the same topology on \(A\).
    \end{remark}

    \begin{definition}\label{def:residue_semi-norm}
        Let \(M\) be a semi-normed abelian group (or ring, \(R\)-module, \(R\)-algebra) and \(N \subseteq M\) be a subgroup (or ideal, \(R\)-submodule, ideal).
        The \emph{residue semi-norm} on the quotient group \(M/N\) is defined as
        \[
            \|x + N\|_{M/N} = \inf_{y \in N} \|x + y\|_M.
        \]
    \end{definition}

    \Yang{Is this always a semi-norm? In particular, \(\|1\| = 1\)?}

    Unless otherwise specified, we always equip the quotient \(M/N\) with the residue semi-norm.

    \begin{remark}\label{rmk:residue_semi-norm}
        The residue semi-norm is a norm if and only if \(N\) is closed in \(M\).
    \end{remark}

    \begin{definition}\label{def:bounded_and_admissible_homomorphism}
        Let \(M\) and \(N\) be two semi-normed abelian groups (or rings, \(R\)-modules, \(R\)-algebras).
        A homomorphism \(f: M \to N\) is called \emph{bounded} if there exists a constant \(C > 0\) such that \(\forall x \in M, \|f(x)\|_N \leq C\|x\|_M\).
        
        A bounded homomorphism \(f: M \to N\) is called \emph{admissible} if the induced isomorphism \(M/\ker f \to \Image f\) is an isometry, i.e., \(\forall x \in M, \|f(x)\|_N = \|x\|_{M/\ker f}\).
    \end{definition}

    % \begin{definition}\label{def:semi-normed_rings}
    %     Let \(R\) be a ring (commutative with unity).
    %     A \emph{semi-norm} on \(R\) is a semi-norm \(\|\cdot\|\) on the underlying abelian group of \(R\) such that \(\forall x,y \in R, \|xy\| \leq \|x\|\|y\|\) and \(\|1\| = 1\).
    %     A \emph{semi-normed ring} is a ring equipped with a semi-norm.
    % \end{definition}

    \begin{definition}\label{def:multiplicative_and_power_multiplicative_semi-norm}
        A semi-norm \(\|\cdot\|\) on a ring \(R\) is called \emph{multiplicative} if \(\forall x,y \in R, \|xy\| = \|x\|\|y\|\).
        It is called \emph{power-multiplicative} if \(\forall x \in R, \|x^n\| = \|x\|^n\) for all integers \(n \geq 1\).
        % A power-multiplicative semi-norm is also called \emph{uniform}.
    \end{definition}

    % \begin{definition}\label{def:semi-normed_modules}
    %     Let \((R, \|\cdot\|_R)\) be a normed ring.
    %     A \emph{semi-normed \(R\)-module} is a pair \((M, \|\cdot\|_M)\) where \(M\) is an \(R\)-module and \(\|\cdot\|_M\) is a semi-norm on the underlying abelian group of \(M\) such that there exists \(C > 0\) with \(\forall a \in R, x \in M, \|ax\|_M \leq C \|a\|_R \|x\|_M\).
    % \end{definition}

    % One can talk about boundedness, admissibility and residue semi-norms in the contexts of semi-normed rings and semi-normed modules similar to those in semi-normed abelian groups.


    % \Yang{To be continued...}

\subsection{Banach rings}

    \begin{definition}\label{def:complete_semi-normed_abelian_group}
        A (semi-)norm on an abelian group \(M\) induces a (pseudo-)metric \(d(x,y) = \|x - y\|\) on \(M\).
        A (semi-)normed abelian group \(M\) is called \emph{complete} if it is complete as a (pseudo-)metric space.
    \end{definition}

    \begin{definition}\label{def:banach_ring}
        A \emph{banach ring} is a complete normed ring.
    \end{definition}

    \Yang{The counterpart of prime ideal is multiplicative seminorm}.

    \begin{definition}\label{def:completion_of_normed_algebraic_structures}
        Let \((A, \|\cdot\|_A)\) be a (semi-)normed algebraic structure, e.g., a (semi-)normed abelian group, a (semi-)normed ring, or a (semi-)normed module.
        The \emph{completion} of \(A\), denoted by \(\widehat{A}\), is the completion of \(A\) as a (pseudo-)metric space.
        Since \(A\) is dense in its completion, the algebraic operations and (semi-)norms on \(A\) can be uniquely extended to the completion.
        % \Yang{To be continued.}
    \end{definition}
    
    Let \(R\) be a normed ring and \(M,N\) be semi-normed \(R\)-modules.
    There is a natural semi-norm on the tensor product \(M \otimes_R N\) defined as
    \[
        \|z\|_{M \otimes_R N} = \inf \left\{ \sum_{i} \|x_i\|_M \|y_i\|_N : z = \sum_i x_i \otimes y_i, x_i \in M, y_i \in N \right\}.
    \]

    \begin{definition}\label{def:complete_tensor_product}
        Let \(R\) be a complete normed ring and \(M,N\) complete semi-normed \(R\)-modules.
        The \emph{complete tensor product} \(M \widehat{\otimes}_R N\) is defined as the completion of the semi-normed \(R\)-module \(M \otimes_R N\).
    \end{definition}

    \begin{definition}\label{def:spectral_radius_on_banach_rings}
        Let \(R\) be a banach ring.
        For each \(f \in R\), the \emph{spectral radius} of \(f\) is defined as
        \[
            \rho(f) = \lim_{n \to \infty} \|f^n\|^{1/n}.
        \]
    \end{definition}

    \Yang{Since , \(\rho(f)\) exists.}

    \begin{definition}\label{def:uniform_banach_ring}
        A banach ring \(R\) is called \emph{uniform} if its norm is power-multiplicative.
    \end{definition}

    \begin{proposition}\label{prop:spectral_radius_defines_a_power-multiplicative_semi-norm}
        Let \((R,\|\cdot\|)\) be a banach ring.
        The spectral radius \(\rho(\cdot)\) defines a power-multiplicative semi-norm on \(R\) that is bounded by \(\|\cdot\|\).
    \end{proposition}
    \begin{proof}
        \Yang{To be continued.}
    \end{proof}

    \begin{definition}\label{def:quasi_nilpotent_element}
        Let \(R\) be a banach ring.
        An element \(f \in R\) is called \emph{quasi-nilpotent} if \(\rho(f) = 0\).
        All quasi-nilpotent elements of \(R\) form an ideal, denoted by \(\Qnil(R)\).
    \end{definition}

    \begin{definition}\label{def:uniformization_of_banach_rings}
        Let \(R\) be a banach ring.
        The \emph{uniformization} of \(R\), denoted by \(R \to R^u\), is the banach ring with the universal property among all bounded homomorphisms from \(R\) to uniform banach rings.
        \Yang{To be continued.}
    \end{definition}

    \begin{proposition}\label{prop:the_uniformization_of_banach_rings_given_by_spectral_radius}
        Let \(R\) be a banach ring.
        The completion of \(R/\Qnil(R)\) with respect to the spectral radius \(\rho(\cdot)\) is the uniformization of \(R\).
    \end{proposition}
    \begin{proof}
        \Yang{To be continued.}
    \end{proof}

    % \Yang{To be continued...}


\subsection{Complete tensor product}


\subsection{Examples}

    \begin{example}\label{eg:trivial_normed_rings}
        Let \(R\) be arbitrary ring.
        The \emph{trivial norm} on \(R\) is defined as \(\|x\| = 0\) if \(x = 0\) and \(\|x\| = 1\) if \(x \neq 0\).
        The ring \(R\) equipped with the trivial norm is a normed ring.
    \end{example}

    \begin{example}\label{eg:C_and_R_as_complete_fields}
        The fields \(\bbC\) and \(\bbR\) equipped with the usual absolute value are complete fields.
    \end{example}

    \begin{example}\label{eg:p-adic_fields_as_complete_fields}
        The field \(\bbQ_p\) of \(p\)-adic numbers equipped with the \(p\)-adic norm is a complete non-Archimedean field.
    \end{example}

    \begin{example}\label{eg:ring_of_absolutely_convergent_power_series_as_banach_rings}
        Let \(R\) be a banach ring and \(r > 0\) be a real number.
        We define the ring of absolutely convergent power series over \(\kk\) with radius \(r\) as
        \[ R\left<T/r\right> \coloneqq \left\{\sum_{n=0}^{\infty} a_n T^n \in R[[T]] : \sum_{n=0}^{\infty} \|a_n\| r^n < \infty \right\}. \]
        Equipped with the norm \(\|\sum_{n=0}^{\infty} a_n T^n\| = \sum_{n=0}^{\infty} \|a_n\| r^n\), the ring \(R\left<T/r\right>\) is a banach ring.

        When \(R = \kk\) is a 
        \Yang{To be checked.}
    \end{example}

    \begin{example}\label{eg:Tate_algebras_with_Gauss_norm_as_banach_algebras}
        Let \(\kk\) be a non-Archimedean complete field.
        We define
        \[ \kk\{T_1/r_1, \ldots, T_n/r_n\} \coloneqq \left\{\sum_{I \in \bbN^n} a_I T^I \in \kk[[T_1, \ldots, T_n]] : \lim_{|I| \to \infty} |a_I| r^I = 0 \right\}, \]
        where \(r = (r_1, \ldots, r_n)\) is an n-tuple of positive real numbers, \(T^I = T_1^{i_1} \cdots T_n^{i_n}\) for \(I = (i_1, \ldots, i_n)\), and \(|I| = i_1 + \cdots + i_n\).
        Equipped with the norm \(\|\sum_{I \in \bbN^n} a_I T^I\| = \sup_{I \in \bbN^n} |a_I| r^I\), the affinoid \(\kk\)-algebra \(\kk\{T_1/r_1, \ldots, T_n/r_n\}\) is a banach \(\kk\)-algebra.
        This is called the \emph{Tate algebra} over \(\kk\) with polyradius \(r\) equipped with the \emph{Gauss norm}.
        We will denote \(\kk\{\underline{T/r}\} = \kk\{T_1/r_1, \ldots, T_n/r_n\}\) for simplicity.
    \end{example}
    \Yang{To be continued...}
      \section{Affinoid algebras}

\subsection{The first properties}

    \begin{definition}\label{def:affinoid_algebras}
        Let \(\kk\) be a non-archimedean field. 
        A banach \(\kk\)-algebra \(A\) is called a \emph{ affinoid \(\kk\)-algebra} if there exists an admissible surjective homomorphism
        \[
            \varphi: \kk\{ r_1^{-1} T_1, \ldots, r_n^{-1} T_n \} \twoheadrightarrow A
        \]
        for some \(n \in \bbN\) and \(r_1, \ldots, r_n \in \bbR_{>0}\).

        If one can choose \(r_1 = \cdots = r_n = 1\), then we say that \(A\) is a \emph{strict affinoid \(\kk\)-algebra}.
    \end{definition}

    \begin{definition}\label{def:restricted_Laurent_series}
        Let \(\kk\) be a non-archimedean field.
        We define the \emph{ring of restricted Laurent series} over \(\kk\) as 
        \[ \KK_r = \bfL_{\kk,r} = \left\{\sum_{n \in \bbZ} a_n T^n : a_n \in \kk, \lim_{|n| \to \infty} |a_n| r^n = 0 \right\} \]
        equipped with the norm
        \[ \|f\| = \sup_{n \in \bbZ} |a_n| r^n. \]
    \end{definition}

    \Yang{Is \(\KK_r\) always a field?}
    \Yang{Do we have \(\bfL_{\kk,r} = \Frac (\kk\{T/r\})\)?}

    \begin{proposition}\label{prop:restricted_Laurent_series_is_a_field_when_r_is_not_root_of_absolute_value}
        Let \(\kk\) be a non-archimedean field.
        If \(r \notin \sqrt{|\kk^\times|}\), then \(\KK_r\) is a complete non-archimedean field with non-trivial absolute value extending that of \(\kk\).
    \end{proposition}

    \begin{proposition}\label{prop:affinoid_algebra_is_noetherian_and_ideal_is_clsoed}
        Let \(A\) be an affinoid \(\kk\)-algebra.
        Then \(A\) is noetherian, and every ideal of \(A\) is closed.
    \end{proposition}

    \begin{proposition}\label{prop:the_norm_on_affinoid_is_bounded_by_spectral_radius}
        Let \(A\) be an affinoid \(\kk\)-algebra.
        Then there exists a constant \(C > 0\) and \(N > 0\) such that for all \(f \in A\) and \(n \geq N\), we have
        \[
            \|f^n\| \leq C \rho(f)^n.   
        \]
    \end{proposition}

    \begin{proposition}\label{prop:affinoid_algebra_is_strict_when_radius_in_the_radical_of_valuation_set}
        Let \(A\) be an affinoid \(\kk\)-algebra.
        If and only if \(\rho(f) \in \sqrt{|\kk|}\) for all \(f \in A\), then \(A\) is strict.
        \Yang{To be complete.}
    \end{proposition}

\subsection{Noetherian normalization theorem}
      \section{Finite field extensions}

\subsection{Finite-dimensional vector space}

    \begin{definition}\label{def:norm_on_vector_space_over_valuation_field}
        Let \(\kk\) be a valuation field and \(V\) a vector space over \(\kk\).
        A \emph{norm} on \(V\) is a function \(\|\cdot\|: V \to \bbR_{\geq 0}\) satisfying the following properties for all \(x, y \in V\) and \(a \in \kk\):
        \begin{enumerate}
            \item \(\|x\| = 0\) if and only if \(x = 0\);
            \item \(\|a x\| = |a| \cdot \|x\|\);
            \item \(\|x + y\| \leq \|x\| + \|y\|\).
        \end{enumerate}
    \end{definition}

    \begin{example}\label{def:maximal_norm_on_finite-dimensional_vector_space}
        Let \(\kk\) be a valuation field and \(V\) a finite-dimensional vector space over \(\kk\) with basis \(\{e_1, e_2, \ldots, e_n\}\).
        For any \(x = a_1 e_1 + a_2 e_2 + \cdots + a_n e_n \in V\), define
        \[
            \|x\|_{\max} := \max_{1 \leq i \leq n} |a_i|.
        \]
        Then \(\|\cdot\|_{\max}\) is a norm on \(V\), called the \emph{maximal norm} with respect to the basis \(\{e_1, e_2, \ldots, e_n\}\).
    \end{example}

    \begin{example}\label{def:1-norm_no_finite-dimensional_vector_space}
        Setting as in \cref{def:maximal_norm_on_finite-dimensional_vector_space}, for any \(x = a_1 e_1 + a_2 e_2 + \cdots + a_n e_n \in V\), define
        \[
            \|x\|_1 := |a_1| + |a_2| + \cdots + |a_n|.
        \]
        Then \(\|\cdot\|_1\) is also a norm on \(V\).
    \end{example}

    \begin{definition}\label{def:equivalent_norm_on_vector_space}
        Let \(\kk\) be a valuation field and \(V\) a vector space over \(\kk\).
        Two norms \(\|\cdot\|_1\) and \(\|\cdot\|_2\) on \(V\) are said to be \emph{equivalent} if there exist positive constants \(C_1, C_2 > 0\) such that for all \(x \in V\),
        \[
            C_1 \|x\|_1 \leq \|x\|_2 \leq C_2 \|x\|_1.
        \]
    \end{definition}

    \begin{lemma}\label{prop:norms_are_equivalent_iff_induces_the_same_topology}
        Let \(\kk\) be a valuation field and \(V\) a vector space over \(\kk\).
        Two norms \(\|\cdot\|_1\) and \(\|\cdot\|_2\) on \(V\) are equivalent if and only if they induce the same topology on \(V\).
    \end{lemma}
    \begin{proof}
        The sufficiency is clear.
        Now suppose that \(\|\cdot\|_1\) and \(\|\cdot\|_2\) induce the same topology on \(V\).
        Hence the unit open ball with respect to \(\|\cdot\|_1\) contains a unit open ball with respect to \(\|\cdot\|_2\).
        That is, 
        \[ \{x \in V : \|x\|_1 < 1\} \supseteq \{x \in V : \|x\|_2 < C\}. \]
        Then for every \(x \in V\) with \(\|x\|_1 = 1\), we have \(\|x\|_2 \geq C = C \|x\|_1\).
        By scaling, we get that for every \(x \in V\),
        \[ \|x\|_2 \geq C \|x\|_1. \]
        Similar for the other direction, we conclude that \(\|\cdot\|_1\) and \(\|\cdot\|_2\) are equivalent.
    \end{proof}

    \begin{proposition}\label{prop:finite_dimensional_vector_space_over_complete_fields_is_complete}
        Let \(V\) be a normed finite-dimensional vector space over a complete valuation field \(\kk\).
        Then \(V\) is complete.
    \end{proposition}
    \begin{proof}
        \Yang{To be added.}
    \end{proof}

    \begin{theorem}\label{prop:norm_on_finite_dimensional_vector_space_are_equivalent}
        Let \(V\) be a finite-dimensional vector space over a complete field \(\kk\).
        Then all norms on \(V\) are equivalent.
    \end{theorem}
    \begin{proof}
        Fix a basis \(\{e_1, e_2, \ldots, e_n\}\) of \(V\) and let \(\|\cdot\|_{\max}\) be the maximal norm with respect to this basis as in \cref{def:maximal_norm_on_finite-dimensional_vector_space}.
        Let \(\|\cdot\|\) be any norm on \(V\).
        It suffices to show that \(\|\cdot\|\) and \(\|\cdot\|_{\max}\) are equivalent.
        First we have 
        \[ \|y\| \leq \sum_{i=1}^n |a_i| \|e_i\| \leq \left(\sum_{i=1}^n \|e_i\|\right) \|y\|_{\max} \]
        for any \(y = a_1 e_1 + a_2 e_2 + \cdots + a_n e_n \in V\).
        It remains to show that there exists a constant \(C > 0\) such that for any \(y \in V\),
        \[ \|y\|_{\max} \leq C \|y\|. \]
        \Yang{To be added.}
    \end{proof}

    \begin{remark}\label{rmk:finite-dimensional_vector_space_over_non_complete_fields}
        If the base field \(\kk\) is not complete, then \cref{prop:norm_on_finite_dimensional_vector_space_are_equivalent} may fail.
        For example, let \(\kk = \bbQ\) with the usual absolute value, and let \(V = \bbQ[\alpha]\) with \(\alpha^2-\alpha-1=0\).
        There are two embeddings of \(V\) into \(\bbR\):
        \[ \iota_1: a + b\alpha \mapsto a + b\frac{1+\sqrt{5}}{2}, \quad \iota_2: a + b\alpha \mapsto a + b\frac{1-\sqrt{5}}{2}. \]
        Define two norms on \(V\) by
        \[
            \|x\|_1 := |\iota_1(x)|, \quad \|x\|_2 := |\iota_2(x)|,
        \]
        where \(|\cdot|\) is the usual absolute value on \(\bbR\).
        Then \(\|\cdot\|_1\) and \(\|\cdot\|_2\) are not equivalent since \(\iota_2(\alpha^n) \to 0\) as \(n \to \infty\) while \(\iota_1(\alpha^n) \to \infty\).
    \end{remark}

    The following lemma is a classical result in functional analysis, which will be used in the next subsection.

    \begin{lemma}\label{prop:operator_norm_on_M_n_k_with_k_complete}
        Let \(\kk\) be a complete field and \(V\) a normed finite-dimensional vector space over \(\kk\).
        Then 
        \[ \|\cdot\| : \End_{\kk}(V) \to \bbR_{\geq 0}, \quad T \mapsto \sup_{x \in V \setminus \{0\}} \frac{\|T(x)\|}{\|x\|} \]
        defines a norm on the \(\kk\)-vector space \(\End_{\kk}(V)\) satisfying
        \[ \|AB\| \leq \|A\| \cdot \|B\|, \quad \forall A, B \in \End_{\kk}(V). \]
    \end{lemma}
    \begin{proof}
        First we show the existence of the supremum, i.e., there exists \(C > 0\) such that for all \(x \in V \setminus \{0\}\), \(\|T(x)\| \leq C \|x\|\).
        Fix a basis \(\{e_1, e_2, \ldots, e_n\}\) of \(V\) and let \(\|\cdot\|_{\max}\) be the maximal norm with respect to this basis.
        Since all norms on \(V\) are bounded by each other by \cref{prop:norm_on_finite_dimensional_vector_space_are_equivalent}, we only need to show that there exists \(C > 0\) such that for all \(x \in V \setminus \{0\}\), \(\|T(x)\|_{1} \leq C \|x\|_{\max}\).
        Write \(T(e_i) = \sum_{j=1}^n a_{ij} e_j\) for \(1 \leq i \leq n\).
        For any \(x = \sum_{i=1}^n x_i e_i \in V\), we have
        \[ \|T(x)\|_1 = \left\|\sum_{j=1}^n \left(\sum_{i=1}^n a_{ij} x_i\right) e_j \right\|_1 = \sum_{j=1}^n \left|\sum_{i=1}^n a_{ij} x_i\right| \leq \left(\sum_{1 \leq i, j \leq n} |a_{ij}|\right) \|x\|_{\max}. \]
        Thus the supremum is finite.

        The linearity and positive-definiteness of \(\|\cdot\|\) are clear.
        It remains to show the triangle inequality and sub-multiplicativity.
        For any \(A, B \in \End_{\kk}(V)\), we have
        \[ \frac{\|(A + B)(x)\|}{\|x\|} = \frac{\|A(x)\|}{\|x\|} + \frac{\|B(x)\|}{\|x\|} \leq \|A\| + \|B\|. \]
        Taking supremum over all \(x \in V \setminus \{0\}\) gives \(\|A + B\| \leq \|A\| + \|B\|\).
        We have 
        \[ \|AB(x)\| \leq \|A\| \cdot \|B(x)\| \leq \|A\| \cdot \|B\| \cdot \|x\| \]
        and hence \(\|AB(x)\|/\|x\| \leq \|A\| \cdot \|B\|\).
        Taking supremum we get \(\|AB\| \leq \|A\| \cdot \|B\|\). 
    \end{proof}

\subsection{Finite field extensions}

    \begin{lemma}\label{prop:existence_of_absolute_value_on_finite_extension_of_complete_fields}
        Let \(\kk\) be a complete field and \(\bfl\) a finite extension of \(\kk\).
        Then there exists an absolute value on \(\bfl\) extending the absolute value on \(\kk\).
    \end{lemma}
    \begin{proof}
        Fix a norm \(\|\cdot\|_V\) on the \(\kk\)-vector space \(V = \bfl\).
        The norm \(\|\cdot\|_V\) induces an operator norm \(\|\cdot\|_{\op}\) on the \(\kk\)-vector space \(\End_{\kk}(V)\) as in \cref{prop:operator_norm_on_M_n_k_with_k_complete}.
        For any \(a \in \bfl\), let \(\mu_a \in \End_{\kk}(V)\) be the \(\kk\)-linear map defined by multiplication by \(a\).
        Note that \(a \mapsto \mu_a\) gives a embedding of \(\kk\)-algebras and if \(a \in \kk\), \(\|\mu_a\|_{\op} = \|a\|_{\kk}\).
        Thus the restriction of \(\|\cdot\|_{\op}\) to \(\bfl\) gives an norm on \(\bfl\) extending that on \(\kk\).
        The normed ring \((\bfl, \|\cdot\|_{\op})\) is a Banach ring since it is a finite-dimensional vector space over the complete field \(\kk\).
        By \cref{thm:norm_spectrum_of_Banach_rings_is_nonempty}, there exists a multiplicative seminorm \(\|\cdot\|_{\bfl}\) on \(\bfl\) bounded by \(\|\cdot\|_{\op}\).
        In particular, \(\|\cdot\|_{\bfl}\) is bounded by \(\|\cdot\|_{\kk}\) on \(\kk\).
        On a field, if one norm is bounded by another norm, then they must be equal (consider the inverse elements).
        Thus \(\|\cdot\|_{\bfl}\) extends the absolute value on \(\kk\).
    \end{proof}

    \begin{theorem}\label{prop:absolute_value_on_finite_extension_of_complete_fields}
        Let \(\kk\) be a complete field and \(\bfl\) a finite extension of \(\kk\).
        Then the absolute value on \(\bfl\) which extends the absolute value on \(\kk\) is uniquely determined by the absolute value on \(\kk\).
        Furthermore, we have 
        \[ \|\cdot\|_{\bfl} = \|N_{\bfl/\kk}(\cdot)\|_{\kk}^{1/n}, \]
        where \(n = [\bfl : \kk]\) and \(N_{\bfl/\kk}\) is the norm map from \(\bfl\) to \(\kk\).
        % \Yang{To be checked.}
    \end{theorem}
    \begin{proof}
        Let \(\|\cdot\|_{\bfl}\) be arbitrary absolute value on \(\bfl\) extending that on \(\kk\).
        We will show that \(\|\cdot\|_{\bfl}\) must be equal to \(\|N_{\bfl/\kk}(\cdot)\|_{\kk}^{1/n}\).
        For any \(a \in \bfl\), set \(b = a^n/N_{\bfl/\kk}(a) \in \bfl\).
        Then \(N_{\bfl/\kk}(b) = 1\) and 
        \[ \|b\|_{\bfl} = \frac{\|a\|_{\bfl}^n}{\|N_{\bfl/\kk}(a)\|_{\kk}}. \]
        Thus it suffices to show that \(\|b\|_{\bfl} = 1\) whenever \(N_{\bfl/\kk}(b) = 1\).

        Note that the norm map \(N_{\bfl/\kk}: \bfl \to \kk\) is the determinant of the \(\kk\)-linear map \(\mu_b \in \End_{\kk}(V)\) defined by multiplication by \(b\).
        Hence it is continuous on \(\bfl\) (since it is a polynomial in the entries of the matrix representation).
        If \(\|b\|_{\bfl} < 1\), then \(\|b^m\|_{\bfl} \to 0\) as \(m \to \infty\).
        Thus \(N_{\bfl/\kk}(b^m) = \det(\mu_{b^m}) \to 0\) as \(m \to \infty\), contradicting the fact that \(N_{\bfl/\kk}(b^m) = 1\) for all \(m\).
        Similarly, if \(\|b\|_{\bfl} > 1\), then just consider \(b^{-1}\).

        % Fix a norm \(\|\cdot\|_V\) on the \(\kk\)-vector space \(V = \bfl\).
        % The norm \(\|\cdot\|_V\) induces an operator norm \(\|\cdot\|_{\op}\) on the \(\kk\)-vector space \(\End_{\kk}(V)\) as in \cref{prop:operator_norm_on_M_n_k_with_k_complete}.
        % For any \(a \in \bfl\), let \(\mu_a \in \End_{\kk}(V)\) be the \(\kk\)-linear map defined by multiplication by \(a\).
        % Note that \(a \mapsto \mu_a\) is a \(\kk\)-algebra homomorphism and if \(a \in \kk\), \(\|\mu_a\|_{\op} = \|a\|_{\kk}\).

        % \begin{step}\label{step_in_thm_absolute_value_on_finite_extension_of_complete_fields:the_field_norm_is_the_spectral_radius}
        %     Show that 
        %     \[ \|N_{\bfl/\kk}(a)\|_{\kk} = \lim_{m \to \infty} \sqrt[m]{\|\mu_{a^{mn}}\|_{\op}} \eqqcolon \rho(a^n). \]
        % \end{step}

        % The existence of the limit \(\rho(a) = \lim_{m \to \infty} \sqrt[m]{\|\mu_{a^m}\|_{\op}}\) follows from the sub-multiplicativity of the operator norm.
        % And note that \(\rho(a^n) = \rho(a)^n\) for all \(a \in \bfl\) and \(n \in \bbN\).
        % We can assume that \(a \neq 0\).
        % Let \(b = a^n/N_{\bfl/\kk}(a) \in \bfl\).
        % Then \(N_{\bfl/\kk}(b) = 1\) and 
        % \[ \rho(b) = \lim_{m \to \infty} \sqrt[m]{\|\mu_{b^m}\|_{\op}} = \lim_{m \to \infty} \sqrt[m]{\frac{\|\mu_{a^{mn}}\|_{\op}}{\|N_{\bfl/\kk}(a)^m\|_{\kk}}} = \frac{\rho(a^n)}{\|N_{\bfl/\kk}(a)\|_{\kk}} \]
        % since \(N_{\bfl/\kk}\) is multiplicative and \(N_{\bfl/\kk}(a) \in \kk\).
        % Thus it suffices to show that \(\rho(b) = 1\). 

        % Suppose that \(\rho(b) < 1\).
        % Then there exists \(\delta \in (0,1)\) and \(M > 0\) such that for all \(m \geq M\), \(\|\mu_{b^m}\|_{\op} < \delta^m\).
        % In particular, we have \(\mu_{b^m} \to 0\) as \(m \to \infty\) in the normed vector space \(\End_{\kk}(V)\).
        % However, note that \(\det\) is continuous on \(\End_{\kk}(V)\) (since it is a polynomial in the entries of the matrix representation).
        % Thus we have \(N_{\bfl/\kk}(b^m) = \det(\mu_{b^m}) \to 0\) as \(m \to \infty\), contradicting the fact that \(N_{\bfl/\kk}(b^m) = 1\) for all \(m\).
        
        % if \(\rho(b) > 1\), 
        % \Yang{We can not get \(\rho(b^{-1}) < 1\) directly.}

        % \begin{step}\label{step_in_thm_absolute_value_on_finite_extension_of_complete_fields:triangle_inequality}
        %     Show that 
        %     \[ \rho(a + b) \leq \rho(a) + \rho(b) \]
        %     and hence \(\|\cdot\|_{\bfl} = \|N_{\bfl/\kk}(\cdot)\|_{\kk}^{1/n}\) defines an absolute value on \(\bfl\) extending that on \(\kk\).
        % \end{step}


        % \begin{step}\label{step_in_thm_absolute_value_on_finite_extension_of_complete_fields:uniqueness_of_absolute_value}
        %     Show the uniqueness of the absolute value on \(\bfl\) extending that on \(\kk\).
        % \end{step}




        % \Yang{To be added.}
    \end{proof}

    % \begin{remark}\label{rmk:compatiblity_of_extension_and_completion}
    %     \Yang{I want to discuss some compatiblity of extension and completion.}
    % \end{remark}

    \begin{proposition}\label{prop:completion_of_algebraically_closed_valuation_fields_is_algebraically_closed}
        Let \(\kk\) be an algebraically closed non-archimedean field.
        Then its completion \(\widehat{\kk}\) is also algebraically closed.
        % \Yang{To be checked.}
    \end{proposition}
    \begin{proof}
        Let \(f \in \widehat{\kk}[X]\) be a non-constant polynomial.
        We will show that \(f\) has a root in \(\widehat{\kk}\).
        Take a sequence of polynomials \(\{f_n\}_{n \in \bbN}\) in \(\kk[X]\) converging to \(f\) coefficient-wisely and of the same degree \(d\).
        Since \(\kk\) is algebraically closed, each \(f_n\) splits completely in \(\kk\) and hence in \(\widehat{\kk}\).
        Write \(f_n(X) = \prod_{i=1}^d (X - \alpha_{n,i})\) with \(\alpha_{n,i} \in \widehat{\kk}\).

        Let \(\bfl\) be a finite extension of \(\widehat{\kk}\) such that \(f\) has a root \(\alpha\) in \(\bfl\).
        For every \(\varepsilon > 0\), if there are infinitely many \(n\) such that \(\alpha_{n,i} \notin B(\alpha, \varepsilon)\) for all \(1 \leq i \leq d\), 
        then we have \(|f_n(\alpha)| \geq \varepsilon^d\) for infinitely many \(n\), contradicting the fact that \(f_n(\alpha) \to f(\alpha) = 0\).
        Thus for every \(\varepsilon > 0\), there exists \(N > 0\) such that for all \(n \geq N\), there exists \(1 \leq i \leq d\) with \(\alpha_{n,i} \in B(\alpha, \varepsilon)\).
        That is, we can find a sequence \(\alpha_{n,i_n} \in \kk\) converging to \(\alpha\).
        Since \(\widehat{\kk}\) is complete, we have \(\alpha \in \widehat{\kk}\).
        % \Yang{To be added.}
    \end{proof}


   \chapter{Affinoid spaces}
      \section{Spectrum}

    % Let \(A\) be an integral \(\kk\)-algebra with \(\kk\) complete non-archimedean algebraically closed field.
    % For every \(\frakm \in \Spec A\), the absolute value on \(A/\frakm \cong \kk\) induces a multiplicative semi-norm on \(A\) by \(|f|_\frakm := |f \mod \frakm|\) for each \(f \in A\).

    % Suppose that we have an absolute value \(\|\cdot\|\) on \(\KK = \Frac A\) extending that on \(\kk\).
    % It induces a norm on \(A\) by restriction, denoted by \(\|\cdot\|_\eta\).

    % Then we can ``deform'' \(\|\cdot\|_\eta\) to \(|\cdot|_\frakm\) as follows:

    Let \(\kk\) be a spherically complete non-archimedean field which is algebraically closed and \(A = \kk[T]\).
    We want to consider the ``analytic structure'' on \(\mSpec A\).
    However, unlike the complex case, the set \(\mSpec A\) is totally disconnected with respect to the topology induced by the absolute value on \(\kk\) (\cref{prop:ultra-metric_space_is_totally_disconnected}).
    To overcome this difficulty, Berkovich uses multiplicative semi-norms to ``fill in the gaps'' between the points in \(\mSpec A\), leading to the notion of the spectrum of a Banach ring.

    We first consider the local model.
    Hence we should consider the Tate algebra \(\kk\{T\}\) instead of the polynomial ring \(\kk[T]\).
    \Yang{The maximal ideal of \(\kk\{T\}\) corresponding to the point in the disk \(\{a \in \kk\colon a \leq 1 \}\)}.
    \Yang{Closed or open disk?}

\subsection{Definition}

    \begin{definition}\label{def:spectrum_of_Banach_rings}
        Let \(R\) be a Banach ring.
        The \emph{spectrum} \(\scrM(R)\) of \(R\) is defined as the set of all multiplicative semi-norms on \(R\) that are bounded with respect to the given norm on \(R\).
        For every point \(x \in \scrM(R)\), we denote the corresponding multiplicative semi-norm by \(|\cdot|_x\).
        We equip \(\scrM(R)\) with the weakest topology such that for each \(f \in R\), the evaluation map \(\scrM(R) \to \bbR_{\geq 0}\), defined by \(x \mapsto |f|_x\eqqcolon f(x)\), is continuous.
    \end{definition}
    
    \begin{definition}\label{def:pullback_of_ring_homomorphism_of_banach_rings_on_spectrum}
        Let \(\varphi: R \to S\) be a bounded ring homomorphism of Banach rings.
        The \emph{pullback} map \(\scrM(\varphi): \scrM(S) \to \scrM(R)\) is defined by \(\scrM(\varphi)(x) = x \circ \varphi: f \mapsto |\varphi(f)|_x\) for each \(x \in \scrM(S)\).
        % \Yang{To be revised.}
    \end{definition}

    % For \(x \in \scrM(R)\), \Yang{the kernel of the multiplicative semi-norm \(|\cdot|_x\) is a closed prime ideal of \(R\)}, denoted by \(\wp_x\).
    % The semi-norm \(|\cdot|_x\) induces a multiplicative norm on the residue field \(\rkk(x) = \Frac(R/\wp_x)\), denoted by \(|\cdot|_{x}\) as well. 

    % \begin{definition}\label{def:character_of_banach_rings}
    %     Let \(R\) be a Banach ring.
    %     A \emph{character} of \(R\) is a bounded ring homomorphism \(\chi: R \to K\), where \(K\) is a completed field.
    %     Two characters \(\chi_1: R \to K_1\) and \(\chi_2: R \to K_2\) are said to be \emph{equivalent} if there exists a commutative diagram of bounded ring homomorphisms
    %     \[
    %         \begin{tikzcd}
    %             & R \arrow[dl, "\chi_1"'] \arrow[dr, "\chi_2"] \arrow[d] & \\
    %             K_1 & K \arrow[l,hook'] \arrow[r,hook] & K_2
    %         \end{tikzcd}
    %     \]
    %     for some completed field \(K\).
    %     % \Yang{This is wrong, need to revised.}
    % \end{definition}

    % \begin{proposition}\label{prop:spectrum_of_banach_rings_and_equivalence_class_of_characters}
    %     Let \(R\) be a Banach ring.
    %     The spectrum \(\scrM(R)\) is in bijection with the equivalence classes of characters of \(R\).
    % \end{proposition}
    % \begin{proof}
    %     \Yang{To be completed}
    % \end{proof}

    \begin{proposition}\label{prop:map_from_M_R_to_Spec_R}
        Let \(R\) be a Banach ring.
        For each \(x \in \scrM(R)\), let \(\wp_x\) be the kernel of the multiplicative semi-norm \(|\cdot|_x\).
        Then \(\wp_x\) is a closed prime ideal of \(R\), and \(x \mapsto \wp_x\) defines a continuous map from \(\scrM(R)\) to \(\Spec(R)\) equipped with the Zariski topology.
    \end{proposition}
    \begin{proof}
        \Yang{To be completed}
    \end{proof}

    \begin{definition}
        Let \(R\) be a Banach ring.
        For each \(x \in \scrM(R)\), the \emph{completed residue field} at the point \(x\) is defined as the completion of the residue field \(\rkk(x) = \Frac(R/\wp_x)\) with respect to the multiplicative norm induced by the semi-norm \(|\cdot|_x\), denoted by \(\scrH(x)\).
    \end{definition}

    \begin{definition}\label{def:Gelfand_transform_of_banach_rings}
        Let \(R\) be a Banach ring.
        The \emph{Gel'fand transform} of \(R\) is the bounded ring homomorphism
        \[
            \Gamma: R \to \prod_{x \in \scrM(R)} \scrH(x), \quad f \mapsto (f(x))_{x \in \scrM(R)},
        \]
        where the norm on the product \(\prod_{x \in \scrM(R)} \scrH(x)\) is given by the supremum norm.
    \end{definition}

    \begin{proposition}\label{prop:Gelfand_transform_and_the_uniformization_of_a_banach_ring}
        The Gel'fand transform \(\Gamma: R \to \prod_{x \in \scrM(R)} \scrH(x)\) of a Banach ring \(R\) factors through the uniformization \(R^u\) of \(R\), and the induced map \(R^u \to \prod_{x \in \scrM(R)} \scrH(x)\) is an isometric embedding.
        \Yang{To be checked.}
    \end{proposition}

    \begin{theorem}\label{thm:norm_spectrum_of_Banach_rings_is_nonempty}
        Let \(R\) be a Banach ring.
        The spectrum \(\scrM(R)\) is nonempty.
    \end{theorem}
    \begin{proof}
        \Yang{To be continued.}
    \end{proof}

    \begin{lemma}\label{lem:spectrum_of_product_of_completed_fields}
        Let \(\{K_i\}_{i \in I}\) be a family of completed fields.
        Consider the Banach ring \(R = \prod_{i \in I} K_i\) equipped with the product norm.
        The spectrum \(\scrM(R)\) is homeomorphic to the Stone-\v{C}ech compactification of the discrete space \(I\).
    \end{lemma}

    \begin{remark}\label{rmk:some_fact_about_the_Stone_Cech_compactification}
        The Stone-\v{C}ech compactification of a discrete space is the largest compact Hausdorff space in which the original space can be densely embedded.
        \Yang{To be checked.}
    \end{remark}

    \begin{theorem}\label{norm_spectrum_of_Banach_rings_is_compact_Hausdorff}
        Let \(R\) be a Banach ring.
        The spectrum \(\scrM(R)\) is a compact Hausdorff space.
    \end{theorem}
    \begin{proof}
        \Yang{To be added.}
    \end{proof}

    \begin{proposition}\label{prop:the_Galois_action_on_the_spectrum_of_banach_rings}
        Let \(K/k\) be a Galois extension of complete fields, and let \(R\) be a Banach \(k\)-algebra.
        The Galois group \(\Gal(K/k)\) acts on the spectrum \(\scrM(R \widehat{\otimes}_k K)\) via
        \[
            g \cdot x: f \mapsto |(1 \otimes g^{-1})(f)|_x
        \]
        for each \(g \in \Gal(K/k)\), \(x \in \scrM(R \widehat{\otimes}_k K)\) and \(f \in R \widehat{\otimes}_k K\).
        Moreover, the natural map \(\scrM(R \widehat{\otimes}_k K) \to \scrM(R)\) induces a homeomorphism
        \[
            \scrM(R \widehat{\otimes}_k K) / \Gal(K/k) \xrightarrow{\sim} \scrM(R).
        \]
        \Yang{To be checked.}
    \end{proposition}


\subsection{Examples}

    \begin{example}\label{eg:spectrum_of_valuation_field}
        Let \((\kk, |\cdot|)\) be a complete valuation field.
        The spectrum \(\scrM(\kk)\) consists of a single point corresponding to the given absolute value \(|\cdot|\) on \(\kk\).
        \Yang{To be checked.}
    \end{example}

    \begin{example}\label{eg:spectrum_of_Z_with_absolute_value_norm}
        Consider the Banach ring \((\bbZ, \|\cdot\|)\) with \(\|\cdot\| = |\cdot|_\infty\) is the usual absolute value norm on \(\bbZ\).
        Let \(|\cdot|_p\) denote the \(p\)-adic norm for each prime number \(p\), i.e., \(|n|_p = p^{-v_p(n)}\) for each \(n \in \bbZ\), where \(v_p(n)\) is the \(p\)-adic valuation of \(n\).
        The spectrum 
        \[ \scrM(\bbZ) = \{|\cdot|_{\infty}^\varepsilon \colon \varepsilon \in (0,1]\} \cup \{| \cdot |_p^\alpha \colon p \text{ is prime}, \alpha \in (0,\infty] \} \cup \{|\cdot|_0\}, \]
        where \(|a|_p^\infty := \lim_{\alpha \to \infty} |a|_p^\alpha\) for each \(a \in \bbZ\) and \(|\cdot|_0\) is the trivial norm on \(\bbZ\).
        \Yang{To be checked.}
    \end{example}

    % \begin{example}\label{eg:spectrum_of_Tate_algebras}
    \paragraph{Spectrum of Tate algebra in one variable} Let \(\kk\) be a complete non-archimedean field, and let \(A = \kk\{T/r\}\).
    We list some types of points in the spectrum \(\scrM(A)\).

    For each \(a \in \kk\) with \(|a| \leq r\), we have the \emph{type I} point \(x_a\) corresponding to the evaluation at \(a\), i.e., \(|f|_{x_a} := |f(a)|\) for each \(f \in A\).
    For each closed disk \(E = E(a,s) := \{b \in \kk : |b - a| \leq s\}\) with center \(a \in \kk\) and radius \(s \leq r\), we have the point \(x_{a,s}\) corresponding to the multiplicative semi-norm defined by
    \[ |f|_{x_E} := \sup_{b \in E(a,s)} |f(b)| \]
    for each \(f \in A\).
    If \(s \in |\kk^\times|\), then the point \(x_E\) is called a \emph{type II} point; otherwise, it is called a \emph{type III} point.
    
    Let \(\{E^{(s)}\}_s\) be a family of closed disks in \(\kk\) such that \(E^{(s)}\) is of radius \(s\), \(E^{(s_1)} \subsetneq E^{(s_2)}\) for any \(s_1 < s_2\) and \(\bigcap_s E^{(s)} = \emptyset\).
    Then we have the point \(x_{\{E^{(s)}\}}\) corresponding to the multiplicative semi-norm defined by
    \[ |f|_{x_{\{E^{(s)}\}}} := \inf_s |f|_{x_{E^{(s)}}} \]
    for each \(f \in A\).
    Such a point is called a \emph{type IV} point.

    \Yang{To be completed.}
    % \end{example}

    \begin{proposition}\label{prop:four_types_of_points_in_spectrum_of_Tate_algebra_in_one_variable}
        Let \(\kk\) be a complete non-archimedean field, and let \(r > 0\) be a positive real number.
        Consider the Tate algebra \(\kk\{r^{-1}T\}\) equipped with the Gauss norm.
        The points in the spectrum \(\scrM(\kk\{r^{-1}T\})\) can be classified into four types as described above.
        \Yang{To be checked}
    \end{proposition}
    \begin{proof}
        \Yang{To be completed.}
    \end{proof}

    \begin{proposition}\label{prop:the_complete_residue_field_of_all_four_types_points_in__spectrum_of_Tate_algebra_in_one_variable}
        Let \(\kk\) be a complete non-archimedean field, and let \(r > 0\) be a positive real number.
        Consider the Tate algebra \(\kk\{r^{-1}T\}\) equipped with the Gauss norm.
        The completed residue fields of the four types of points in the spectrum \(\scrM(\kk\{r^{-1}T\})\) are described as follows:
        \begin{itemize}
            \item For a type I point \(x_a\) with \(a \in \kk\) and \(|a| \leq r\), the completed residue field \(\scrH(x_a)\) is isomorphic to \(\kk\).
            \item For a type II point \(x_{a,s}\) with \(a \in \kk\) and \(s \in |\kk^\times|\), the completed residue field \(\scrH(x_{a,s})\) is isomorphic to the field of Laurent series over the residue field \(\calk_\kk\), i.e., \(\calk_\kk((t))\).
            \item For a type III point \(x_{a,s}\) with \(a \in \kk\) and \(s \notin |\kk^\times|\), the completed residue field \(\scrH(x_{a,s})\) is isomorphic to a transcendental extension of \(\calk_\kk\) of degree one.
            \item For a type IV point \(x_{\{E^{(s)}\}}\), the completed residue field \(\scrH(x_{\{E^{(s)}\}})\) is isomorphic to a transcendental extension of \(\calk_\kk\) of infinite degree.
        \end{itemize}
        \Yang{To be checked.}
    \end{proposition}

    \begin{example}\label{eg:completed_residue_field_in_spectrum_of_Tate_algebra_in_one_variable_over_Q_p}
        The completed residue field \(\scrH(x_a)\) for a type I point \(x_a\) with \(a \in \kk\) and \(|a| \leq r\) is isomorphic to \(\kk\).
        \Yang{To be complete.}
    \end{example}

    \paragraph{Spectrum of Tate algebra in servel variables} Let \(\kk\) be a complete non-archimedean field, and let \(A = \kk\{r_1^{-1}T_1, \ldots, r_n^{-1}T_n\}\).
    We can consider the spectrum \(\scrM(A)\) similarly.
      \section{Finite field extensions}

\subsection{Finite-dimensional vector space}

    \begin{definition}\label{def:norm_on_vector_space_over_valuation_field}
        Let \(\kk\) be a valuation field and \(V\) a vector space over \(\kk\).
        A \emph{norm} on \(V\) is a function \(\|\cdot\|: V \to \bbR_{\geq 0}\) satisfying the following properties for all \(x, y \in V\) and \(a \in \kk\):
        \begin{enumerate}
            \item \(\|x\| = 0\) if and only if \(x = 0\);
            \item \(\|a x\| = |a| \cdot \|x\|\);
            \item \(\|x + y\| \leq \|x\| + \|y\|\).
        \end{enumerate}
    \end{definition}

    \begin{example}\label{def:maximal_norm_on_finite-dimensional_vector_space}
        Let \(\kk\) be a valuation field and \(V\) a finite-dimensional vector space over \(\kk\) with basis \(\{e_1, e_2, \ldots, e_n\}\).
        For any \(x = a_1 e_1 + a_2 e_2 + \cdots + a_n e_n \in V\), define
        \[
            \|x\|_{\max} := \max_{1 \leq i \leq n} |a_i|.
        \]
        Then \(\|\cdot\|_{\max}\) is a norm on \(V\), called the \emph{maximal norm} with respect to the basis \(\{e_1, e_2, \ldots, e_n\}\).
    \end{example}

    \begin{example}\label{def:1-norm_no_finite-dimensional_vector_space}
        Setting as in \cref{def:maximal_norm_on_finite-dimensional_vector_space}, for any \(x = a_1 e_1 + a_2 e_2 + \cdots + a_n e_n \in V\), define
        \[
            \|x\|_1 := |a_1| + |a_2| + \cdots + |a_n|.
        \]
        Then \(\|\cdot\|_1\) is also a norm on \(V\).
    \end{example}

    \begin{definition}\label{def:equivalent_norm_on_vector_space}
        Let \(\kk\) be a valuation field and \(V\) a vector space over \(\kk\).
        Two norms \(\|\cdot\|_1\) and \(\|\cdot\|_2\) on \(V\) are said to be \emph{equivalent} if there exist positive constants \(C_1, C_2 > 0\) such that for all \(x \in V\),
        \[
            C_1 \|x\|_1 \leq \|x\|_2 \leq C_2 \|x\|_1.
        \]
    \end{definition}

    \begin{lemma}\label{prop:norms_are_equivalent_iff_induces_the_same_topology}
        Let \(\kk\) be a valuation field and \(V\) a vector space over \(\kk\).
        Two norms \(\|\cdot\|_1\) and \(\|\cdot\|_2\) on \(V\) are equivalent if and only if they induce the same topology on \(V\).
    \end{lemma}
    \begin{proof}
        The sufficiency is clear.
        Now suppose that \(\|\cdot\|_1\) and \(\|\cdot\|_2\) induce the same topology on \(V\).
        Hence the unit open ball with respect to \(\|\cdot\|_1\) contains a unit open ball with respect to \(\|\cdot\|_2\).
        That is, 
        \[ \{x \in V : \|x\|_1 < 1\} \supseteq \{x \in V : \|x\|_2 < C\}. \]
        Then for every \(x \in V\) with \(\|x\|_1 = 1\), we have \(\|x\|_2 \geq C = C \|x\|_1\).
        By scaling, we get that for every \(x \in V\),
        \[ \|x\|_2 \geq C \|x\|_1. \]
        Similar for the other direction, we conclude that \(\|\cdot\|_1\) and \(\|\cdot\|_2\) are equivalent.
    \end{proof}

    \begin{proposition}\label{prop:finite_dimensional_vector_space_over_complete_fields_is_complete}
        Let \(V\) be a normed finite-dimensional vector space over a complete valuation field \(\kk\).
        Then \(V\) is complete.
    \end{proposition}
    \begin{proof}
        \Yang{To be added.}
    \end{proof}

    \begin{theorem}\label{prop:norm_on_finite_dimensional_vector_space_are_equivalent}
        Let \(V\) be a finite-dimensional vector space over a complete field \(\kk\).
        Then all norms on \(V\) are equivalent.
    \end{theorem}
    \begin{proof}
        Fix a basis \(\{e_1, e_2, \ldots, e_n\}\) of \(V\) and let \(\|\cdot\|_{\max}\) be the maximal norm with respect to this basis as in \cref{def:maximal_norm_on_finite-dimensional_vector_space}.
        Let \(\|\cdot\|\) be any norm on \(V\).
        It suffices to show that \(\|\cdot\|\) and \(\|\cdot\|_{\max}\) are equivalent.
        First we have 
        \[ \|y\| \leq \sum_{i=1}^n |a_i| \|e_i\| \leq \left(\sum_{i=1}^n \|e_i\|\right) \|y\|_{\max} \]
        for any \(y = a_1 e_1 + a_2 e_2 + \cdots + a_n e_n \in V\).
        It remains to show that there exists a constant \(C > 0\) such that for any \(y \in V\),
        \[ \|y\|_{\max} \leq C \|y\|. \]
        \Yang{To be added.}
    \end{proof}

    \begin{remark}\label{rmk:finite-dimensional_vector_space_over_non_complete_fields}
        If the base field \(\kk\) is not complete, then \cref{prop:norm_on_finite_dimensional_vector_space_are_equivalent} may fail.
        For example, let \(\kk = \bbQ\) with the usual absolute value, and let \(V = \bbQ[\alpha]\) with \(\alpha^2-\alpha-1=0\).
        There are two embeddings of \(V\) into \(\bbR\):
        \[ \iota_1: a + b\alpha \mapsto a + b\frac{1+\sqrt{5}}{2}, \quad \iota_2: a + b\alpha \mapsto a + b\frac{1-\sqrt{5}}{2}. \]
        Define two norms on \(V\) by
        \[
            \|x\|_1 := |\iota_1(x)|, \quad \|x\|_2 := |\iota_2(x)|,
        \]
        where \(|\cdot|\) is the usual absolute value on \(\bbR\).
        Then \(\|\cdot\|_1\) and \(\|\cdot\|_2\) are not equivalent since \(\iota_2(\alpha^n) \to 0\) as \(n \to \infty\) while \(\iota_1(\alpha^n) \to \infty\).
    \end{remark}

    The following lemma is a classical result in functional analysis, which will be used in the next subsection.

    \begin{lemma}\label{prop:operator_norm_on_M_n_k_with_k_complete}
        Let \(\kk\) be a complete field and \(V\) a normed finite-dimensional vector space over \(\kk\).
        Then 
        \[ \|\cdot\| : \End_{\kk}(V) \to \bbR_{\geq 0}, \quad T \mapsto \sup_{x \in V \setminus \{0\}} \frac{\|T(x)\|}{\|x\|} \]
        defines a norm on the \(\kk\)-vector space \(\End_{\kk}(V)\) satisfying
        \[ \|AB\| \leq \|A\| \cdot \|B\|, \quad \forall A, B \in \End_{\kk}(V). \]
    \end{lemma}
    \begin{proof}
        First we show the existence of the supremum, i.e., there exists \(C > 0\) such that for all \(x \in V \setminus \{0\}\), \(\|T(x)\| \leq C \|x\|\).
        Fix a basis \(\{e_1, e_2, \ldots, e_n\}\) of \(V\) and let \(\|\cdot\|_{\max}\) be the maximal norm with respect to this basis.
        Since all norms on \(V\) are bounded by each other by \cref{prop:norm_on_finite_dimensional_vector_space_are_equivalent}, we only need to show that there exists \(C > 0\) such that for all \(x \in V \setminus \{0\}\), \(\|T(x)\|_{1} \leq C \|x\|_{\max}\).
        Write \(T(e_i) = \sum_{j=1}^n a_{ij} e_j\) for \(1 \leq i \leq n\).
        For any \(x = \sum_{i=1}^n x_i e_i \in V\), we have
        \[ \|T(x)\|_1 = \left\|\sum_{j=1}^n \left(\sum_{i=1}^n a_{ij} x_i\right) e_j \right\|_1 = \sum_{j=1}^n \left|\sum_{i=1}^n a_{ij} x_i\right| \leq \left(\sum_{1 \leq i, j \leq n} |a_{ij}|\right) \|x\|_{\max}. \]
        Thus the supremum is finite.

        The linearity and positive-definiteness of \(\|\cdot\|\) are clear.
        It remains to show the triangle inequality and sub-multiplicativity.
        For any \(A, B \in \End_{\kk}(V)\), we have
        \[ \frac{\|(A + B)(x)\|}{\|x\|} = \frac{\|A(x)\|}{\|x\|} + \frac{\|B(x)\|}{\|x\|} \leq \|A\| + \|B\|. \]
        Taking supremum over all \(x \in V \setminus \{0\}\) gives \(\|A + B\| \leq \|A\| + \|B\|\).
        We have 
        \[ \|AB(x)\| \leq \|A\| \cdot \|B(x)\| \leq \|A\| \cdot \|B\| \cdot \|x\| \]
        and hence \(\|AB(x)\|/\|x\| \leq \|A\| \cdot \|B\|\).
        Taking supremum we get \(\|AB\| \leq \|A\| \cdot \|B\|\). 
    \end{proof}

\subsection{Finite field extensions}

    \begin{lemma}\label{prop:existence_of_absolute_value_on_finite_extension_of_complete_fields}
        Let \(\kk\) be a complete field and \(\bfl\) a finite extension of \(\kk\).
        Then there exists an absolute value on \(\bfl\) extending the absolute value on \(\kk\).
    \end{lemma}
    \begin{proof}
        Fix a norm \(\|\cdot\|_V\) on the \(\kk\)-vector space \(V = \bfl\).
        The norm \(\|\cdot\|_V\) induces an operator norm \(\|\cdot\|_{\op}\) on the \(\kk\)-vector space \(\End_{\kk}(V)\) as in \cref{prop:operator_norm_on_M_n_k_with_k_complete}.
        For any \(a \in \bfl\), let \(\mu_a \in \End_{\kk}(V)\) be the \(\kk\)-linear map defined by multiplication by \(a\).
        Note that \(a \mapsto \mu_a\) gives a embedding of \(\kk\)-algebras and if \(a \in \kk\), \(\|\mu_a\|_{\op} = \|a\|_{\kk}\).
        Thus the restriction of \(\|\cdot\|_{\op}\) to \(\bfl\) gives an norm on \(\bfl\) extending that on \(\kk\).
        The normed ring \((\bfl, \|\cdot\|_{\op})\) is a Banach ring since it is a finite-dimensional vector space over the complete field \(\kk\).
        By \cref{thm:norm_spectrum_of_Banach_rings_is_nonempty}, there exists a multiplicative seminorm \(\|\cdot\|_{\bfl}\) on \(\bfl\) bounded by \(\|\cdot\|_{\op}\).
        In particular, \(\|\cdot\|_{\bfl}\) is bounded by \(\|\cdot\|_{\kk}\) on \(\kk\).
        On a field, if one norm is bounded by another norm, then they must be equal (consider the inverse elements).
        Thus \(\|\cdot\|_{\bfl}\) extends the absolute value on \(\kk\).
    \end{proof}

    \begin{theorem}\label{prop:absolute_value_on_finite_extension_of_complete_fields}
        Let \(\kk\) be a complete field and \(\bfl\) a finite extension of \(\kk\).
        Then the absolute value on \(\bfl\) which extends the absolute value on \(\kk\) is uniquely determined by the absolute value on \(\kk\).
        Furthermore, we have 
        \[ \|\cdot\|_{\bfl} = \|N_{\bfl/\kk}(\cdot)\|_{\kk}^{1/n}, \]
        where \(n = [\bfl : \kk]\) and \(N_{\bfl/\kk}\) is the norm map from \(\bfl\) to \(\kk\).
        % \Yang{To be checked.}
    \end{theorem}
    \begin{proof}
        Let \(\|\cdot\|_{\bfl}\) be arbitrary absolute value on \(\bfl\) extending that on \(\kk\).
        We will show that \(\|\cdot\|_{\bfl}\) must be equal to \(\|N_{\bfl/\kk}(\cdot)\|_{\kk}^{1/n}\).
        For any \(a \in \bfl\), set \(b = a^n/N_{\bfl/\kk}(a) \in \bfl\).
        Then \(N_{\bfl/\kk}(b) = 1\) and 
        \[ \|b\|_{\bfl} = \frac{\|a\|_{\bfl}^n}{\|N_{\bfl/\kk}(a)\|_{\kk}}. \]
        Thus it suffices to show that \(\|b\|_{\bfl} = 1\) whenever \(N_{\bfl/\kk}(b) = 1\).

        Note that the norm map \(N_{\bfl/\kk}: \bfl \to \kk\) is the determinant of the \(\kk\)-linear map \(\mu_b \in \End_{\kk}(V)\) defined by multiplication by \(b\).
        Hence it is continuous on \(\bfl\) (since it is a polynomial in the entries of the matrix representation).
        If \(\|b\|_{\bfl} < 1\), then \(\|b^m\|_{\bfl} \to 0\) as \(m \to \infty\).
        Thus \(N_{\bfl/\kk}(b^m) = \det(\mu_{b^m}) \to 0\) as \(m \to \infty\), contradicting the fact that \(N_{\bfl/\kk}(b^m) = 1\) for all \(m\).
        Similarly, if \(\|b\|_{\bfl} > 1\), then just consider \(b^{-1}\).

        % Fix a norm \(\|\cdot\|_V\) on the \(\kk\)-vector space \(V = \bfl\).
        % The norm \(\|\cdot\|_V\) induces an operator norm \(\|\cdot\|_{\op}\) on the \(\kk\)-vector space \(\End_{\kk}(V)\) as in \cref{prop:operator_norm_on_M_n_k_with_k_complete}.
        % For any \(a \in \bfl\), let \(\mu_a \in \End_{\kk}(V)\) be the \(\kk\)-linear map defined by multiplication by \(a\).
        % Note that \(a \mapsto \mu_a\) is a \(\kk\)-algebra homomorphism and if \(a \in \kk\), \(\|\mu_a\|_{\op} = \|a\|_{\kk}\).

        % \begin{step}\label{step_in_thm_absolute_value_on_finite_extension_of_complete_fields:the_field_norm_is_the_spectral_radius}
        %     Show that 
        %     \[ \|N_{\bfl/\kk}(a)\|_{\kk} = \lim_{m \to \infty} \sqrt[m]{\|\mu_{a^{mn}}\|_{\op}} \eqqcolon \rho(a^n). \]
        % \end{step}

        % The existence of the limit \(\rho(a) = \lim_{m \to \infty} \sqrt[m]{\|\mu_{a^m}\|_{\op}}\) follows from the sub-multiplicativity of the operator norm.
        % And note that \(\rho(a^n) = \rho(a)^n\) for all \(a \in \bfl\) and \(n \in \bbN\).
        % We can assume that \(a \neq 0\).
        % Let \(b = a^n/N_{\bfl/\kk}(a) \in \bfl\).
        % Then \(N_{\bfl/\kk}(b) = 1\) and 
        % \[ \rho(b) = \lim_{m \to \infty} \sqrt[m]{\|\mu_{b^m}\|_{\op}} = \lim_{m \to \infty} \sqrt[m]{\frac{\|\mu_{a^{mn}}\|_{\op}}{\|N_{\bfl/\kk}(a)^m\|_{\kk}}} = \frac{\rho(a^n)}{\|N_{\bfl/\kk}(a)\|_{\kk}} \]
        % since \(N_{\bfl/\kk}\) is multiplicative and \(N_{\bfl/\kk}(a) \in \kk\).
        % Thus it suffices to show that \(\rho(b) = 1\). 

        % Suppose that \(\rho(b) < 1\).
        % Then there exists \(\delta \in (0,1)\) and \(M > 0\) such that for all \(m \geq M\), \(\|\mu_{b^m}\|_{\op} < \delta^m\).
        % In particular, we have \(\mu_{b^m} \to 0\) as \(m \to \infty\) in the normed vector space \(\End_{\kk}(V)\).
        % However, note that \(\det\) is continuous on \(\End_{\kk}(V)\) (since it is a polynomial in the entries of the matrix representation).
        % Thus we have \(N_{\bfl/\kk}(b^m) = \det(\mu_{b^m}) \to 0\) as \(m \to \infty\), contradicting the fact that \(N_{\bfl/\kk}(b^m) = 1\) for all \(m\).
        
        % if \(\rho(b) > 1\), 
        % \Yang{We can not get \(\rho(b^{-1}) < 1\) directly.}

        % \begin{step}\label{step_in_thm_absolute_value_on_finite_extension_of_complete_fields:triangle_inequality}
        %     Show that 
        %     \[ \rho(a + b) \leq \rho(a) + \rho(b) \]
        %     and hence \(\|\cdot\|_{\bfl} = \|N_{\bfl/\kk}(\cdot)\|_{\kk}^{1/n}\) defines an absolute value on \(\bfl\) extending that on \(\kk\).
        % \end{step}


        % \begin{step}\label{step_in_thm_absolute_value_on_finite_extension_of_complete_fields:uniqueness_of_absolute_value}
        %     Show the uniqueness of the absolute value on \(\bfl\) extending that on \(\kk\).
        % \end{step}




        % \Yang{To be added.}
    \end{proof}

    % \begin{remark}\label{rmk:compatiblity_of_extension_and_completion}
    %     \Yang{I want to discuss some compatiblity of extension and completion.}
    % \end{remark}

    \begin{proposition}\label{prop:completion_of_algebraically_closed_valuation_fields_is_algebraically_closed}
        Let \(\kk\) be an algebraically closed non-archimedean field.
        Then its completion \(\widehat{\kk}\) is also algebraically closed.
        % \Yang{To be checked.}
    \end{proposition}
    \begin{proof}
        Let \(f \in \widehat{\kk}[X]\) be a non-constant polynomial.
        We will show that \(f\) has a root in \(\widehat{\kk}\).
        Take a sequence of polynomials \(\{f_n\}_{n \in \bbN}\) in \(\kk[X]\) converging to \(f\) coefficient-wisely and of the same degree \(d\).
        Since \(\kk\) is algebraically closed, each \(f_n\) splits completely in \(\kk\) and hence in \(\widehat{\kk}\).
        Write \(f_n(X) = \prod_{i=1}^d (X - \alpha_{n,i})\) with \(\alpha_{n,i} \in \widehat{\kk}\).

        Let \(\bfl\) be a finite extension of \(\widehat{\kk}\) such that \(f\) has a root \(\alpha\) in \(\bfl\).
        For every \(\varepsilon > 0\), if there are infinitely many \(n\) such that \(\alpha_{n,i} \notin B(\alpha, \varepsilon)\) for all \(1 \leq i \leq d\), 
        then we have \(|f_n(\alpha)| \geq \varepsilon^d\) for infinitely many \(n\), contradicting the fact that \(f_n(\alpha) \to f(\alpha) = 0\).
        Thus for every \(\varepsilon > 0\), there exists \(N > 0\) such that for all \(n \geq N\), there exists \(1 \leq i \leq d\) with \(\alpha_{n,i} \in B(\alpha, \varepsilon)\).
        That is, we can find a sequence \(\alpha_{n,i_n} \in \kk\) converging to \(\alpha\).
        Since \(\widehat{\kk}\) is complete, we have \(\alpha \in \widehat{\kk}\).
        % \Yang{To be added.}
    \end{proof}

      \section{Finite field extensions}

\subsection{Finite-dimensional vector space}

    \begin{definition}\label{def:norm_on_vector_space_over_valuation_field}
        Let \(\kk\) be a valuation field and \(V\) a vector space over \(\kk\).
        A \emph{norm} on \(V\) is a function \(\|\cdot\|: V \to \bbR_{\geq 0}\) satisfying the following properties for all \(x, y \in V\) and \(a \in \kk\):
        \begin{enumerate}
            \item \(\|x\| = 0\) if and only if \(x = 0\);
            \item \(\|a x\| = |a| \cdot \|x\|\);
            \item \(\|x + y\| \leq \|x\| + \|y\|\).
        \end{enumerate}
    \end{definition}

    \begin{example}\label{def:maximal_norm_on_finite-dimensional_vector_space}
        Let \(\kk\) be a valuation field and \(V\) a finite-dimensional vector space over \(\kk\) with basis \(\{e_1, e_2, \ldots, e_n\}\).
        For any \(x = a_1 e_1 + a_2 e_2 + \cdots + a_n e_n \in V\), define
        \[
            \|x\|_{\max} := \max_{1 \leq i \leq n} |a_i|.
        \]
        Then \(\|\cdot\|_{\max}\) is a norm on \(V\), called the \emph{maximal norm} with respect to the basis \(\{e_1, e_2, \ldots, e_n\}\).
    \end{example}

    \begin{example}\label{def:1-norm_no_finite-dimensional_vector_space}
        Setting as in \cref{def:maximal_norm_on_finite-dimensional_vector_space}, for any \(x = a_1 e_1 + a_2 e_2 + \cdots + a_n e_n \in V\), define
        \[
            \|x\|_1 := |a_1| + |a_2| + \cdots + |a_n|.
        \]
        Then \(\|\cdot\|_1\) is also a norm on \(V\).
    \end{example}

    \begin{definition}\label{def:equivalent_norm_on_vector_space}
        Let \(\kk\) be a valuation field and \(V\) a vector space over \(\kk\).
        Two norms \(\|\cdot\|_1\) and \(\|\cdot\|_2\) on \(V\) are said to be \emph{equivalent} if there exist positive constants \(C_1, C_2 > 0\) such that for all \(x \in V\),
        \[
            C_1 \|x\|_1 \leq \|x\|_2 \leq C_2 \|x\|_1.
        \]
    \end{definition}

    \begin{lemma}\label{prop:norms_are_equivalent_iff_induces_the_same_topology}
        Let \(\kk\) be a valuation field and \(V\) a vector space over \(\kk\).
        Two norms \(\|\cdot\|_1\) and \(\|\cdot\|_2\) on \(V\) are equivalent if and only if they induce the same topology on \(V\).
    \end{lemma}
    \begin{proof}
        The sufficiency is clear.
        Now suppose that \(\|\cdot\|_1\) and \(\|\cdot\|_2\) induce the same topology on \(V\).
        Hence the unit open ball with respect to \(\|\cdot\|_1\) contains a unit open ball with respect to \(\|\cdot\|_2\).
        That is, 
        \[ \{x \in V : \|x\|_1 < 1\} \supseteq \{x \in V : \|x\|_2 < C\}. \]
        Then for every \(x \in V\) with \(\|x\|_1 = 1\), we have \(\|x\|_2 \geq C = C \|x\|_1\).
        By scaling, we get that for every \(x \in V\),
        \[ \|x\|_2 \geq C \|x\|_1. \]
        Similar for the other direction, we conclude that \(\|\cdot\|_1\) and \(\|\cdot\|_2\) are equivalent.
    \end{proof}

    \begin{proposition}\label{prop:finite_dimensional_vector_space_over_complete_fields_is_complete}
        Let \(V\) be a normed finite-dimensional vector space over a complete valuation field \(\kk\).
        Then \(V\) is complete.
    \end{proposition}
    \begin{proof}
        \Yang{To be added.}
    \end{proof}

    \begin{theorem}\label{prop:norm_on_finite_dimensional_vector_space_are_equivalent}
        Let \(V\) be a finite-dimensional vector space over a complete field \(\kk\).
        Then all norms on \(V\) are equivalent.
    \end{theorem}
    \begin{proof}
        Fix a basis \(\{e_1, e_2, \ldots, e_n\}\) of \(V\) and let \(\|\cdot\|_{\max}\) be the maximal norm with respect to this basis as in \cref{def:maximal_norm_on_finite-dimensional_vector_space}.
        Let \(\|\cdot\|\) be any norm on \(V\).
        It suffices to show that \(\|\cdot\|\) and \(\|\cdot\|_{\max}\) are equivalent.
        First we have 
        \[ \|y\| \leq \sum_{i=1}^n |a_i| \|e_i\| \leq \left(\sum_{i=1}^n \|e_i\|\right) \|y\|_{\max} \]
        for any \(y = a_1 e_1 + a_2 e_2 + \cdots + a_n e_n \in V\).
        It remains to show that there exists a constant \(C > 0\) such that for any \(y \in V\),
        \[ \|y\|_{\max} \leq C \|y\|. \]
        \Yang{To be added.}
    \end{proof}

    \begin{remark}\label{rmk:finite-dimensional_vector_space_over_non_complete_fields}
        If the base field \(\kk\) is not complete, then \cref{prop:norm_on_finite_dimensional_vector_space_are_equivalent} may fail.
        For example, let \(\kk = \bbQ\) with the usual absolute value, and let \(V = \bbQ[\alpha]\) with \(\alpha^2-\alpha-1=0\).
        There are two embeddings of \(V\) into \(\bbR\):
        \[ \iota_1: a + b\alpha \mapsto a + b\frac{1+\sqrt{5}}{2}, \quad \iota_2: a + b\alpha \mapsto a + b\frac{1-\sqrt{5}}{2}. \]
        Define two norms on \(V\) by
        \[
            \|x\|_1 := |\iota_1(x)|, \quad \|x\|_2 := |\iota_2(x)|,
        \]
        where \(|\cdot|\) is the usual absolute value on \(\bbR\).
        Then \(\|\cdot\|_1\) and \(\|\cdot\|_2\) are not equivalent since \(\iota_2(\alpha^n) \to 0\) as \(n \to \infty\) while \(\iota_1(\alpha^n) \to \infty\).
    \end{remark}

    The following lemma is a classical result in functional analysis, which will be used in the next subsection.

    \begin{lemma}\label{prop:operator_norm_on_M_n_k_with_k_complete}
        Let \(\kk\) be a complete field and \(V\) a normed finite-dimensional vector space over \(\kk\).
        Then 
        \[ \|\cdot\| : \End_{\kk}(V) \to \bbR_{\geq 0}, \quad T \mapsto \sup_{x \in V \setminus \{0\}} \frac{\|T(x)\|}{\|x\|} \]
        defines a norm on the \(\kk\)-vector space \(\End_{\kk}(V)\) satisfying
        \[ \|AB\| \leq \|A\| \cdot \|B\|, \quad \forall A, B \in \End_{\kk}(V). \]
    \end{lemma}
    \begin{proof}
        First we show the existence of the supremum, i.e., there exists \(C > 0\) such that for all \(x \in V \setminus \{0\}\), \(\|T(x)\| \leq C \|x\|\).
        Fix a basis \(\{e_1, e_2, \ldots, e_n\}\) of \(V\) and let \(\|\cdot\|_{\max}\) be the maximal norm with respect to this basis.
        Since all norms on \(V\) are bounded by each other by \cref{prop:norm_on_finite_dimensional_vector_space_are_equivalent}, we only need to show that there exists \(C > 0\) such that for all \(x \in V \setminus \{0\}\), \(\|T(x)\|_{1} \leq C \|x\|_{\max}\).
        Write \(T(e_i) = \sum_{j=1}^n a_{ij} e_j\) for \(1 \leq i \leq n\).
        For any \(x = \sum_{i=1}^n x_i e_i \in V\), we have
        \[ \|T(x)\|_1 = \left\|\sum_{j=1}^n \left(\sum_{i=1}^n a_{ij} x_i\right) e_j \right\|_1 = \sum_{j=1}^n \left|\sum_{i=1}^n a_{ij} x_i\right| \leq \left(\sum_{1 \leq i, j \leq n} |a_{ij}|\right) \|x\|_{\max}. \]
        Thus the supremum is finite.

        The linearity and positive-definiteness of \(\|\cdot\|\) are clear.
        It remains to show the triangle inequality and sub-multiplicativity.
        For any \(A, B \in \End_{\kk}(V)\), we have
        \[ \frac{\|(A + B)(x)\|}{\|x\|} = \frac{\|A(x)\|}{\|x\|} + \frac{\|B(x)\|}{\|x\|} \leq \|A\| + \|B\|. \]
        Taking supremum over all \(x \in V \setminus \{0\}\) gives \(\|A + B\| \leq \|A\| + \|B\|\).
        We have 
        \[ \|AB(x)\| \leq \|A\| \cdot \|B(x)\| \leq \|A\| \cdot \|B\| \cdot \|x\| \]
        and hence \(\|AB(x)\|/\|x\| \leq \|A\| \cdot \|B\|\).
        Taking supremum we get \(\|AB\| \leq \|A\| \cdot \|B\|\). 
    \end{proof}

\subsection{Finite field extensions}

    \begin{lemma}\label{prop:existence_of_absolute_value_on_finite_extension_of_complete_fields}
        Let \(\kk\) be a complete field and \(\bfl\) a finite extension of \(\kk\).
        Then there exists an absolute value on \(\bfl\) extending the absolute value on \(\kk\).
    \end{lemma}
    \begin{proof}
        Fix a norm \(\|\cdot\|_V\) on the \(\kk\)-vector space \(V = \bfl\).
        The norm \(\|\cdot\|_V\) induces an operator norm \(\|\cdot\|_{\op}\) on the \(\kk\)-vector space \(\End_{\kk}(V)\) as in \cref{prop:operator_norm_on_M_n_k_with_k_complete}.
        For any \(a \in \bfl\), let \(\mu_a \in \End_{\kk}(V)\) be the \(\kk\)-linear map defined by multiplication by \(a\).
        Note that \(a \mapsto \mu_a\) gives a embedding of \(\kk\)-algebras and if \(a \in \kk\), \(\|\mu_a\|_{\op} = \|a\|_{\kk}\).
        Thus the restriction of \(\|\cdot\|_{\op}\) to \(\bfl\) gives an norm on \(\bfl\) extending that on \(\kk\).
        The normed ring \((\bfl, \|\cdot\|_{\op})\) is a Banach ring since it is a finite-dimensional vector space over the complete field \(\kk\).
        By \cref{thm:norm_spectrum_of_Banach_rings_is_nonempty}, there exists a multiplicative seminorm \(\|\cdot\|_{\bfl}\) on \(\bfl\) bounded by \(\|\cdot\|_{\op}\).
        In particular, \(\|\cdot\|_{\bfl}\) is bounded by \(\|\cdot\|_{\kk}\) on \(\kk\).
        On a field, if one norm is bounded by another norm, then they must be equal (consider the inverse elements).
        Thus \(\|\cdot\|_{\bfl}\) extends the absolute value on \(\kk\).
    \end{proof}

    \begin{theorem}\label{prop:absolute_value_on_finite_extension_of_complete_fields}
        Let \(\kk\) be a complete field and \(\bfl\) a finite extension of \(\kk\).
        Then the absolute value on \(\bfl\) which extends the absolute value on \(\kk\) is uniquely determined by the absolute value on \(\kk\).
        Furthermore, we have 
        \[ \|\cdot\|_{\bfl} = \|N_{\bfl/\kk}(\cdot)\|_{\kk}^{1/n}, \]
        where \(n = [\bfl : \kk]\) and \(N_{\bfl/\kk}\) is the norm map from \(\bfl\) to \(\kk\).
        % \Yang{To be checked.}
    \end{theorem}
    \begin{proof}
        Let \(\|\cdot\|_{\bfl}\) be arbitrary absolute value on \(\bfl\) extending that on \(\kk\).
        We will show that \(\|\cdot\|_{\bfl}\) must be equal to \(\|N_{\bfl/\kk}(\cdot)\|_{\kk}^{1/n}\).
        For any \(a \in \bfl\), set \(b = a^n/N_{\bfl/\kk}(a) \in \bfl\).
        Then \(N_{\bfl/\kk}(b) = 1\) and 
        \[ \|b\|_{\bfl} = \frac{\|a\|_{\bfl}^n}{\|N_{\bfl/\kk}(a)\|_{\kk}}. \]
        Thus it suffices to show that \(\|b\|_{\bfl} = 1\) whenever \(N_{\bfl/\kk}(b) = 1\).

        Note that the norm map \(N_{\bfl/\kk}: \bfl \to \kk\) is the determinant of the \(\kk\)-linear map \(\mu_b \in \End_{\kk}(V)\) defined by multiplication by \(b\).
        Hence it is continuous on \(\bfl\) (since it is a polynomial in the entries of the matrix representation).
        If \(\|b\|_{\bfl} < 1\), then \(\|b^m\|_{\bfl} \to 0\) as \(m \to \infty\).
        Thus \(N_{\bfl/\kk}(b^m) = \det(\mu_{b^m}) \to 0\) as \(m \to \infty\), contradicting the fact that \(N_{\bfl/\kk}(b^m) = 1\) for all \(m\).
        Similarly, if \(\|b\|_{\bfl} > 1\), then just consider \(b^{-1}\).

        % Fix a norm \(\|\cdot\|_V\) on the \(\kk\)-vector space \(V = \bfl\).
        % The norm \(\|\cdot\|_V\) induces an operator norm \(\|\cdot\|_{\op}\) on the \(\kk\)-vector space \(\End_{\kk}(V)\) as in \cref{prop:operator_norm_on_M_n_k_with_k_complete}.
        % For any \(a \in \bfl\), let \(\mu_a \in \End_{\kk}(V)\) be the \(\kk\)-linear map defined by multiplication by \(a\).
        % Note that \(a \mapsto \mu_a\) is a \(\kk\)-algebra homomorphism and if \(a \in \kk\), \(\|\mu_a\|_{\op} = \|a\|_{\kk}\).

        % \begin{step}\label{step_in_thm_absolute_value_on_finite_extension_of_complete_fields:the_field_norm_is_the_spectral_radius}
        %     Show that 
        %     \[ \|N_{\bfl/\kk}(a)\|_{\kk} = \lim_{m \to \infty} \sqrt[m]{\|\mu_{a^{mn}}\|_{\op}} \eqqcolon \rho(a^n). \]
        % \end{step}

        % The existence of the limit \(\rho(a) = \lim_{m \to \infty} \sqrt[m]{\|\mu_{a^m}\|_{\op}}\) follows from the sub-multiplicativity of the operator norm.
        % And note that \(\rho(a^n) = \rho(a)^n\) for all \(a \in \bfl\) and \(n \in \bbN\).
        % We can assume that \(a \neq 0\).
        % Let \(b = a^n/N_{\bfl/\kk}(a) \in \bfl\).
        % Then \(N_{\bfl/\kk}(b) = 1\) and 
        % \[ \rho(b) = \lim_{m \to \infty} \sqrt[m]{\|\mu_{b^m}\|_{\op}} = \lim_{m \to \infty} \sqrt[m]{\frac{\|\mu_{a^{mn}}\|_{\op}}{\|N_{\bfl/\kk}(a)^m\|_{\kk}}} = \frac{\rho(a^n)}{\|N_{\bfl/\kk}(a)\|_{\kk}} \]
        % since \(N_{\bfl/\kk}\) is multiplicative and \(N_{\bfl/\kk}(a) \in \kk\).
        % Thus it suffices to show that \(\rho(b) = 1\). 

        % Suppose that \(\rho(b) < 1\).
        % Then there exists \(\delta \in (0,1)\) and \(M > 0\) such that for all \(m \geq M\), \(\|\mu_{b^m}\|_{\op} < \delta^m\).
        % In particular, we have \(\mu_{b^m} \to 0\) as \(m \to \infty\) in the normed vector space \(\End_{\kk}(V)\).
        % However, note that \(\det\) is continuous on \(\End_{\kk}(V)\) (since it is a polynomial in the entries of the matrix representation).
        % Thus we have \(N_{\bfl/\kk}(b^m) = \det(\mu_{b^m}) \to 0\) as \(m \to \infty\), contradicting the fact that \(N_{\bfl/\kk}(b^m) = 1\) for all \(m\).
        
        % if \(\rho(b) > 1\), 
        % \Yang{We can not get \(\rho(b^{-1}) < 1\) directly.}

        % \begin{step}\label{step_in_thm_absolute_value_on_finite_extension_of_complete_fields:triangle_inequality}
        %     Show that 
        %     \[ \rho(a + b) \leq \rho(a) + \rho(b) \]
        %     and hence \(\|\cdot\|_{\bfl} = \|N_{\bfl/\kk}(\cdot)\|_{\kk}^{1/n}\) defines an absolute value on \(\bfl\) extending that on \(\kk\).
        % \end{step}


        % \begin{step}\label{step_in_thm_absolute_value_on_finite_extension_of_complete_fields:uniqueness_of_absolute_value}
        %     Show the uniqueness of the absolute value on \(\bfl\) extending that on \(\kk\).
        % \end{step}




        % \Yang{To be added.}
    \end{proof}

    % \begin{remark}\label{rmk:compatiblity_of_extension_and_completion}
    %     \Yang{I want to discuss some compatiblity of extension and completion.}
    % \end{remark}

    \begin{proposition}\label{prop:completion_of_algebraically_closed_valuation_fields_is_algebraically_closed}
        Let \(\kk\) be an algebraically closed non-archimedean field.
        Then its completion \(\widehat{\kk}\) is also algebraically closed.
        % \Yang{To be checked.}
    \end{proposition}
    \begin{proof}
        Let \(f \in \widehat{\kk}[X]\) be a non-constant polynomial.
        We will show that \(f\) has a root in \(\widehat{\kk}\).
        Take a sequence of polynomials \(\{f_n\}_{n \in \bbN}\) in \(\kk[X]\) converging to \(f\) coefficient-wisely and of the same degree \(d\).
        Since \(\kk\) is algebraically closed, each \(f_n\) splits completely in \(\kk\) and hence in \(\widehat{\kk}\).
        Write \(f_n(X) = \prod_{i=1}^d (X - \alpha_{n,i})\) with \(\alpha_{n,i} \in \widehat{\kk}\).

        Let \(\bfl\) be a finite extension of \(\widehat{\kk}\) such that \(f\) has a root \(\alpha\) in \(\bfl\).
        For every \(\varepsilon > 0\), if there are infinitely many \(n\) such that \(\alpha_{n,i} \notin B(\alpha, \varepsilon)\) for all \(1 \leq i \leq d\), 
        then we have \(|f_n(\alpha)| \geq \varepsilon^d\) for infinitely many \(n\), contradicting the fact that \(f_n(\alpha) \to f(\alpha) = 0\).
        Thus for every \(\varepsilon > 0\), there exists \(N > 0\) such that for all \(n \geq N\), there exists \(1 \leq i \leq d\) with \(\alpha_{n,i} \in B(\alpha, \varepsilon)\).
        That is, we can find a sequence \(\alpha_{n,i_n} \in \kk\) converging to \(\alpha\).
        Since \(\widehat{\kk}\) is complete, we have \(\alpha \in \widehat{\kk}\).
        % \Yang{To be added.}
    \end{proof}


   \chapter{Analytic spaces}

   % \appendix



   \printbibliography[heading=bibintoc, title={References}] % 打印参考文献

\end{document}
