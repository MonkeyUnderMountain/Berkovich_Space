\documentclass[sectionlevel=chapter]{noteformyself}
\usepackage{draftwatermark} % 水印宏包

\input{../../Accessories/notation.tex}
\addbibresource{../../Accessories/ref.bib}
\newcommand{\Yang}[1]{\textcolor{red}{Yang: #1}}


\title{Affinoid spaces}
\author{Tianle Yang}
\date{\today}
\authorpage{\href{www.example.com}{My Homepage}}

% available on the mode "\sectionlevel=chapter" or "\sectionlevel=book".
\setCJKfamilyfont{lxgwwenkai}{LXGW WenKai} % 定义霞鹜文楷,若未安装,请去掉相关代码编译或使用其他字体
\coversentence{\CJKfamily{lxgwwenkai}阿巴阿巴!}
\coverimage{} % 封面图片
\covertitlefont{Allura} % 封面标题字体, 若未安装,请去掉相关代码编译或使用其他字体
% \coverlinecolor{red} % 封面线条颜色
% \covertextcolor{green} % 封面文字颜色


\begin{document}

    \maketitle

    \tableofcontents

    Affinoid spaces are the local models for rigid analytic spaces, similar to how affine schemes serve as the local models for schemes in algebraic geometry. 
    They are defined over non-archimedean fields and provide a framework for studying analytic geometry in this context.

    \printbibliography[heading=bibintoc, title={References}]
    
\end{document}
