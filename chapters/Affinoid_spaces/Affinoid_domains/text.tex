\section{Affinoid domains}


    \begin{definition}\label{def:affinoid_domains}
        Let \(A\) be a \(\kk\)-affinoid algebra, and let \(X = \scrM(A)\) be the associated affinoid space.
        A closed subset \(V \subseteq X\) is called an \emph{affinoid domain} if there exists a \(\kk\)-affinoid algebra \(A_V\) and a morphism of \(\kk\)-affinoid algebras \(\varphi: A \to A_V\) satisfying the following universal property:
        for every bounded homomorphism of \(\kk\)-affinoid algebras \(\psi: A \to B\) such that the induced map on spectra \(\scrM(\psi): \scrM(B) \to X\) has its image contained in \(V\), there exists a unique bounded homomorphism \(\theta: A_V \to B\) such that the following diagram commutes:
        \[
        \begin{tikzcd}
            & A_V  \arrow[dr, gray, "\theta"] & \\
            A \arrow[ur, "\varphi"] \arrow[rr, "\psi"'] & & B
        \end{tikzcd}
        \]
        In this case, we say that \(V\) is represented by the affinoid algebra \(A_V\).
    \end{definition}
    \begin{slogan}
        A closed subset \(V \subset X\) is an affinoid domain if the functor ``\(\Mor(-, V)\)'' is representable.
    \end{slogan}

    \Yang{Why we consider closed subset rather that open subset?}


    \begin{construction}\label{cons:weierstrass_domain}
        Let \(f=(f_1,\ldots,f_n),g = (g_1,\ldots,g_m)\) be two tuples of elements in \(A\).
        Set \(p=(p_1,\ldots,p_n)\) and \(q=(q_1,\ldots,q_m)\) be two tuples of positive real numbers.
        We define the following closed subsets of \(X\):
        \[ X\left(\underline{f/p},\underline{q/g}\right) := \left\{ x \in X \colon |f_i(x)| \le p_i, |g_j(x)| \ge q_j, 1 \le i \le n, 1 \le j \le m \right\}. \]
        Such a closed subset is called a \emph{Weierstrass domain} of \(X\).
        Moreover, we can define a \(\kk\)-affinoid algebra
        \[ A\left\langle \underline{f/p},\underline{q/g} \right\rangle := A\left\langle \frac{f_1}{p_1},\ldots,\frac{f_n}{p_n}, \frac{q_1}{g_1},\ldots,\frac{q_m}{g_m} \right\rangle, \]
        which is the quotient of the Tate algebra
        \[ A\left\langle T_1,\ldots,T_n,S_1,\ldots,S_m \right\rangle \]
        by the ideal generated by the elements \(p_i T_i - f_i\) for \(1 \le i \le n\) and \(g_j S_j - q_j\) for \(1 \le j \le m\).
        There is a natural bounded homomorphism \(\varphi: A \to A\langle \underline{f/p},\underline{q/g} \rangle\) induced by the inclusion.
        It can be shown that the closed subset \(X(\underline{f/p},\underline{q/g})\) is an affinoid domain represented by the affinoid algebra \(A\langle \underline{f/p},\underline{q/g} \rangle\).
        \Yang{To be checked}
    \end{construction}

    \begin{construction}\label{cons:rational_domain}
        Let \(f=(f_1,\ldots,f_n),g\) be elements in \(A\) such that the ideal generated by them is the whole algebra \(A\).
        Set \(p=(p_1,\ldots,p_n)\) be a tuple of positive real numbers.
        We define the following closed subset of \(X\):
        \[ X\left(\underline{f/p},g\right) := \left\{ x \in X \colon |f_i(x)| \le p_i |g(x)|, 1 \le i \le n \right\}. \]
        Such a closed subset is called a \emph{rational domain} of \(X\).
        Moreover, we can define a \(\kk\)-affinoid algebra
        \[ A\left\langle \underline{f/p},g^{-1} \right\rangle := A\left\langle \frac{f_1}{p_1 g},\ldots,\frac{f_n}{p_n g} \right\rangle, \]
        which is the quotient of the Tate algebra
        \[ A\left\langle T_1,\ldots,T_n \right\rangle \]
        by the ideal generated by the elements \(p_i g T_i - f_i\) for \(1 \le i \le n\).
        There is a natural bounded homomorphism \(\varphi: A \to A\langle \underline{f/p},g^{-1} \rangle\) induced by the inclusion.
        It can be shown that the closed subset \(X(\underline{f/p},g)\) is an affinoid domain represented by the affinoid algebra \(A\langle \underline{f/p},g^{-1} \rangle\).
        \Yang{To be checked}
    \end{construction}

    \begin{proposition}\label{prop:affinoid_domain_is_flat_over_base}
        Let \(A\) be a \(\kk\)-affinoid algebra, and let \(X = \scrM(A)\) be the associated affinoid space.
        Let \(V \subseteq X\) be an affinoid domain represented by the \(\kk\)-affinoid algebra \(A_V\).
        Then the natural bounded homomorphism \(\varphi: A \to A_V\) is flat.

        We have \(\scrM(A_V) \cong V\).
    \end{proposition}