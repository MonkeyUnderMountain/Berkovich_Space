\section{Analytic functions}

\subsection{Failure of continuous and differentiable functions}

    \begin{definition}\label{def:differentiable_functions}
        Let \((\kk,\|\cdot\|)\) be a non-archimedean field and \(U \subset \kk\) be an open subset.
        A function \(f:U \to \kk\) is said to be \emph{differentiable} at a point \(a \in U\) if the limit
        \[
            f'(a) := \lim_{x \to a} \frac{f(x)-f(a)}{x-a}
        \]
        exists in \(\kk\).
        If \(f\) is differentiable at every point in \(U\), we say that \(f\) is differentiable on \(U\).
        \Yang{to be revised.}
    \end{definition}

    \begin{proposition}\label{prop:failure_of_mean_valuable_theorem}
        Let \((\kk,\|\cdot\|)\) be a non-archimedean field.
        Then there exists a continuous function \(f:\kk \to \kk\) such that for any \(x,y \in \kk\) with \(x \neq y\), we have
        \[
            \frac{f(x)-f(y)}{x-y} = 0.
        \]
    \end{proposition}

\subsection{Power series}

    \begin{lemma}\label{prop:convergence_of_series}
        Let \((\kk,\|\cdot\|)\) be a non-archimedean field and \(\sum_{n=0}^{+\infty} a_n\) be a series in \(\kk\).
        Then the series \(\sum_{n=0}^{+\infty} a_n\) converges if and only if \(\lim_{n \to +\infty} a_n = 0\).
        \Yang{To be checked.}
    \end{lemma}

    \begin{proposition}\label{prop:convergent_radius_of_power_series}
        Let \((\kk,\|\cdot\|)\) be a non-archimedean field and \(\sum_{n=0}^{+\infty} a_n (x - c)^n\) be a power series in \(\kk\).
        Then there exists a radius of convergence \(R \in \bbR_{\geq 0} \cup \{+\infty\}\) such that the series converges for all \(x \in \kk\) with \(\|x - c\| < R\) and diverges for all \(x \in \kk\) with \(\|x - c\| > R\).
        \Yang{To be revised.}
    \end{proposition}

    \begin{proposition}\label{prop:convergent_series_over_non_archimedean_fields}
        Let \((\kk,\|\cdot\|)\) be a complete non-archimedean field and \(\sum_{n=0}^{+\infty} a_n\) be a series in \(\kk\).
        Then the series \(\sum_{n=0}^{+\infty} a_n\) converges if and only if \(\lim_{n \to +\infty} a_n = 0\).
        \Yang{To be checked.}
    \end{proposition}

    \begin{definition}\label{def:Tate_algebra}
        Let \((\kk,\|\cdot\|)\) be a complete non-archimedean field.
    \end{definition}

    \begin{proposition}\label{prop:norm_of_Tate_algebra_coincides_with_the_suprum_norm}
        Let \((\kk,\|\cdot\|)\) be a complete non-archimedean field.
        Then the norm on the Tate algebra \(\kk\langle x_1, \ldots, x_n \rangle\) coincides with the supremum norm on the closed unit polydisc in \(\kk^n\).
        \Yang{To be checked.}
        
    \end{proposition}

\subsection{Analytic functions and maps}

    As in the case of real analysis, we can define analytic functions over non-archimedean fields using power series.

    \begin{definition}\label{def:analytic_functions}
        Let \((\kk,\|\cdot\|)\) be a complete non-archimedean field and \(U \subset \kk\) be an open subset.
        A function \(f:U \to \kk\) is said to be \emph{analytic} at a point \(c \in U\) if there exists a power series \(\sum_{n=0}^{+\infty} a_n (x - c)^n\) that converges to \(f(x)\) for all \(x\) in some neighborhood of \(c\).
        If \(f\) is analytic at every point in \(U\), we say that \(f\) is analytic on \(U\).
        \Yang{to be revised.}
    \end{definition}

    \begin{theorem}\label{prop:analytic_function_is_lipschitz}
        Let \((\kk,\|\cdot\|)\) be a complete non-archimedean field and \(U \subset \kk\) be an open subset.
        If \(f:U \to \kk\) is an analytic function, then \(f\) is locally Lipschitz continuous on \(U\).
        \Yang{To be checked.}
    \end{theorem}

    \begin{theorem}\label{prop:maximal_module_principle}
        Let \((\kk,\|\cdot\|)\) be a complete non-archimedean field and \(U \subset \kk\) be an open subset.
        If \(f:U \to \kk\) is an analytic function, then \(f\) satisfies the maximum modulus principle, i.e., if there exists a point \(x_0 \in U\) such that \(\|f(x_0)\| \geq \|f(x)\|\) for all \(x \in U\), then \(f\) is constant on \(U\).
        \Yang{To be checked.}
    \end{theorem}