\section{Analytic functions}

% \subsection{Strictly differentiable functions}

%     Recalling the definition of differentiable functions over valuation fields.

%     \begin{definition}\label{def:differentiable_functions}
%         Let \(\kk\) be a valuation field and \(U \subset \kk\) be an open subset.
%         A function \(f:U \to \kk\) is said to be \emph{differentiable} at a point \(a \in U\) if the limit
%         \[
%             f'(a) := \lim_{x \to a} \frac{f(x)-f(a)}{x-a}
%         \]
%         exists in \(\kk\).
%         If \(f\) is differentiable at every point in \(U\), we say that \(f\) is differentiable on \(U\).
%     \end{definition}

%     Unlike the case of real or complex analysis, the differentiable functions over non-archimedean fields may behave very differently.
%     There exists differentiable functions with zero derivative that are not locally constant.

%     \begin{proposition}\label{prop:existence_of_almost_constant_functions}
%         Let \(\kk\) be a non-archimedean field.
%         Then there exists a function \(f:\kk \to \kk\) that is differentiable everywhere with \(f'(x) = 0\) for all \(x \in \kk\), but \(f\) is not locally constant.
%         % \Yang{To be checked.}
%     \end{proposition}
%     \begin{proof}
%         Fix \(r \in (0,1)\).
%         Consider a descending sequence of open ball \(\{B(0,r^n)\}\) and \(a_n \in \kk\) with \(\|a_n\| = r^{2n}\).
%         Define 
%         \[ f:\kk \to \kk, \quad x \mapsto \begin{cases}
%             a_n, & x \in B(0,r^n) \setminus B(0,r^{n+1}) \\
%             0, & x = 0
%         \end{cases} \]
%         Then 
%         \[ f'(0) = \lim_{x \to 0} \frac{f(x)-f(0)}{x-0} = \lim_{n \to \infty} \frac{a_n - 0}{x_n - 0} \]
%         for any sequence \(x_n \to 0\) with \(x_n \in B(0,r^n) \setminus B(0,r^{n+1})\).
%         Since \(\|x_n\| \geq r^{n+1}\), we have
%         \[ \left\| \frac{a_n}{x_n} \right\| \leq \frac{r^{2n}}{r^{n+1}} = r^{n-1} \to 0 \]
%         as \(n \to \infty\).
%         Thus \(f'(0) = 0\) and then \(f'(x) = 0\) for all \(x \in \kk\).
%         However, \(f\) is not locally constant near \(0\).
%     \end{proof}

%     \begin{definition}\label{def:strictly_differentiable_function}
%         Let \(\kk\) be a valuation field and \(U \subset \kk\) be an open subset.
%         A function \(f:U \to \kk\) is said to be \emph{strictly differentiable} at a point \(a \in U\) if the limit
%         \[
%             f'(a) := \lim_{\substack{(x,y) \to (a,a) \\ x \neq y}} \frac{f(x)-f(y)}{x-y}
%         \]
%         exists in \(\kk\).
%         If \(f\) is strictly differentiable at every point in \(U\), we say that \(f\) is strictly differentiable on \(U\).
%     \end{definition}

%     \begin{remark}\label{rmk:strictly_differentiable_function_on_archimedean_fields}
%         If \(\kk\) is a complete archimedean field (i.e., \(\bbR\) or \(\bbC\)), then a function \(f:U \to \kk\) is strictly differentiable at a point \(a \in U\) if and only if \(f\) is differentiable at \(a\) and the derivative \(f'\) is continuous at \(a\).
%         \Yang{To be revised.}
%     \end{remark}

%     \begin{proposition}\label{prop:strictly_differentiable_functions_are_locally_isometric}
%         Let \(\kk\) be a non-archimedean complete field and \(U \subset \kk\) be an open subset.
%         Suppose that \(f:U \to \kk\) is strictly differentiable and \(f'(a) \neq 0\) for some \(a \in U\).
%         There exists an open neighborhood \(V \subset U\) of \(a\) such that \(x \mapsto f(x)/f'(a)\) is an isometry on \(V\).
%         % \Yang{To be checked.}
%     \end{proposition}
%     \begin{proof}
%         \Yang{To be added.}
%     \end{proof}


%     \begin{example}\label{eg:existence_of_almost_constant_functions_on_non-archimedean_fields}
%         Let \(\kk\) be a non-archimedean field with non-trivial valuation.
%         Then there exists a function \(f:\kk \to \kk\) that is differentiable everywhere with \(f'(x) = 0\) for all \(x \in \kk\), but \(f\) is not locally constant.
%         % \Yang{To be checked.}

%         Fix \(r \in (0,1)\).
%         Consider a descending sequence of open ball \(\{B(0,r^n)\}\) and \(a_n \in \kk\) with \(\|a_n\| = r^{2n}\).
%         Define 
%         \[ f:\kk \to \kk, \quad x \mapsto \begin{cases}
%             a_n, & x \in B(0,r^n) \setminus B(0,r^{n+1}) \\
%             0, & x = 0
%         \end{cases} \]
%         Then 
%         \[ f'(0) = \lim_{x \to 0} \frac{f(x)-f(0)}{x-0} = \lim_{n \to \infty} \frac{a_n - 0}{x_n - 0} \]
%         for any sequence \(x_n \to 0\) with \(x_n \in B(0,r^n) \setminus B(0,r^{n+1})\).
%         Since \(\|x_n\| \geq r^{n+1}\), we have
%         \[ \left\| \frac{a_n}{x_n} \right\| \leq \frac{r^{2n}}{r^{n+1}} = r^{n-1} \to 0 \]
%         as \(n \to \infty\).
%         Thus \(f'(0) = 0\) and then \(f'(x) = 0\) for all \(x \in \kk\).
%         However, \(f\) is not locally constant near \(0\).
%     \end{example}

% \subsection{Restricted power series}

\subsection{Tate algebras}

    \begin{notation}\label{notation:mult-label_for_Tate_algebra}
        Let \(T = (T_1, \ldots, T_n)\) be a tuple of \(n\) indeterminates, \(r = (r_1, \ldots, r_n)\) be a tuple of \(n\) positive real numbers, and \(\alpha = (\alpha_1, \ldots, \alpha_n) \in \bbN^n\) be a multi-index.
        We use the following notations:
        \begin{itemize}
            \item \(T^\alpha := T_1^{\alpha_1} T_2^{\alpha_2} \cdots T_n^{\alpha_n}\) and \(r^\alpha := r_1^{\alpha_1} r_2^{\alpha_2} \cdots r_n^{\alpha_n}\);
            \item \(\underline{T/r} := (T_1/r_1, T_2/r_2, \ldots, T_n/r_n)\);
            \item \(|\alpha| := \alpha_1 + \alpha_2 + \cdots + \alpha_n\);
            \item \(E(x,\underline{r}) = \{y \in \kk^n \mid \|y_i - x_i\| \leq r_i, i = 1, \ldots, n\}\) and \(B(x,\underline{r}) = \{y \in \kk^n \mid \|y_i - x_i\| < r_i, i = 1, \ldots, n\}\) for \(x = (x_1, \ldots, x_n) \in \kk^n\).
            \item Let \(\{x_{\alpha}\}_{\alpha \in \bbN^n}\) be a set of elements in a metric space \(X\) indexed by multi-indices \(\alpha \in \bbN^n\).
                We say that \(\lim_{|\alpha| \to +\infty} x_\alpha = x \in X\) if for every \(\varepsilon > 0\), there exists \(N \in \bbN\) such that for all \(\alpha \in \bbN^n\) with \(|\alpha| > N\), we have \(d(x_\alpha, x) < \varepsilon\).
        \end{itemize}
    \end{notation}


    % \begin{proposition}\label{prop:convergent_radius_of_power_series}
    %     Let \(\kk\) be a complete non-archimedean field and \(f = \sum_{\alpha \in \bbN^n} a_\alpha T^\alpha \in \kk[[T_1, \ldots, T_n]]\).
    %     \Yang{To be continued.}
    % \end{proposition}

    % \begin{proposition}
    %     % \label{prop:convergent_radius_of_power_series}
    %     Let \(\kk\) be a complete non-archimedean field and \(f = \sum_{n=0}^{+\infty} a_n T^n \in \kk[[T]]\).
    %     Set 
    %     \[
    %         R := \frac{1}{\limsup_{n \to +\infty} \|a_n\|^{1/n}} \in \bbR_{\geq 0} \cup \{+\infty\}.
    %     \]
    %     Then we have 
    %     \begin{enumerate}
    %         \item if \(R = 0\), then the series \(f(x)\) converges only at \(x = 0\);
    %         \item if \(R = +\infty\), then the series \(f(x)\) converges for all \(x \in \kk\);
    %         \item if \(0 < R < +\infty\), then the series \(f(x)\) converges for all \(x \in \kk\) with \(\|x\| < R\) and diverges for all \(x \in \kk\) with \(\|x\| > R\).
    %     \end{enumerate}
    %     Suppose that \(0 < R < +\infty\).
    %     Then the series \(f(x)\) converges for all \(x \in \kk\) with \(\|x\| = R\) if and only if \(\lim_{n \to +\infty} \|a_n\| R^n = 0\).
    % \end{proposition}
    % \begin{proof}
    %     By \cref{prop:convergence_in_ultra-metric_space}, we only need to check when the terms \(a_n x^n\) tend to zero as \(n \to +\infty\).
    %     If \(\|x\| < R\), there exists \(r \in (0,1)\) such that \(\|x\| < r^2R\).
    %     Then there exists \(N \in \bbN\) such that for all \(n \geq N\), we have \(\|a_n\|^{1/n} < 1/(rR)\) and thus
    %     \[
    %         \|a_n x^n\| = \|a_n\| \|x\|^n < \|a_n\| (r^2R)^n < (r^2R)^n \cdot \frac{1}{(rR)^n} = r^n \to 0.
    %     \]
    %     Thus the series \(f(x)\) converges for all \(x \in \kk\) with \(\|x\| < R\).

    %     Suppose that \(\|x\| > R\).
    %     There exists \(s > 1\) such that \(\|x\| > R/s\).
    %     By the definition of \(R\), there exist infinitely many \(n \in \bbN\) such that \(\|a_n\|^{1/n} > s/R\) and thus
    %     \[
    %         \|a_n x^n\| = \|a_n\| \|x\|^n > \|a_n\| \frac{R^n}{s^n} > \left(\frac{s}{R}\right)^n \cdot \frac{R^n}{s^n} = 1.
    %     \]
    %     Thus the series \(f(x)\) diverges for all \(x \in \kk\) with \(\|x\| > R\).

    %     Finally, the case \(\|x\| = R\) is direct from \cref{prop:convergence_in_ultra-metric_space}.
    % \end{proof}

    % \Yang{What about the multi-index?}


    \begin{definition}\label{def:Tate_algebra}
        Let \(\kk\) be a complete non-archimedean field.
        Let \(T = (T_1, \ldots, T_n)\) be a tuple of \(n\) indeterminates and \(r = (r_1, \ldots, r_n)\) be a tuple of \(n\) positive real numbers.
        The \emph{Tate algebra} (or \emph{ring of restricted power series}) is defined as 
        \[
            \kk\langle \underline{r^{-1}T} \rangle := \kk \{ \underline{r^{-1}T} \} := \left\{ \sum_{\alpha \in \bbN^n} a_\alpha T^\alpha \mid a_\alpha \in \kk, \lim_{|\alpha| \to +\infty} \|a_\alpha\| r^\alpha = 0 \right\}.
        \]
    \end{definition}

    \begin{proposition}\label{prop:Tate_algebra_is_a_banach_algebra_over_k}
        Let \(\kk\) be a complete non-archimedean field.
        Then the Tate algebra \(\kk\{ \underline{T/r} \} \) is a non-archimedean multiplicative banach \(\kk\)-algebra with respect to the \emph{gauss norm}
        \[
            \left\| \sum_{\alpha \in \bbN^n} a_\alpha T^\alpha \right\| := \sup_{\alpha \in \bbN^n} \|a_\alpha\|r^\alpha = \max_{\alpha \in \bbN^n} \|a_\alpha\|r^\alpha.
        \]
    \end{proposition}    
    \Yang{For the definition of banach ring, see}
    \begin{proof}
        The proof splits into several parts.
        Every parts is straightforward and standard.

        \begin{step}\label{step_in_prop:Tate_algebra_is_a_banach_algebra_over_k:k-algebra}
            We first show that \(\kk\{ \underline{T/r} \} \) is a \(\kk\)-algebra.
        \end{step}

        Easily to see that it is closed under addition and scalar multiplication.
        Suppose that \(f = \sum_{\alpha \in \bbN^n} a_\alpha T^\alpha\) and \(g = \sum_{\alpha \in \bbN^n} b_\alpha T^\alpha\) are two elements in \(\kk\{ \underline{T/r} \} \).
        Given \(\varepsilon > 0\), there exists \(N \in \bbN\) such that for all \(|\alpha| > N\), we have \(\|a_\alpha\| r^\alpha < \varepsilon/\|g\|\) and \(\|b_\alpha\| r^\alpha < \varepsilon/\|f\|\).
        For any \(|\gamma| > 2N\), we have
        \[
            \left\| \sum_{\alpha + \beta = \gamma} a_\alpha b_\beta \right\| r^\gamma \leq \max_{\alpha + \beta = \gamma} \|a_\alpha\| r^\alpha \cdot \|b_\beta\| r^\beta < \max\left\{ \frac{\varepsilon}{\|g\|} \|b_{\beta}\|r^\beta, \frac{\varepsilon}{\|f\|} \|a_{\alpha}\|r^\alpha \right\} \leq \varepsilon.
        \]
        Hence \(f \cdot g \in \kk\{ \underline{T/r} \} \) and it shows that \(\kk\{ \underline{T/r} \} \) is a \(\kk\)-algebra.

        \begin{step}\label{step_in_prop:Tate_algebra_is_a_banach_algebra_over_k:normed_k-algebra}
            Show that the gauss norm is a non-archimedean norm on \(\kk\{ \underline{T/r} \} \).
        \end{step}

        The linearity and positive-definiteness of the gauss norm are direct from the definition.
        % Suppose that \(f = \sum_{\alpha \in \bbN^n} a_\alpha T^\alpha\) and \(g = \sum_{\alpha \in \bbN^n} b_\alpha T^\alpha\) are two elements in \(\kk\{ \underline{T/r} \} \).
        We have
        \[
            \|f + g\| = \sup_{\alpha \in \bbN^n} \|a_\alpha + b_\alpha\| r^\alpha \leq \sup_{\alpha \in \bbN^n} \max\{\|a_\alpha\| + \|b_\alpha\|\} r^\alpha \leq \max\{\|f\|, \|g\|\}
        \]
        and 
        \begin{align*}
            \| f \cdot g \| &= \left\| \sum_{\gamma \in \bbN^n} \left( \sum_{\alpha + \beta = \gamma} a_\alpha b_\beta \right) T^\gamma \right\| = \sup_{\gamma \in \bbN^n} \left\| \sum_{\alpha + \beta = \gamma} a_\alpha b_\beta \right\| r^\gamma \\
            &\leq \sup_{\gamma \in \bbN^n} \max_{\alpha + \beta = \gamma} \|a_\alpha\| \|b_\beta\| r^\alpha r^\beta = \|a_{\alpha_0}\| r^{\alpha_0} \cdot \|b_{\beta_0}\| r^{\beta_0} \leq \|f\| \cdot \|g\|.
        \end{align*}
        These show that Tate algebra with the gauss norm is a non-archimedean normed \(\kk\)-algebra.

        \begin{step}\label{step_in_prop:Tate_algebra_is_a_banach_algebra_over_k:multiplicativity}
            Show that the gauss norm is multiplicative.
        \end{step}

        Suppose that \(\|f\| = \|a_{\alpha_1}\| r^{\alpha_1}\) and \(\|a_{\alpha}\|r^\alpha < \|f\|\) for all \(\alpha <_{\text{total}} \alpha_1\).
        Similar to \(\|b_{\beta_1}\| r^{\beta_1}\).
        Then we have
        \[
            \|f\| \cdot \|g\| = \|a_{\alpha_1}\| r^{\alpha_1} \cdot \|b_{\beta_1}\| r^{\beta_1} = \max_{\alpha + \beta = \alpha_1 + \beta_1} \|a_\alpha\| \|b_\beta\| r^\alpha r^\beta = \left\| \sum_{\alpha + \beta = \alpha_1 + \beta_1} a_\alpha b_\beta \right\| r^{\alpha_1 + \beta_1} \leq \| f \cdot g \|,
        \]
        where the third equality holds since \((\alpha_1, \beta_1)\) is the unique pair such that \(\|a_{\alpha_1}\| r^{\alpha_1} \cdot \|b_{\beta_1}\| r^{\beta_1}\) is maximized and by \cref{prop:all_triangles_in_ultra-metric_space_are_isosceles}.
        Thus the gauss norm is multiplicative.

        \begin{step}\label{step_in_prop:Tate_algebra_is_a_banach_algebra_over_k:completeness}
            Finally show that \(\kk\{ \underline{T/r} \} \) is complete with respect to the gauss norm.
        \end{step}

        Let \(\{f_m = \sum a_{\alpha,m}T^\alpha\}\) be a cauchy sequence in \(\kk\{ \underline{T/r} \} \).
        We have
        \[ \|a_{\alpha,m} - a_{\alpha,l}\| r^\alpha \leq \|f_m - f_l\|. \]
        Thus for each \(\alpha \in \bbN^n\), the sequence \(\{a_{\alpha,m}\}\) is a cauchy sequence in \(\kk\).
        Since \(\kk\) is complete, set \(a_\alpha := \lim_{m \to +\infty} a_{\alpha,m}\) and \(f = \sum_{\alpha \in \bbN^n} a_\alpha T^\alpha\).
        Given \(\varepsilon > 0\), there exists \(M \in \bbN\) such that for all \(m,l > M\), we have \(\|f_m - f_l\| < \varepsilon\).
        Fixing \(m > M\), there exists \(N \in \bbN\) such that for all \(|\alpha| > N\), we have \(\|a_{\alpha,m}\| r^\alpha < \varepsilon\).
        Hence for all \(|\alpha| > N\) and \(l > M\), we have
        \[ \|a_{\alpha,l}\| r^\alpha \leq \|a_{\alpha,l} - a_{\alpha,m}\| r^\alpha + \|a_{\alpha,m}\| r^\alpha < 2\varepsilon. \]
        Taking \(l \to +\infty\), we have \(\|a_\alpha\| r^\alpha \leq 2\varepsilon\) for all \(|\alpha| > N\).
        It follows that \(f \in \kk\{ \underline{T/r} \} \).

        For every \(\varepsilon > 0\), there exists \(N \in \bbN\) such that for all \(m,l > N\), we have \(\|f_m - f_l\| < \varepsilon\).
        Thus for all \(\alpha \in \bbN^n\) and \(m,l > N\), we have
        \[ \|a_{\alpha,m} - a_{\alpha,l}\| r^\alpha \leq \|f_m - f_l\| < \varepsilon. \]
        Taking \(l \to +\infty\), we have \(\|a_{\alpha,m} - a_\alpha\| r^\alpha \leq \varepsilon\) for all \(m > N\).
        It follows that
        \[ \|f - f_m\| = \sup_{\alpha \in \bbN^n} \|a_\alpha - a_{\alpha,m}\| r^\alpha \leq \varepsilon \]
        for all \(m > N\).
    \end{proof}

    \begin{proposition}\label{prop:Tate_algebra_as_functions_ring}
        Let \(\kk\) be a complete non-archimedean field.
        Then the Tate algebra \(\kk\{ \underline{T/r} \} \) can be identified with a subring of the ring of all functions from the closed polydisc \(E(0,\underline{r}) \subset \kk^n\) to \(\kk\).
        % \Yang{To be checked.}
    \end{proposition}
    \begin{proof}
        Given \(f = \sum_{\alpha \in \bbN^n} a_\alpha T^\alpha \in \kk\{ \underline{T/r} \} \) and \(x = (x_1, \ldots, x_n) \in E(0,\underline{r})\), we have
        \[ \left\|\sum_{|\alpha| = n} a_\alpha x^\alpha\right\| \leq \max_{|\alpha| = n} \|a_\alpha\| r^\alpha \to 0 \quad \text{ as } n \to +\infty. \]
        Hence by \cref{prop:convergence_in_ultra-metric_space}, the series \(f(x) := \sum_{\alpha \in \bbN^n} a_\alpha x^\alpha\) converges in \(\kk\).
        This defines a function \(f:E(0,\underline{r}) \to \kk\).

        Let \(g = \sum_{\alpha \in \bbN^n} b_\alpha T^\alpha \in \kk\{ \underline{T/r} \} \).
        Set 
        \[ A_n = \sum_{|\alpha|<n} a_\alpha x^\alpha, \quad B_n = \sum_{|\beta|<n} b_\beta x^\beta, \quad C_n = \sum_{|\gamma|<n} \left(\sum_{\alpha+\beta = \gamma} a_\alpha b_\beta \right) x^\gamma. \]
        We need to show that \(f(x)g(x) = \lim A_nB_n = \lim C_n = (fg)(x)\).
        Note that 
        \[ A_nB_n - C_n = \sum_{\substack{|\alpha|<n, |\beta|<n \\ |\alpha + \beta| \geq n}} a_\alpha b_\beta x^{\alpha + \beta}. \]
        Given \(\varepsilon > 0\), there exists \(N \in \bbN\) such that for all \(|\alpha| > N\), we have \(\|a_\alpha\| r^\alpha < \varepsilon/\|g\|\) and \(\|b_\alpha\| r^\alpha < \varepsilon/\|f\|\).
        For any \(n > 2N\), we have
        \[ \|A_nB_n - C_n\| \leq \max_{\substack{|\alpha|<n, |\beta|<n \\ |\alpha + \beta| \geq n}} \|a_\alpha\| \|b_\beta\| \|x^{\alpha + \beta}\| < \max\left\{ \frac{\varepsilon}{\|g\|} \|b_{\beta}\|r^\beta, \frac{\varepsilon}{\|f\|} \|a_{\alpha}\|r^\alpha \right\} \leq \varepsilon. \]
        Thus \(f(x)g(x) = (fg)(x)\).
        The addition and scalar multiplication can be verified directly.
        Hence above assignments define a homomorphism of \(\kk\)-algebras from \(\kk\{ \underline{T/r} \} \) to the ring of all functions from \(E(0,\underline{r})\) to \(\kk\).

        Finally, \cref{prop:norm_of_Tate_algebra_coincides_with_the_suprum_norm} ensures that the homomorphism is injective.
    \end{proof}

    \begin{proposition}\label{prop:norm_of_Tate_algebra_coincides_with_the_suprum_norm}
        Let \(\kk\) be a complete non-archimedean field.
        Then the gauss norm on the Tate algebra \(\kk\{ \underline{T/r} \} \) coincides with the supremum norm
        \[
            \|f\|_{\sup} := \sup_{x \in E(0,\underline{r})} \|f(x)\|_{\kk}.
        \]
        % \Yang{To be checked.}
    \end{proposition}
    \begin{proof}
        \Yang{To be added.}
    \end{proof}

    \begin{remark}
        \Yang{comparison with the Weierstrass-Stone theorem in classical analysis.}
    \end{remark}



\subsection{Fundamental properties}

    Then following shows that analytic functions over non-archimedean fields share some nice properties as in the case of complex analysis.
    \Yang{To be revised.}

    \begin{theorem}\label{prop:analytic_function_is_lipschitz}
        Let \((\kk,\|\cdot\|)\) be a complete non-archimedean field and \(U \subset \kk\) be an open subset.
        If \(f:U \to \kk\) is an analytic function, then \(f\) is locally Lipschitz continuous on \(U\).
        \Yang{To be checked.}
    \end{theorem}

    \begin{theorem}[Strassman]\label{prop:rigidity_of_analytic_series}
        Let \(\kk\) be a complete non-archimedean field with non-trivial valuation and \(f = \sum a_n T^n \in \kk\{T/r\}\) be an analytic function.
        Suppose that \(\|a_N\| > \|a_n\|\) for all \(n > N\).
        Then \(f\) has at most \(N\) zeros in the closed ball \(E(0,r)\).
        \Yang{To be checked.}
    \end{theorem}
    \begin{proof}
        \Yang{To be add.}
    \end{proof}