\section{Ultra-metric spaces}

    \begin{definition}\label{def:ultra-metric_space}
        A metric space \((X,d)\) is called an \emph{ultra-metric space} if its metric \(d\) satisfies the \emph{strong triangle inequality}:
        \[ d(x,z) \leq \max\{d(x,y), d(y,z)\},\quad\forall x,y,z\in X. \]
    \end{definition}

    \begin{proposition}\label{prop:balls_in_ultra-metric_space}
        Let \((X,d)\) be an ultra-metric space.
        Then for any \(x \in X\) and \(r > 0\), the closed ball \(B(x,r) := \{y \in X\colon d(x,y) \leq r\}\) satisfies the following properties:
        \begin{enumerate}
            \item For any \(y \in B(x,r)\), we have \(B(x,r) = B(y,r)\).
            \item Any two closed balls in \(X\) are either disjoint or one is contained in the other.
        \end{enumerate}
        \Yang{To be revised.}
    \end{proposition}

    We will use \(B(x,r)\) to denote the open ball with center \(x\) and radius \(r\).
    We will use \(E(x,r)\) to denote the closed ball with center \(x\) and radius \(r\).

    \begin{proposition}\label{prop:ultra-metric_space_is_totally_disconnected}
        Let \((X,d)\) be an ultra-metric space.
        Then \(X\) is totally disconnected, i.e., the only connected subsets of \(X\) are the singletons.
        \Yang{To be revised.}
    \end{proposition}
