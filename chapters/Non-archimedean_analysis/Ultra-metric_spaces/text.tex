\section{Ultra-metric spaces}

    We will use \(B(x,r)\) (resp. \(E(x,r)\)) to denote the open ball (resp. closed ball) with center \(x\) and radius \(r\).
    % We will use \(E(x,r)\) to denote the closed ball with center \(x\) and radius \(r\).

    \begin{definition}\label{def:ultra-metric_space}
        A metric space \((X,d)\) is called an \emph{ultra-metric space} if its metric \(d\) satisfies the \emph{strong triangle inequality}:
        \[ d(x,z) \leq \max\{d(x,y), d(y,z)\},\quad\forall x,y,z\in X. \]
    \end{definition}

    If \((\kk,\|\cdot\|)\) is a non-archimedean field, then the metric \(d(x,y) := \|x-y\|\) on \(\kk\) makes \((\kk,d)\) an ultra-metric space.

    \begin{proposition}\label{prop:all_triangles_in_ultra-metric_space_are_isosceles}
        Let \((X,d)\) be an ultra-metric space.
        Then for any \(x,y,z \in X\), at least two of the three distances \(d(x,y), d(y,z), d(z,x)\) are equal.
        And the third distance is less than or equal to the common value of the other two.
    \end{proposition}
    \begin{proof}
        Suppose that \(d(x,y) \geq d(y,z)\).
        By the strong triangle inequality, we have
        \[ d(x,z) \leq \max\{d(x,y), d(y,z)\} = d(x,y). \]
        On the other hand, by the strong triangle inequality again, we have
        \[ d(x,y) \leq \max\{d(x,z), d(z,y)\} = \max\{d(x,z), d(y,z)\} \leq d(x,y). \]
        This shows that \(d(x,y) = \max\{d(x,z), d(y,z)\}\).
        Thus either \(d(x,z) = d(x,y) \geq d(y,z)\) or \(d(y,z) = d(x,y) \geq d(x,z)\).
        %\Yang{To be continued.}
    \end{proof}

    \begin{proposition}\label{prop:balls_in_ultra-metric_space_form_a_tree}
        Let \((X,d)\) be an ultra-metric space.
        Let \(D_i\) be (open or closed) ball in \(X\) for \(i=1,2\).
        If \(D_1 \cap D_2 \neq \emptyset\), then either \(D_1 \subseteq D_2\) or \(D_2 \subseteq D_1\).
    \end{proposition}
    \begin{proof}
        Suppose that \(D_i\) has center \(x_i\) and radius \(r_i\) for \(i=1,2\).
        Let \(y \in D_1 \cap D_2\).
        We have 
        \[ d(x_1,x_2) \leq \max\{d(x_1,y), d(y,x_2)\}. \]
        Without loss of generality, we may assume that \(d(x_1,x_2) \leq d(x_1,y)\).
        It follows that \(x_2 \in D_1\) since \(d(x_1,y) < r_1\) (or \(\leq r_1\)).

        If there exists \(z \in D_2 \setminus D_1\), we claim that \(D_1 \subseteq D_2\).
        We have \(d(x_1,z) > d(x_1,x_2)\).
        Then by \cref{prop:all_triangles_in_ultra-metric_space_are_isosceles},
        \[ r_1 \leq d(x_1,z) = d(x_2,z) \leq r_2. \]
        In particular, if \(D_2\) is an open ball, then we have strict inequality \(r_1 < r_2\).
        For any \(w \in D_1\), we have 
        \[ d(x_2,w) \leq \max\{d(x_2,x_1), d(x_1,w)\} \leq r_1 \leq r_2. \]
        Thus \(w \in D_2\) whatever \(D_2\) is open or closed, and it shows that \(D_1 \subseteq D_2\).
    \end{proof}

    \begin{proposition}\label{prop:all_balls_in_ultra-metric_space_are_clopen}
        Let \((X,d)\) be an ultra-metric space.
        Then both \(B(x,r)\) and \(E(x,r)\) are closed and open subsets of \(X\) for any \(x \in X\) and \(r > 0\).
    \end{proposition}
    \begin{proof}
        We show that the sphere \(S(x,r) := \{y \in X \mid d(x,y) = r\}\) is open in \(X\).
        Note that if \(y \in S(x,r)\), then for any \(r' < r\), we have \(B(y,r') \cap E(x,r) \neq \emptyset\) and \(x \in E(x,r) \setminus B(y,r')\).
        Thus by \cref{prop:balls_in_ultra-metric_space_form_a_tree}, we have \(B(y,r') \subseteq E(x,r)\).
        If \(B(y,r') \cap B(x,r) \neq \emptyset\), then by \cref{prop:balls_in_ultra-metric_space_form_a_tree} again, we have \(B(y,r') \subseteq B(x,r)\).
        However, \(y \in B(y,r') \setminus B(x,r)\), a contradiction.
        Thus \(B(y,r') \subseteq E(x,r) \setminus B(x,r) = S(x,r)\).
        It yields that \(S(x,r) = \bigcup_{y \in S(x,r)} B(y,r/2)\) is open in \(X\).

        Since \(E(x,r) = B(x,r) \cup S(x,r)\) and \(B(x,r) = E(x,r) \setminus S(x,r)\), both \(B(x,r)\) and \(E(x,r)\) are open and closed in \(X\).
    \end{proof}

    \begin{corollary}\label{prop:ultra-metric_space_is_totally_disconnected}
        Let \((X,d)\) be an ultra-metric space.
        Then \(X\) is totally disconnected, i.e., the only connected subsets of \(X\) are the set with at most one point.
    \end{corollary}
    \begin{proof}
        Suppose that \(S \subset X\) has at least two distinct points \(x,y \in S\).
        Let \(r := d(x,y) > 0\).
        Consider the open ball \(B(x,r/2)\).
        By \cref{prop:all_balls_in_ultra-metric_space_are_clopen}, \(B(x,r/2)\) is both open and closed in \(X\).
        Thus \(B(x,r/2) \cap S\) is both open and closed in \(S\), however, it is non-empty and not equal to \(S\) since it contains \(x\) but not \(y\).
        This shows that \(S\) is disconnected.
    \end{proof}
