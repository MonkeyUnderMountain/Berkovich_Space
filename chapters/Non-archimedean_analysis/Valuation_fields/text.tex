\section{Valuation fields}

\subsection{Absolute values and completion}

    \begin{definition}\label{def:valuation_field}
        Let \(\kk\) be a field.
        An \emph{absolute value} on \(\kk\) is a function \(\|\cdot|:\kk\to\bbR_{\ge0}\) satisfying the following properties for all \(x,y\in\kk\):
        \begin{enumerate}
            \item \(\|x\|=0\) if and only if \(x=0\);
            \item \(\|xy\|=\|x\|\cdot\|y\|\);
            \item \(\|x+y\|\leq\|x\|+\|y\|\).
        \end{enumerate}
        A field \(\kk\) equipped with an absolute value \(\|\cdot\|\) is called a \emph{valuation field}.
    \end{definition}

    \begin{remark}\label{rmk:additive_and_multiplicative_valuation_on_a_field}
        Let \(\kk\) be a field.
        Recall that a \emph{valuation} on \(\kk\) is a function \(v: \kk^\times \to \bbR\) such that
        \begin{itemize}
            \item \(\forall x,y \in \kk^\times, v(xy) = v(x) + v(y)\);
            \item \(\forall x,y \in \kk^\times, v(x + y) \geq \min\{v(x), v(y)\}\).
        \end{itemize}
        We can extend \(v\) to the whole field \(\kk\) by defining \(v(0) = +\infty\).
        Fix a real number \(\varepsilon \in (0,1)\).
        Then \(v\) induces an absolute value \(|\cdot|_v: \kk \to \bbR_+\) defined by \(|x|_v = \varepsilon^{v(x)}\) for each \(x \in \kk\).

        In some literature, the valuation \(v\) is called an \emph{additive valuation} and the induced absolute value \(|\cdot|_v\) is called a \emph{multiplicative valuation}.
        In this note, the term \emph{valuation} always refers to the additive valuation.
    \end{remark}

    \begin{definition}\label{def:complete_valuation_field}
        Let \((\kk,\|\cdot\|)\) be a valuation field.
        We say that \(\kk\) is \emph{complete} if the metric \(d(x,y) := \|x - y\|\) makes \(\kk\) a complete metric space.
    \end{definition}

    \begin{lemma}\label{lem:completion_of_valuation_field}
        Let \((\kk,\|\cdot\|)\) be a valuation field.
        Let \((\widehat{\kk},\|\cdot\|)\) be its completion as a metric space.
        Then the operations of addition and multiplication on \(\kk\) can be extended to \(\widehat{\kk}\) uniquely, making \((\widehat{\kk},\|\cdot\|)\) a complete valuation field containing \(\kk\) as a dense subfield.
    \end{lemma}
    \begin{proof}
        \Yang{To be added.}
    \end{proof}

    Unlike the real number field \(\bbR\), even a valuation field is complete, we can not expect the theorem of nested intervals to hold.

    \begin{definition}\label{def:spherically_complete}
        A valuation field \((\kk,\|\cdot\|)\) is called \emph{spherically complete} if every decreasing sequence of closed balls in \(\kk\) has a non-empty intersection.
    \end{definition}

    \begin{example}\label{eg:p-adic_complex_field_is_not_spherically_complete}
        The field \(\bbC_p\) of \(p\)-adic complex numbers is not spherically complete, see \Yang{to be added.}
    \end{example}

\subsection{Non-archimedean fields}

    \begin{definition}\label{def:non-archimedean_fields}
        Let \((\kk,\|\cdot\|)\) be a valuation field.
        We say that \(\kk\) is \emph{non-archimedean} if its absolute value \(\|\cdot\|\) satisfies the \emph{strong triangle inequality}:
        \[ \|x+y\|\leq\max\{\|x\|,\|y\|\},\quad\forall x,y\in\kk. \]
        Otherwise, we say that \(\kk\) is \emph{archimedean}.
    \end{definition}

    Let \(\kk\) be a non-archimedean field.
    Then easily see that \(\{x \in \kk\colon \|x\|\leq 1\}\) is a subring of \(\kk\).
    Moreover, it is a local ring whose maximal ideal is \(\{x \in \kk\colon \|x\| < 1\}\).

    \begin{definition}\label{def:non-archimedean_field_ring_of_integers_maximal_ideal_and_residue_field}
        Let \(\kk\) be a non-archimedean field.
        The \emph{ring of integers} of \(\kk\) is defined as
        \[ \kk^\circ := \{x \in \kk\colon \|x\|\leq 1\}. \]
        Its maximal ideal is
        \[ \kk^{\circ\circ} := \{x \in \kk\colon \|x\| < 1\}. \]
        The \emph{residue field} of \(\kk\) is defined as
        \[ \calk_\kk := \widetilde{\kk} := \kk^\circ / \kk^{\circ\circ}. \]
    \end{definition}

    \Yang{Is the valuation on residue field trivial?}

    \begin{lemma}\label{prop:totally_bounded_and_complete_and_compact}
        Recall that a metric space is \emph{totally bounded} if for every \(\varepsilon > 0\), it can be covered by finitely many balls of radius \(\varepsilon\).
        A metric space is compact if and only if it is complete and totally bounded.
    \end{lemma}
    \begin{proof}
        \Yang{To be added.}
    \end{proof}

    \begin{definition}\label{def:residue_absolute_value_on_residue_fields}
        Let \(\kk\) be a non-archimedean field.
        The \emph{residue absolute value} on the residue field \(\calk_\kk\) is defined as
        \[ |x| := \inf_{y \in \varphi^{-1}(x)} \|y\|, \quad \forall x \in \calk_\kk, \]
        where \(\varphi: \kk^\circ \to \calk_\kk\) is the canonical projection.
    \end{definition}

    \begin{proposition}\label{prop:residue_absolute_value_on_residue_fields_is_trivial}
        Let \(\kk\) be a non-archimedean field.
        Then the residue absolute value on the residue field \(\calk_\kk\) is trivial.
    \end{proposition}
    \begin{proof}
        For any \(x \in \calk_\kk\), if \(x = 0\), then by definition \(|x| = 0\).
        If \(x \neq 0\), then \(\forall y \in \varphi^{-1}(x)\), we have \(y \in \kk^\circ \setminus \kk^{\circ\circ}\), i.e., \(\|y\| = 1\).
        Thus by definition \(|x| = 1\).
        % \Yang{To be added.}
    \end{proof}

    \begin{slogan}
        Locally compact \(\iff\) complete + pro-finite.
    \end{slogan}

    % \begin{proposition}\label{prop:balls_in_ultra-metric_space}
    %     Let \((X,d)\) be an ultra-metric space.
    %     Then for any \(x \in X\) and \(r > 0\), the closed ball \(B(x,r) := \{y \in X\colon d(x,y) \leq r\}\) satisfies the following properties:
    %     \begin{enumerate}
    %         \item For any \(y \in B(x,r)\), we have \(B(x,r) = B(y,r)\).
    %         \item Any two closed balls in \(X\) are either disjoint or one is contained in the other.
    %     \end{enumerate}
    %     \Yang{To be revised.}
    % \end{proposition}

    % We will use \(B(x,r)\) to denote the open ball with center \(x\) and radius \(r\).
    % We will use \(E(x,r)\) to denote the closed ball with center \(x\) and radius \(r\).

    % \begin{proposition}\label{prop:ultra-metric_space_is_totally_disconnected}
    %     Let \((X,d)\) be an ultra-metric space.
    %     Then \(X\) is totally disconnected, i.e., the only connected subsets of \(X\) are the singletons.
    %     \Yang{To be revised.}
    % \end{proposition}



    
