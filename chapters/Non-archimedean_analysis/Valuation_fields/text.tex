\section{Valuation fields}

\subsection{Absolute values and completion}

    \begin{definition}\label{def:valuation_field}
        Let \(\kk\) be a field.
        An \emph{absolute value} on \(\kk\) is a function \(\|\cdot|:\kk\to\bbR_{\ge0}\) satisfying the following properties for all \(x,y\in\kk\):
        \begin{enumerate}
            \item \(\|x\|=0\) if and only if \(x=0\);
            \item \(\|xy\|=\|x\|\cdot\|y\|\);
            \item \(\|x+y\|\leq\|x\|+\|y\|\).
        \end{enumerate}
        A field \(\kk\) equipped with an absolute value \(\|\cdot\|\) is called a \emph{valuation field}.
    \end{definition}

    \begin{remark}\label{rmk:additive_and_multiplicative_valuation_on_a_field}
        Let \(\kk\) be a field.
        Recall that a \emph{valuation} on \(\kk\) is a function \(v: \kk^\times \to \bbR\) such that
        \begin{itemize}
            \item \(\forall x,y \in \kk^\times, v(xy) = v(x) + v(y)\);
            \item \(\forall x,y \in \kk^\times, v(x + y) \geq \min\{v(x), v(y)\}\).
        \end{itemize}
        We can extend \(v\) to the whole field \(\kk\) by defining \(v(0) = +\infty\).
        Fix a real number \(\varepsilon \in (0,1)\).
        Then \(v\) induces an absolute value \(|\cdot|_v: \kk \to \bbR_+\) defined by \(|x|_v = \varepsilon^{v(x)}\) for each \(x \in \kk\).

        In some literature, the valuation \(v\) is called an \emph{additive valuation} and the induced absolute value \(|\cdot|_v\) is called a \emph{multiplicative valuation}.
        In this note, the term \emph{valuation} always refers to the additive valuation.
    \end{remark}

    \begin{example}\label{eg:trivial_absolute_value_on_any_fields}
        Let \(\kk\) be a field.
        The \emph{trivial absolute value} on \(\kk\) is defined as
        \[ \|x\| := \begin{cases}
            0, & x = 0; \\
            1, & x \neq 0.
        \end{cases} \]
    \end{example}

    \begin{definition}\label{def:valuation_group_of_a_valuation_field}
        The \emph{(multiplicative) valuation group} of a valuation field \((\kk,\|\cdot\|)\) is defined as the subgroup of \(\bbR_{>0}\) given by
        \[ |\kk^\times| := \{\|x\|\colon x \in \kk^\times\}. \]
        We use the notation \(\sqrt{|\kk^\times|}\) to denote the set \(\{\|x\|^{1/n}\colon x \in \kk^\times, n \in \bbZ_{>0}\}\).
    \end{definition}

    \begin{definition}\label{def:equivalent_of_absolute_values}
        Let \(\kk\) be a field.
        Two absolute values \(\|\cdot\|_1\) and \(\|\cdot\|_2\) on \(\kk\) are said to be \emph{equivalent} if there exists a real number \(c \in (0,1)\) such that
        \[ \|x\|_1 = \|x\|_2^c, \quad \forall x \in \kk. \]
    \end{definition}

    Note that equivalent absolute values induce the same topology on the field \(\kk\).
    Moreover, the following lemma shows that the converse is also true.

    \begin{lemma}\label{prop:equivalent_of_absolute_values_and_topology_and_unit_disk}
        Let \(\kk\) be a field and \(\|\cdot\|_1\), \(\|\cdot\|_2\) be two absolute values on \(\kk\).
        Then the following statements are equivalent:
        \begin{enumerate}
            \item \(\|\cdot\|_1\) and \(\|\cdot\|_2\) are equivalent;
            \item \(\|\cdot\|_1\) and \(\|\cdot\|_2\) induce the same topology on \(\kk\);
            \item The unit disks \(D_1 = \{x \in \kk\colon \|x\|_1 < 1\}\) and \(D_2 = \{x \in \kk\colon \|x\|_2 < 1\}\) are the same.
        \end{enumerate}
    \end{lemma}
    \begin{proof}
        The implications (a) \(\Rightarrow\) (b) is obvious.
        Now we prove (b) \(\Rightarrow\) (c).
        For any \(x \in D_1\), we have \(x^n \to 0\) as \(n \to \infty\) under the absolute value \(\|\cdot\|_1\) and thus under \(\|\cdot\|_2\).
        Therefore, \(\|x\|_2^n \to 0\) as \(n \to \infty\), which implies that \(\|x\|_2 < 1\), i.e., \(x \in D_2\).
        Similarly, we can prove that \(D_2 \subseteq D_1\).

        Finally, we prove (c) \(\Rightarrow\) (a).
        If \(\|\cdot\|_1\) is trivial, then \(D_1 = \{0\}\) and thus \(\|\cdot\|_2\) is also trivial.
        In this case, they are equivalent.
        Suppose that both \(\|\cdot\|_1\) and \(\|\cdot\|_2\) are non-trivial.
        Pick any \(x,y \notin D_1 = D_2\).
        Then there exist real numbers \(\alpha, \beta > 0\) such that \(\|x\|_1 = \|x\|_2^\alpha\) and \(\|y\|_1 = \|y\|_2^\beta\).
        Suppose the contrary that \(\alpha \neq \beta\).
        Consider the domain \(\Omega \subseteq \bbZ^2\) defined by
        \[ \begin{cases}
            n \log \|x\|_2 < m \log \|y\|_2; \\
            n \alpha \log \|x\|_2 > m \beta \log \|y\|_2.
        \end{cases} \]
        Since \(\alpha \neq \beta\), the two lines defined by the equalities are not parallel.
        Thus \(\Omega\) is non-empty.
        Pick \((n,m) \in \Omega\) and set \(z := x^n y^{-m}\).
        Then we have \(\|z\|_2 < 1\) and \(\|z\|_1 > 1\), a contradiction.
    \end{proof}

    \begin{definition}\label{def:complete_valuation_field}
        Let \((\kk,\|\cdot\|)\) be a valuation field.
        We say that \(\kk\) is \emph{complete} if the metric \(d(x,y) := \|x - y\|\) makes \(\kk\) a complete metric space.
    \end{definition}

    \begin{lemma}\label{lem:completion_of_valuation_field}
        Let \((\kk,\|\cdot\|)\) be a valuation field and \((\widehat{\kk},\|\cdot\|)\) its completion as a metric space.
        Then the operations of addition and multiplication on \(\kk\) can be extended to \(\widehat{\kk}\) uniquely, making \((\widehat{\kk},\|\cdot\|)\) a complete valuation field containing \(\kk\) as a dense subfield.
    \end{lemma}
    \begin{proof}
        Simple analysis.
    \end{proof}

    \begin{example}\label{eg:usual_absolute_value_on_Q}
        Let \(|\cdot|_\infty\) be the usual absolute value on the field \(\bbQ\) of rational numbers.
        Then \((\bbQ, |\cdot|_\infty)\) is a valuation field.
        Its completion is the field \(\bbR\) of real numbers equipped with the usual absolute value.
    \end{example}

    \begin{example}\label{eg:p-adic_absolute_value_on_Q}
        Let \(p\) be a prime number.
        For any non-zero rational number \(x \in \bbQ\), we can write it as \(x = p^n \frac{a}{b}\), where \(n \in \bbZ\) and \(a,b \in \bbZ\) are integers not divisible by \(p\).
        The \emph{\(p\)-adic absolute value} on \(\bbQ\) is defined as
        \[ |x|_p := \begin{cases}
            0, & x = 0; \\
            p^{-n}, & x = p^n \frac{a}{b} \text{ as above}.
        \end{cases} \]
        Then \((\bbQ, |\cdot|_p)\) is a valuation field.
        Its completion is the field 
        \[ \bbQ_p = \left\{\sum_{n = k}^{+\infty} a_n p^n \colon k \in \bbZ, a_n \in \{0,1,\ldots,p-1\}\right\} \]
        of \(p\)-adic numbers equipped with the \(p\)-adic absolute value; see \Yang{to be added.}.
    \end{example}
    
    \begin{proposition}\label{prop:non-trivial_complete_field_is_uncountable}
        Let \((\kk,\|\cdot\|)\) be a complete valuation field with non-trivial absolute value.
        Then \(\kk\) is uncountable.
    \end{proposition}
    \begin{proof}
        Since the absolute value \(\|\cdot\|\) is non-trivial, we can construct a sequence \(\{x_n\}_{n=1}^\infty \subseteq \kk\) inductively such that \(\|x_n\| < \|x_{n-1}\|/2\) for any \(n\geq 1\) and \(\|x_0\| < 1\).
        Then there is an injective map from \(\bbN^{\{0,1\}}\) to \(\kk\) defined by
        \[ (a_n)_{n=1}^\infty \mapsto \sum_{n=1}^\infty a_n x_n, \quad a_n \in \{0,1\}. \]
        Since \(\|x_n\| < 2^{-n}\), the series \(\sum_{n=1}^\infty a_n x_n\) converges in \(\kk\).
        Note \(\|x_n\| > \|\sum_{m\geq n} x_m\|\) for each \(n\), we have that the map is injective.
        Thus \(\kk\) is uncountable.
        % \Yang{To be added.}
    \end{proof}

    Unlike the real number field \(\bbR\), even a valuation field is complete, we can not expect the theorem of nested intervals to hold.

    \begin{definition}\label{def:spherically_complete}
        A valuation field \((\kk,\|\cdot\|)\) is called \emph{spherically complete} if every decreasing sequence of closed balls in \(\kk\) has a non-empty intersection.
    \end{definition}

    \begin{example}\label{eg:p-adic_complex_field_is_not_spherically_complete}
        The field \(\bbC_p\) of \(p\)-adic complex numbers is not spherically complete, see \Yang{to be added.}
    \end{example}

\subsection{Non-archimedean fields}

    \begin{definition}\label{def:non-archimedean_fields}
        Let \((\kk,\|\cdot\|)\) be a valuation field.
        We say that \(\kk\) is \emph{non-archimedean} if its absolute value \(\|\cdot\|\) satisfies the \emph{strong triangle inequality}:
        \[ \|x+y\|\leq\max\{\|x\|,\|y\|\},\quad\forall x,y\in\kk. \]
        Otherwise, we say that \(\kk\) is \emph{archimedean}.
    \end{definition}

    Let \(\kk\) be a non-archimedean field.
    Then easily see that \(\{x \in \kk\colon \|x\|\leq 1\}\) is a subring of \(\kk\).
    Moreover, it is a local ring whose maximal ideal is \(\{x \in \kk\colon \|x\| < 1\}\).

    \begin{definition}\label{def:non-archimedean_field_ring_of_integers_maximal_ideal_and_residue_field}
        Let \(\kk\) be a non-archimedean field.
        The \emph{ring of integers} of \(\kk\) is defined as
        \[ \kk^\circ := \{x \in \kk\colon \|x\|\leq 1\}. \]
        Its maximal ideal is
        \[ \kk^{\circ\circ} := \{x \in \kk\colon \|x\| < 1\}. \]
        The \emph{residue field} of \(\kk\) is defined as
        \[ \calk_\kk := \widetilde{\kk} := \kk^\circ / \kk^{\circ\circ}. \]
    \end{definition}

    % \Yang{Is the valuation on residue field trivial?}

    % \begin{lemma}\label{prop:totally_bounded_and_complete_and_compact}
    %     Recall that a metric space is \emph{totally bounded} if for every \(\varepsilon > 0\), it can be covered by finitely many balls of radius \(\varepsilon\).
    %     A metric space is compact if and only if it is complete and totally bounded.
    % \end{lemma}
    % \begin{proof}
    %     \Yang{To be added.}
    % \end{proof}

    \begin{definition}\label{def:residue_absolute_value_on_residue_fields}
        Let \(\kk\) be a non-archimedean field.
        The \emph{residue absolute value} on the residue field \(\calk_\kk\) is defined as
        \[ |x| := \inf_{y \in \varphi^{-1}(x)} \|y\|, \quad \forall x \in \calk_\kk, \]
        where \(\varphi: \kk^\circ \to \calk_\kk\) is the canonical projection.
    \end{definition}

    \begin{proposition}\label{prop:residue_absolute_value_on_residue_fields_is_trivial}
        Let \(\kk\) be a non-archimedean field.
        Then the residue absolute value on the residue field \(\calk_\kk\) is trivial.
    \end{proposition}
    \begin{proof}
        For any \(x \in \calk_\kk\), if \(x = 0\), then by definition \(|x| = 0\).
        If \(x \neq 0\), then \(\forall y \in \varphi^{-1}(x)\), we have \(y \in \kk^\circ \setminus \kk^{\circ\circ}\), i.e., \(\|y\| = 1\).
        Thus by definition \(|x| = 1\).
    \end{proof}

    % \begin{proposition}\label{prop:recover_complete_non-archimedean_fields_from_projective_limits}
    %     Let \(\kk\) be a non-archimedean field.
    %     Set \(I_r := \{x \in \kk\colon \|x\| < r\}\) for each \(r \in (0,1)\).
    %     They are ideals of the ring of integers \(\kk^\circ\).
    %     Then we have 
    %     \[ \widehat{\kk}^\circ \cong \varprojlim_{r > 0} \kk^\circ / I_r. \]
    %     \Yang{To be checked.}
    % \end{proposition}

    % \begin{slogan}
    %     Locally compact \(\iff\) pro-finite.
    % \end{slogan}

    % \begin{proposition}\label{prop:locally_compact_NA_field_iff_it_is_pro-finite}
    %     Let \(\kk\) be a non-archimedean field.
    %     Then \(\kk\) is totally bounded iff \(\kk^\circ / I_r\) is finite for each \(r \in (0,1)\).
    % \end{proposition}

    % \begin{proposition}\label{prop:}
    %     \(\kk^\circ\) is noetherian iff \(\kk\) is a discrete valuation field.
    %     and complete.
    %     \Yang{To be revised.}
    % \end{proposition}

