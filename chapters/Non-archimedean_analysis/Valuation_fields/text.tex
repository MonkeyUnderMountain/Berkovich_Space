\section{Valuation fields}


    % \begin{definition}\label{def:valuation_field}
    %     Let \(\kk\) be a field.
    %     An \emph{absolute value} on \(\kk\) is a function \(\|\cdot|:\kk\to\bbR_{\ge0}\) satisfying the following properties for all \(x,y\in\kk\):
    %     \begin{enumerate}
    %         \item \(\|x\|=0\) if and only if \(x=0\);
    %         \item \(\|xy\|=\|x\|\cdot\|y\|\);
    %         \item \(\|x+y\|\leq\|x\|+\|y\|\).
    %     \end{enumerate}
    %     A field \(\kk\) equipped with an absolute value \(\|\cdot\|\) is called a \emph{valuation field}.
    % \end{definition}

    % \begin{remark}\label{rmk:additive_and_multiplicative_valuation_on_a_field}
    %     Let \(\kk\) be a field.
    %     Recall that a \emph{valuation} on \(\kk\) is a function \(v: \kk^\times \to \bbR\) such that
    %     \begin{itemize}
    %         \item \(\forall x,y \in \kk^\times, v(xy) = v(x) + v(y)\);
    %         \item \(\forall x,y \in \kk^\times, v(x + y) \geq \min\{v(x), v(y)\}\).
    %     \end{itemize}
    %     We can extend \(v\) to the whole field \(\kk\) by defining \(v(0) = +\infty\).
    %     Fix a real number \(\varepsilon \in (0,1)\).
    %     Then \(v\) induces an absolute value \(|\cdot|_v: \kk \to \bbR_+\) defined by \(|x|_v = \varepsilon^{v(x)}\) for each \(x \in \kk\).

    %     In some literature, the valuation \(v\) is called an \emph{additive valuation} and the induced absolute value \(|\cdot|_v\) is called a \emph{multiplicative valuation}.
    %     In this note, the term \emph{valuation} always refers to the additive valuation.
    % \end{remark}

    % \begin{definition}\label{def:complete_valuation_field}
    %     Let \((\kk,\|\cdot\|)\) be a valuation field.
    %     We say that \(\kk\) is \emph{complete} if the metric \(d(x,y) := \|x - y\|\) makes \(\kk\) a complete metric space.
    % \end{definition}

    % \begin{lemma}\label{lem:completion_of_valuation_field}
    %     Let \((\kk,\|\cdot\|)\) be a valuation field.
    %     Let \((\widehat{\kk},\|\cdot\|)\) be its completion as a metric space.
    %     Then the operations of addition and multiplication on \(\kk\) can be extended to \(\widehat{\kk}\) uniquely, making \((\widehat{\kk},\|\cdot\|)\) a complete valuation field containing \(\kk\) as a dense subfield.
    % \end{lemma}

    % % \begin{theorem}[Classification of absolute values on \(\bbQ\)]\label{thm:classification_absolute_values_Q}
    % %     Let \(\|\cdot\|\) be an absolute value on \(\bbQ\).
    % %     Then \(\|\cdot\|\) is either equivalent to the usual absolute value \(|\cdot|_\infty\) or to a \(p\)-adic absolute value \(|\cdot|_p\) for some prime \(p\).
    % %     \Yang{The name is to be added.}
    % % \end{theorem}

    % \begin{definition}\label{def:spherically_complete}
    %     A valuation field \((\kk,\|\cdot\|)\) is called \emph{spherically complete} if every decreasing sequence of closed balls in \(\kk\) has a non-empty intersection.
    % \end{definition}

    % \section{Valuation fields}

\subsection{absolute values and completion}

    \begin{definition}\label{def:valuation_field}
        Let \(\kk\) be a field.
        An \emph{absolute value} on \(\kk\) is a function \(\|\cdot|:\kk\to\bbR_{\ge0}\) satisfying the following properties for all \(x,y\in\kk\):
        \begin{enumerate}
            \item \(\|x\|=0\) if and only if \(x=0\);
            \item \(\|xy\|=\|x\|\cdot\|y\|\);
            \item \(\|x+y\|\leq\|x\|+\|y\|\).
        \end{enumerate}
        A field \(\kk\) equipped with an absolute value \(\|\cdot\|\) is called a \emph{valuation field}.
    \end{definition}

    \begin{remark}\label{rmk:additive_and_multiplicative_valuation_on_a_field}
        Let \(\kk\) be a field.
        Recall that a \emph{valuation} on \(\kk\) is a function \(v: \kk^\times \to \bbR\) such that
        \begin{itemize}
            \item \(\forall x,y \in \kk^\times, v(xy) = v(x) + v(y)\);
            \item \(\forall x,y \in \kk^\times, v(x + y) \geq \min\{v(x), v(y)\}\).
        \end{itemize}
        We can extend \(v\) to the whole field \(\kk\) by defining \(v(0) = +\infty\).
        Fix a real number \(\varepsilon \in (0,1)\).
        Then \(v\) induces an absolute value \(|\cdot|_v: \kk \to \bbR_+\) defined by \(|x|_v = \varepsilon^{v(x)}\) for each \(x \in \kk\).

        In some literature, the valuation \(v\) is called an \emph{additive valuation} and the induced absolute value \(|\cdot|_v\) is called a \emph{multiplicative valuation}.
        In this note, the term \emph{valuation} always refers to the additive valuation.
    \end{remark}

    \begin{definition}\label{def:complete_valuation_field}
        Let \((\kk,\|\cdot\|)\) be a valuation field.
        We say that \(\kk\) is \emph{complete} if the metric \(d(x,y) := \|x - y\|\) makes \(\kk\) a complete metric space.
    \end{definition}

    \begin{lemma}\label{lem:completion_of_valuation_field}
        Let \((\kk,\|\cdot\|)\) be a valuation field.
        Let \((\widehat{\kk},\|\cdot\|)\) be its completion as a metric space.
        Then the operations of addition and multiplication on \(\kk\) can be extended to \(\widehat{\kk}\) uniquely, making \((\widehat{\kk},\|\cdot\|)\) a complete valuation field containing \(\kk\) as a dense subfield.
    \end{lemma}

    % \begin{theorem}[Classification of absolute values on \(\bbQ\)]\label{thm:classification_absolute_values_Q}
    %     Let \(\|\cdot\|\) be an absolute value on \(\bbQ\).
    %     Then \(\|\cdot\|\) is either equivalent to the usual absolute value \(|\cdot|_\infty\) or to a \(p\)-adic absolute value \(|\cdot|_p\) for some prime \(p\).
    %     \Yang{The name is to be added.}
    % \end{theorem}

    \begin{definition}\label{def:spherically_complete}
        A valuation field \((\kk,\|\cdot\|)\) is called \emph{spherically complete} if every decreasing sequence of closed balls in \(\kk\) has a non-empty intersection.
    \end{definition}

\subsection{Non-archimedean fields and ultra-metric spaces}

    \begin{definition}\label{def:non-archimedean_fields}
        Let \((\kk,\|\cdot\|)\) be a valuation field.
        We say that \(\kk\) is \emph{non-archimedean} if its absolute value \(\|\cdot\|\) satisfies the \emph{strong triangle inequality}:
        \[ \|x+y\|\leq\max\{\|x\|,\|y\|\},\quad\forall x,y\in\kk. \]
        Otherwise, we say that \(\kk\) is \emph{archimedean}.
    \end{definition}

    Let \(\kk\) be a non-archimedean field.
    Then easily see that \(\{x \in \kk\colon \|x\|\leq 1\}\) is a subring of \(\kk\).
    Moreover, it is a local ring whose maximal ideal is \(\{x \in \kk\colon \|x\| < 1\}\).

    \begin{definition}\label{def:non-archimedean_field_ring_of_integers_maximal_ideal_and_residue_field}
        Let \(\kk\) be a non-archimedean field.
        The \emph{ring of integers} of \(\kk\) is defined as
        \[ \kk^\circ := \{x \in \kk\colon \|x\|\leq 1\}. \]
        Its maximal ideal is
        \[ \kk^{\circ\circ} := \{x \in \kk\colon \|x\| < 1\}. \]
        The \emph{residue field} of \(\kk\) is defined as
        \[ \calk_\kk := \widetilde{\kk} := \kk^\circ / \kk^{\circ\circ}. \]
    \end{definition}

    \Yang{Is the valuation on residue field trivial?}

    \begin{theorem}[Hessel's lemma]\label{thm:Hessel_lemma}
        Let \(\kk\) be a non-archimedean field and \(\calk_\kk\) be its residue field.
        For any polynomial \(\widetilde{f}(X) \in \calk_\kk[X]\) and any simple root \(\widetilde{a} \in \calk_\kk\) of \(\widetilde{f}(X)\), there exists a root \(a \in \kk^\circ\) of \(f(X) \in \kk^\circ[X]\) such that the image of \(a\) in \(\calk_\kk\) is \(\widetilde{a}\).
        \Yang{To be checked.}
    \end{theorem}

    % \begin{definition}\label{def:ultra-metric_space}
    %     A metric space \((X,d)\) is called an \emph{ultra-metric space} if its metric \(d\) satisfies the \emph{strong triangle inequality}:
    %     \[ d(x,z) \leq \max\{d(x,y), d(y,z)\},\quad\forall x,y,z\in X. \]
    % \end{definition}

    % \begin{proposition}\label{prop:balls_in_ultra-metric_space}
    %     Let \((X,d)\) be an ultra-metric space.
    %     Then for any \(x \in X\) and \(r > 0\), the closed ball \(B(x,r) := \{y \in X\colon d(x,y) \leq r\}\) satisfies the following properties:
    %     \begin{enumerate}
    %         \item For any \(y \in B(x,r)\), we have \(B(x,r) = B(y,r)\).
    %         \item Any two closed balls in \(X\) are either disjoint or one is contained in the other.
    %     \end{enumerate}
    %     \Yang{To be revised.}
    % \end{proposition}

    % We will use \(B(x,r)\) to denote the open ball with center \(x\) and radius \(r\).
    % We will use \(E(x,r)\) to denote the closed ball with center \(x\) and radius \(r\).

    % \begin{proposition}\label{prop:ultra-metric_space_is_totally_disconnected}
    %     Let \((X,d)\) be an ultra-metric space.
    %     Then \(X\) is totally disconnected, i.e., the only connected subsets of \(X\) are the singletons.
    %     \Yang{To be revised.}
    % \end{proposition}


\subsection[p-adic fields]{\(p\)-adic fields}

    % Recall the \(p\)-adic absolute value on \(\bbQ\) and its completion, the field \(\bbQ_p\) of \(p\)-adic numbers.
    
    \begin{construction}\label{constr:p-adic_absolute_value_on_number_field}
        Let \(K\) be a number field and \(\frakp\) be a prime ideal of the ring of integers \(\calO_K\) of \(K\).
        Considering the localization \((\calO_K)_\frakp\) of \(\calO_K\) at \(\frakp\), which is a discrete valuation ring, denote by \(v_\frakp: K^\times \to \bbZ\) the corresponding discrete valuation.
        The \emph{\(p\)-adic absolute value} on \(K\) associated to \(\frakp\) is defined as
        \[ |x|_\frakp := N(\frakp)^{-v_\frakp(x)},\quad\forall x \in K, \]
        where \(N(\frakp) := \#(\calO_K / \frakp)\) is the norm of \(\frakp\).

        The completion of \(K\) with respect to the \(p\)-adic absolute value \(|\cdot|_\frakp\) is denoted by \(K_\frakp\), called the \emph{\(\frakp\)-adic field}.
    \end{construction}

    One can just focus on the case \(K = \bbQ\) and \(\frakp = (p)\) for a prime number \(p\).

    \begin{example}\label{eg:p-adic_field}
        Let \(p\) be a prime number. 
        For every \(r \in \bbQ\), we can write \(r\) as \(r = p^n \frac{a}{b}\), where \(n \in \bbZ\) and \(a,b \in \bbZ\) are integers not divisible by \(p\).
        The \emph{\(p\)-adic absolute value} on \(\bbQ\) is defined as
        \[ |r|_p := p^{-n}. \]
      
        The \(p\)-adic field \(\bbQ_p\) can be described concretely as follows:
        \[ \bbQ_p = \left\{ \sum_{i = n}^{+\infty} a_i p^i \middle| n \in \bbZ, a_i \in \{0, 1, \ldots, p-1\} \right\}. \]
        For \(x = \sum_{i = n}^{+\infty} a_i p^i \in \bbQ_p\) with \(a_n \neq 0\), its \(p\)-adic absolute value is given by \(|x|_p = p^{-n}\).
        The operations of addition and multiplication on \(\bbQ_p\) are defined similarly as those on decimal expansions.
    \end{example}

    \begin{construction}\label{constr:p-adic_complex_number}
        Let \(p\) be a prime number.
        The field \(\bbC_p\) of \emph{\(p\)-adic complex numbers} is defined as the completion of the algebraic closure of \(\bbQ_p\) with respect to the unique extension of the \(p\)-adic absolute value \(|\cdot|_p\) on \(\bbQ_p\).
        The field \(\bbC_p\) is algebraically closed and complete with respect to \(|\cdot|_p\).
        \Yang{To be completed.}
    \end{construction}

    \begin{proposition}\label{prop:p-adic_complex_number_is_not_spherically_complete}
        The field \(\bbC_p\) of \(p\)-adic complex numbers is not spherically complete.
    \end{proposition}

    \begin{construction}\label{constr:spherically_complete_p-adic_fields}
        Let \(p\) be a prime number.
        \Yang{We construct the \emph{spherically complete \(p\)-adic field} \(\Omega_p\).}
        \Yang{To be completed.}
    \end{construction}

    \Yang{What is the relation between the finite extension of \(\bbQ_p\) and \(K_\frakp\)?}
