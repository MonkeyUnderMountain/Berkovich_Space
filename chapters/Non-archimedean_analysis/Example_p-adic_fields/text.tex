\section[Example: p-adic fields]{Example: \(p\)-adic fields}

\subsection[p-adic fields]{\(p\)-adic fields}

    \begin{construction}\label{constr:p-adic_absolute_value_on_number_field}
        Let \(K\) be a number field and \(\frakp\) be a prime ideal of the ring of integers \(\calO_K\) of \(K\).
        Considering the localization \((\calO_K)_\frakp\) of \(\calO_K\) at \(\frakp\), which is a discrete valuation ring, denote by \(v_\frakp: K^\times \to \bbZ\) the corresponding discrete valuation.
        The \emph{\(p\)-adic absolute value} on \(K\) associated to \(\frakp\) is defined as
        \[ |x|_\frakp := N(\frakp)^{-v_\frakp(x)},\quad\forall x \in K, \]
        where \(N(\frakp) := \#(\calO_K / \frakp)\) is the norm of \(\frakp\).

        The completion of \(K\) with respect to the \(p\)-adic absolute value \(|\cdot|_\frakp\) is denoted by \(K_\frakp\), called the \emph{\(\frakp\)-adic field}.
    \end{construction}

    One can just focus on the case \(K = \bbQ\) and \(\frakp = (p)\) for a prime number \(p\).

    \begin{example}\label{eg:p-adic_field}
        Let \(p\) be a prime number. 
        For every \(r \in \bbQ\), we can write \(r\) as \(r = p^n \frac{a}{b}\), where \(n \in \bbZ\) and \(a,b \in \bbZ\) are integers not divisible by \(p\).
        The \emph{\(p\)-adic absolute value} on \(\bbQ\) is defined as
        \[ |r|_p := p^{-n}. \]
      
        The \(p\)-adic field \(\bbQ_p\) can be described concretely as follows:
        \[ \bbQ_p = \left\{ \sum_{i = n}^{+\infty} a_i p^i \middle| n \in \bbZ, a_i \in \{0, 1, \ldots, p-1\} \right\}. \]
        For \(x = \sum_{i = n}^{+\infty} a_i p^i \in \bbQ_p\) with \(a_n \neq 0\), its \(p\)-adic absolute value is given by \(|x|_p = p^{-n}\).
        The operations of addition and multiplication on \(\bbQ_p\) are defined similarly as those on decimal expansions.
    \end{example}

    \begin{construction}\label{constr:p-adic_complex_number}
        Let \(p\) be a prime number.
        The field \(\bbC_p\) of \emph{\(p\)-adic complex numbers} is defined as the completion of the algebraic closure of \(\bbQ_p\) with respect to the unique extension of the \(p\)-adic absolute value \(|\cdot|_p\) on \(\bbQ_p\).
        The field \(\bbC_p\) is algebraically closed and complete with respect to \(|\cdot|_p\).
        \Yang{To be completed.}
    \end{construction}

    \begin{proposition}\label{prop:p-adic_complex_number_is_not_spherically_complete}
        The field \(\bbC_p\) of \(p\)-adic complex numbers is not spherically complete.
    \end{proposition}

    \begin{construction}\label{constr:spherically_complete_p-adic_fields}
        Let \(p\) be a prime number.
        \Yang{We construct the \emph{spherically complete \(p\)-adic field} \(\Omega_p\).}
        \Yang{To be completed.}
    \end{construction}

    \Yang{What is the relation between the finite extension of \(\bbQ_p\) and \(K_\frakp\)?}

\subsection[p-adic complex numbers]{\(p\)-adic complex numbers}
