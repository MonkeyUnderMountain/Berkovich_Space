\section{Semi-normed Rings and Modules}


\subsection{Semi-normed algebraic structures}

    \begin{definition}\label{def:semi-normed_abelian_group_rings_modules_and_algebras}
        Let \(G\) be an abelian group.
        A \emph{semi-norm} on \(G\) is a function \(\|\cdot\|: G \to \bbR_{\geq 0}\) such that
        \begin{itemize}
            \item \(\|0\| = 0\);
            \item \(\forall x,y \in G, \|x + y\| \leq \|x\| + \|y\|\).
        \end{itemize}
        Suppose that \(R\) is a ring (commutative with unity) and \(\|\cdot\|\) is a semi-norm on the underlying abelian group of \(R\).
        We further require that
        \begin{itemize}
            \item \(\|1\| = 1\);
            \item \(\forall x,y \in R, \|xy\| \leq \|x\|\|y\|\).
        \end{itemize}
        Suppose that \((M,\|\cdot\|_M)\) is an \(R\)-module and \(\|\cdot\|_M\) is a semi-norm on the underlying abelian group of \(M\).
        We further require that 
        \begin{itemize}
            \item \(\forall a \in R, x \in M, \|ax\|_M \leq \|a\| \|x\|_M\).
        \end{itemize}
        Suppose that \((A, \|\cdot\|_A)\) is an \(R\)-algebra and \(\|\cdot\|_A\) is a semi-norm on the underlying \(R\)-module of \(A\).
        We further require that this semi-norm is a semi-norm on the underlying ring of \(A\).
    \end{definition}

    \begin{definition}\label{def:norm_on_algebraic_structure}
        Let \(A\) be an abelian group (or ring, \(R\)-module, \(R\)-algebra) equipped with a semi-norm \(\|\cdot\|\).
        If \(\forall x \in A, \|x\| = 0 \iff x = 0\), then we say \(\|\cdot\|\) is a \emph{norm}.
    \end{definition}

    \Yang{Note that this definition of semi-normed module is a little different of \cite{Ber90}}

    \begin{definition}\label{def:bounded_semi-norm}
        Let \(\|\cdot\|_1\) and \(\|\cdot\|_2\) be two semi-norms on an abelian group (or ring, \(R\)-module, \(R\)-algebra) \(A\).
        We say \(\|\cdot\|_1\) is \emph{bounded} by \(\|\cdot\|_2\) if there exists a constant \(C > 0\) such that \(\forall x \in A, \|x\|_1 \leq C\|x\|_2\).
        If \(\|\cdot\|_1\) and \(\|\cdot\|_2\) are bounded by each other, we say they are \emph{equivalent}.
    \end{definition}

    \begin{remark}\label{rmk:semi-norms_bounded_by_each_other_induces_the_same_topologies}
        Equivalent semi-norms induce the same topology on \(A\).
    \end{remark}

    \begin{definition}\label{def:residue_semi-norm}
        Let \(M\) be a semi-normed abelian group (or ring, \(R\)-module, \(R\)-algebra) and \(N \subseteq M\) be a subgroup (or ideal, \(R\)-submodule, ideal).
        The \emph{residue semi-norm} on the quotient group \(M/N\) is defined as
        \[
            \|x + N\|_{M/N} = \inf_{y \in N} \|x + y\|_M.
        \]
    \end{definition}

    \Yang{Is this always a semi-norm? In particular, \(\|1\| = 1\)?}

    Unless otherwise specified, we always equip the quotient \(M/N\) with the residue semi-norm.

    \begin{remark}\label{rmk:residue_semi-norm}
        The residue semi-norm is a norm if and only if \(N\) is closed in \(M\).
    \end{remark}

    \begin{definition}\label{def:bounded_and_admissible_homomorphism}
        Let \(M\) and \(N\) be two semi-normed abelian groups (or rings, \(R\)-modules, \(R\)-algebras).
        A homomorphism \(f: M \to N\) is called \emph{bounded} if there exists a constant \(C > 0\) such that \(\forall x \in M, \|f(x)\|_N \leq C\|x\|_M\).
        
        A bounded homomorphism \(f: M \to N\) is called \emph{admissible} if the induced isomorphism \(M/\ker f \to \Image f\) is an isometry, i.e., \(\forall x \in M, \|f(x)\|_N = \|x\|_{M/\ker f}\).
    \end{definition}

    % \begin{definition}\label{def:semi-normed_rings}
    %     Let \(R\) be a ring (commutative with unity).
    %     A \emph{semi-norm} on \(R\) is a semi-norm \(\|\cdot\|\) on the underlying abelian group of \(R\) such that \(\forall x,y \in R, \|xy\| \leq \|x\|\|y\|\) and \(\|1\| = 1\).
    %     A \emph{semi-normed ring} is a ring equipped with a semi-norm.
    % \end{definition}

    \begin{definition}\label{def:multiplicative_and_power_multiplicative_semi-norm}
        A semi-norm \(\|\cdot\|\) on a ring \(R\) is called \emph{multiplicative} if \(\forall x,y \in R, \|xy\| = \|x\|\|y\|\).
        It is called \emph{power-multiplicative} if \(\forall x \in R, \|x^n\| = \|x\|^n\) for all integers \(n \geq 1\).
        % A power-multiplicative semi-norm is also called \emph{uniform}.
    \end{definition}

    % \begin{definition}\label{def:semi-normed_modules}
    %     Let \((R, \|\cdot\|_R)\) be a normed ring.
    %     A \emph{semi-normed \(R\)-module} is a pair \((M, \|\cdot\|_M)\) where \(M\) is an \(R\)-module and \(\|\cdot\|_M\) is a semi-norm on the underlying abelian group of \(M\) such that there exists \(C > 0\) with \(\forall a \in R, x \in M, \|ax\|_M \leq C \|a\|_R \|x\|_M\).
    % \end{definition}

    % One can talk about boundedness, admissibility and residue semi-norms in the contexts of semi-normed rings and semi-normed modules similar to those in semi-normed abelian groups.


    % \Yang{To be continued...}

\subsection{Banach rings}

    \begin{definition}\label{def:complete_semi-normed_abelian_group}
        A (semi-)norm on an abelian group \(M\) induces a (pseudo-)metric \(d(x,y) = \|x - y\|\) on \(M\).
        A (semi-)normed abelian group \(M\) is called \emph{complete} if it is complete as a (pseudo-)metric space.
    \end{definition}

    \begin{definition}\label{def:banach_ring}
        A \emph{banach ring} is a complete normed ring.
    \end{definition}

    \Yang{The counterpart of prime ideal is multiplicative seminorm}.

    \begin{definition}\label{def:completion_of_normed_algebraic_structures}
        Let \((A, \|\cdot\|_A)\) be a (semi-)normed algebraic structure, e.g., a (semi-)normed abelian group, a (semi-)normed ring, or a (semi-)normed module.
        The \emph{completion} of \(A\), denoted by \(\widehat{A}\), is the completion of \(A\) as a (pseudo-)metric space.
        Since \(A\) is dense in its completion, the algebraic operations and (semi-)norms on \(A\) can be uniquely extended to the completion.
        % \Yang{To be continued.}
    \end{definition}
    
    Let \(R\) be a normed ring and \(M,N\) be semi-normed \(R\)-modules.
    There is a natural semi-norm on the tensor product \(M \otimes_R N\) defined as
    \[
        \|z\|_{M \otimes_R N} = \inf \left\{ \sum_{i} \|x_i\|_M \|y_i\|_N : z = \sum_i x_i \otimes y_i, x_i \in M, y_i \in N \right\}.
    \]

    \begin{definition}\label{def:complete_tensor_product}
        Let \(R\) be a complete normed ring and \(M,N\) complete semi-normed \(R\)-modules.
        The \emph{complete tensor product} \(M \widehat{\otimes}_R N\) is defined as the completion of the semi-normed \(R\)-module \(M \otimes_R N\).
    \end{definition}

    \begin{definition}\label{def:spectral_radius_on_banach_rings}
        Let \(R\) be a banach ring.
        For each \(f \in R\), the \emph{spectral radius} of \(f\) is defined as
        \[
            \rho(f) = \lim_{n \to \infty} \|f^n\|^{1/n}.
        \]
    \end{definition}

    \Yang{Since , \(\rho(f)\) exists.}

    \begin{definition}\label{def:uniform_banach_ring}
        A banach ring \(R\) is called \emph{uniform} if its norm is power-multiplicative.
    \end{definition}

    \begin{proposition}\label{prop:spectral_radius_defines_a_power-multiplicative_semi-norm}
        Let \((R,\|\cdot\|)\) be a banach ring.
        The spectral radius \(\rho(\cdot)\) defines a power-multiplicative semi-norm on \(R\) that is bounded by \(\|\cdot\|\).
    \end{proposition}
    \begin{proof}
        \Yang{To be continued.}
    \end{proof}

    \begin{definition}\label{def:quasi_nilpotent_element}
        Let \(R\) be a banach ring.
        An element \(f \in R\) is called \emph{quasi-nilpotent} if \(\rho(f) = 0\).
        All quasi-nilpotent elements of \(R\) form an ideal, denoted by \(\Qnil(R)\).
    \end{definition}

    \begin{definition}\label{def:uniformization_of_banach_rings}
        Let \(R\) be a banach ring.
        The \emph{uniformization} of \(R\), denoted by \(R \to R^u\), is the banach ring with the universal property among all bounded homomorphisms from \(R\) to uniform banach rings.
        \Yang{To be continued.}
    \end{definition}

    \begin{proposition}\label{prop:the_uniformization_of_banach_rings_given_by_spectral_radius}
        Let \(R\) be a banach ring.
        The completion of \(R/\Qnil(R)\) with respect to the spectral radius \(\rho(\cdot)\) is the uniformization of \(R\).
    \end{proposition}
    \begin{proof}
        \Yang{To be continued.}
    \end{proof}

    % \Yang{To be continued...}


\subsection{Complete tensor product}


\subsection{Examples}

    \begin{example}\label{eg:trivial_normed_rings}
        Let \(R\) be arbitrary ring.
        The \emph{trivial norm} on \(R\) is defined as \(\|x\| = 0\) if \(x = 0\) and \(\|x\| = 1\) if \(x \neq 0\).
        The ring \(R\) equipped with the trivial norm is a normed ring.
    \end{example}

    \begin{example}\label{eg:C_and_R_as_complete_fields}
        The fields \(\bbC\) and \(\bbR\) equipped with the usual absolute value are complete fields.
    \end{example}

    \begin{example}\label{eg:p-adic_fields_as_complete_fields}
        The field \(\bbQ_p\) of \(p\)-adic numbers equipped with the \(p\)-adic norm is a complete non-Archimedean field.
    \end{example}

    \begin{example}\label{eg:ring_of_absolutely_convergent_power_series_as_banach_rings}
        Let \(R\) be a banach ring and \(r > 0\) be a real number.
        We define the ring of absolutely convergent power series over \(\kk\) with radius \(r\) as
        \[ R\left<T/r\right> \coloneqq \left\{\sum_{n=0}^{\infty} a_n T^n \in R[[T]] : \sum_{n=0}^{\infty} \|a_n\| r^n < \infty \right\}. \]
        Equipped with the norm \(\|\sum_{n=0}^{\infty} a_n T^n\| = \sum_{n=0}^{\infty} \|a_n\| r^n\), the ring \(R\left<T/r\right>\) is a banach ring.

        When \(R = \kk\) is a 
        \Yang{To be checked.}
    \end{example}

    \begin{example}\label{eg:Tate_algebras_with_Gauss_norm_as_banach_algebras}
        Let \(\kk\) be a non-Archimedean complete field.
        We define
        \[ \kk\{T_1/r_1, \ldots, T_n/r_n\} \coloneqq \left\{\sum_{I \in \bbN^n} a_I T^I \in \kk[[T_1, \ldots, T_n]] : \lim_{|I| \to \infty} |a_I| r^I = 0 \right\}, \]
        where \(r = (r_1, \ldots, r_n)\) is an n-tuple of positive real numbers, \(T^I = T_1^{i_1} \cdots T_n^{i_n}\) for \(I = (i_1, \ldots, i_n)\), and \(|I| = i_1 + \cdots + i_n\).
        Equipped with the norm \(\|\sum_{I \in \bbN^n} a_I T^I\| = \sup_{I \in \bbN^n} |a_I| r^I\), the affinoid \(\kk\)-algebra \(\kk\{T_1/r_1, \ldots, T_n/r_n\}\) is a banach \(\kk\)-algebra.
        This is called the \emph{Tate algebra} over \(\kk\) with polyradius \(r\) equipped with the \emph{Gauss norm}.
        We will denote \(\kk\{\underline{T/r}\} = \kk\{T_1/r_1, \ldots, T_n/r_n\}\) for simplicity.
    \end{example}
    \Yang{To be continued...}