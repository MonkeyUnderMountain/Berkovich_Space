\section{Semi-normed Rings and Modules}


\subsection{Semi-normed abelian groups}

    \begin{definition}\label{def:semi-normed_abelian_group}
        Let \(M\) be an abelian group.
        A \emph{semi-norm} on \(M\) is a function \(\|\cdot\|: M \to \bbR_+\) such that
        \begin{itemize}
            \item \(\|0\| = 0\);
            \item \(\forall x,y \in M, \|x + y\| \leq \|x\| + \|y\|\).
        \end{itemize}
        If we further have \(\|x\| = 0 \iff x = 0\), then we say \(\|\cdot\|\) is a \emph{norm}.
        A \emph{semi-normed abelian group} (resp. \emph{normed abelian group}) is an abelian group equipped with a semi-norm (resp. norm).
    \end{definition}

    \begin{definition}\label{def:complete_semi-normed_abelian_group}
        A semi-norm (resp. norm) on an abelian group \(M\) induces a pseudo-metric (resp. metric) \(d(x,y) = \|x - y\|\) on \(M\).
        A semi-normed (resp. normed) abelian group \(M\) is called \emph{complete} if it is complete as a pseudo-metric (resp. metric) space.
    \end{definition}

    \begin{definition}\label{def:bounded_semi-norm}
        Let \(\|\cdot\|_1\) and \(\|\cdot\|_2\) be two semi-norms on an abelian group \(M\).
        We say \(\|\cdot\|_1\) is \emph{bounded} by \(\|\cdot\|_2\) if there exists a constant \(C > 0\) such that \(\forall x \in M, \|x\|_1 \leq C\|x\|_2\).
    \end{definition}

    \begin{remark}\label{rmk:semi-norms_bounded_by_each_other_induces_the_same_topologies}
        If two semi-norms (resp. norms) on an abelian group \(M\) are bounded by each other, then they induce the same topology on \(M\).
    \end{remark}

    \begin{definition}\label{def:residue_semi-norm}
        Let \(M\) be a semi-normed abelian group and \(N \subseteq M\) be a subgroup.
        The \emph{residue semi-norm} on the quotient group \(M/N\) is defined as
        \[
            \|x + N\|_{M/N} = \inf_{y \in N} \|x + y\|_M.
        \]
    \end{definition}

    \begin{remark}\label{rmk:residue_semi-norm}
        The residue semi-norm is a norm if and only if \(N\) is closed in \(M\).
    \end{remark}

    \begin{definition}\label{def:bounded_and_admissible_homomorphism}
        Let \(M\) and \(N\) be two semi-normed abelian groups.
        A group homomorphism \(f: M \to N\) is called \emph{bounded} if there exists a constant \(C > 0\) such that \(\forall x \in M, \|f(x)\|_N \leq C\|x\|_M\).
        A bounded homomorphism \(f: M \to N\) is called \emph{admissible} if the induced isomorphism \(M/\ker f \to \im f\) is an isometry, i.e., \(\forall x \in M, \|f(x)\|_N = \inf_{y \in \ker f} \|x + y\|_M\).
    \end{definition}

    \Yang{To be continued...}